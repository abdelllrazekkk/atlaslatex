%%%%%%%%%%%%%%%%%%%%%%%%%%%%%%%%%%%%%%%%%%%
%%%              ttgamma                    %%%
%%%%%%%%%%%%%%%%%%%%%%%%%%%%%%%%%%%%%%%%%%%
\subsection[\ttgamma production]{\ttgamma production}
\label{subsec:ttgamma}

This section describes the MC samples used for the modelling of \ttgamma\ production.
Section~\ref{subsubsec:ttgamma_aMCP8} describes the \MGNLOPY[8] samples,
and Section~\ref{subsubsec:ttgamma_aMCH7} describes the \MGNLOHER[7] samples.

\subsubsection[MadGraph5\_aMC@NLO+Pythia8]{\MGNLOPY[8]}
\label{subsubsec:ttgamma_aMCP8}

\paragraph{Samples}
%\label{par:ttgamma_aMCP8_samples}

The descriptions below correspond to the samples in Tables~\ref{tab:ttgamma_aMCP8} and \ref{tab:ttgamma_aMCP8_addRad}.

\begin{table}[htbp]
  \caption{Nominal \ttgamma\ samples produced with \MGNLOPY[8].}%
  \label{tab:ttgamma_aMCP8}
  \centering
  \begin{tabular}{l l}
    \toprule
    DSID range & Description \\
    \midrule
    410389 & \ttgamma, non-all-hadronic \\
    410394 & \ttgamma, all-hadronic \\
    \bottomrule
  \end{tabular}
\end{table}

\begin{table}[htbp]
  \caption{\ttgamma\ samples produced with \MGNLOPY[8] used to estimate initial-state radiation systematic uncertainties.}%
  \label{tab:ttgamma_aMCP8_addRad}
  \centering
  \begin{tabular}{l l}
    \toprule
    DSID range & Description \\
    \midrule
    410404 & \ttgamma\, non-all-hadronic, A14Var3c up \\
    410405 & \ttgamma\, non-all-hadronic, A14Var3c down \\
    410410 & \ttgamma\, all-hadronic, A14Var3c up \\
    410411 & \ttgamma\, all-hadronic, A14Var3c down \\
    \bottomrule
  \end{tabular}
\end{table}

\paragraph{Short description:}

The production of \ttgamma events was modelled using the \MGNLO[2.3.3]~\cite{Alwall:2014hca}
generator at LO with the \NNPDF[2.3lo]~\cite{Ball:2014uwa} parton distribution function~(PDF).
The events were interfaced with \PYTHIA[8.212]~\cite{Sjostrand:2014zea} using the A14 tune~\cite{ATL-PHYS-PUB-2014-021} and the
\NNPDF[2.3lo]~\cite{Ball:2014uwa} PDF set.
The decays of bottom and charm hadrons were simulated using the \EVTGEN[1.6.0] program~\cite{Lange:2001uf}.

The uncertainty due to initial-state radiation (ISR) was estimated by
comparing the nominal \ttgamma sample with two additional samples,
where the Var3c up/down variations of the A14 tune were employed.


\paragraph{Long description:}

The \ttgamma sample was simulated as a 2$\to$7 process at LO including the decay of the top quarks by
\MGNLO[2.3.3]~\cite{Alwall:2014hca} with the \NNPDF[2.3lo]~\cite{Ball:2014uwa} parton distribution function~(PDF), interfaced with
\PYTHIA[8.212]~\cite{Sjostrand:2014zea}, using the A14 set of tuned parameters~\cite{ATL-PHYS-PUB-2014-021} and the
\NNPDF[2.3lo]~\cite{Ball:2014uwa} PDF set. The photon could be radiated from an initial charged parton, an intermediate top quark, 
or any of the charged final-state particles. The top-quark mass, top-quark decay width, $W$-boson decay width, 
and fine structure constant were set to 172.5~\GeV, 1.320~\GeV, 2.085~\GeV, and 1/137, respectively.
The five-flavour scheme was used, where all the quark masses are set to zero, except for the top quark. The renormalisation and the
factorisation scales were set to $0.5\times \sum_i \sqrt{m^2_i+p^2_{\text{T},i}}$, where the sum runs over all the particles generated 
from the matrix element calculation.
The decays of bottom and charm hadrons were simulated using the \EVTGEN[1.6.0] program~\cite{Lange:2001uf}.

The cross-section was calculated at NLO in QCD as reported in Ref.~\cite{Melnikov:2011ta}, resulting in a $K$-factor of 1.24 which was applied 
to the samples, with a relative uncertainty of 14\% from variations of renormalisation and factorisation scales as well 
as the choice of PDF set.

The uncertainty due to initial-state radiation (ISR) was estimated by comparing the nominal \ttV\ sample with two additional samples,
 which had the same settings as the nominal one, but employed the Var3c up or down variation of the A14 tune, which
corresponds to the variation of \alphas for initial-state radiation (ISR) in the A14 tune.

To evaluate the effect of renormalisation and factorisation scale uncertainties, the two scales were varied simultaneously by factors 2.0 and 0.5.
To evaluate the PDF uncertainties for the nominal PDF, the 100 replicas for \NNPDF[2.3lo] were taken into account. 


\subsubsection[MadGraph5\_aMC@NLO+Herwig7]{\MGNLOHER[7]}
\label{subsubsec:ttgamma_aMCH7}

\paragraph{Samples}
%\label{par:ttgamma_aMCH7_samples}

The descriptions below correspond to the samples in Table~\ref{tab:ttgamma_aMCH7}.

\begin{table}[htbp]
  \caption{\ttgamma\ samples produced with \MGNLOHER[7].}%
  \label{tab:ttgamma_aMCH7}
  \centering
  \begin{tabular}{l l}
    \toprule
    DSID range & Description \\
    \midrule
    410395 & \ttgamma non-all-hadronic \\
    410396 & \ttgamma all-hadronic \\
    \bottomrule
  \end{tabular}
\end{table}

\paragraph{Short description:}

Additional \ttgamma samples were produced with the parton shower of the nominal samples replaced by 
\HERWIG[7.04]~\cite{Bahr:2008pv,Bellm:2015jjp} to evaluate the impact of using using a different parton shower and hadronisation model.
The H7UE set of tuned parameters~\cite{Bellm:2015jjp} and the \MMHT[lo] PDF set~\cite{Harland-Lang:2014zoa} were used.
