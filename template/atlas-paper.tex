%-------------------------------------------------------------------------------
% This file provides a skeleton ATLAS paper.
%-------------------------------------------------------------------------------
% \pdfoutput=1
% The \pdfoutput command is needed by arXiv/JHEP/JINST to ensure use of pdflatex.
% It should be included in the first 5 lines of the file.
\pdfinclusioncopyfonts=1
% This command may be needed in order to get \ell in PDF plots to appear. Found in
% https://tex.stackexchange.com/questions/322010/pdflatex-glyph-undefined-symbols-disappear-from-included-pdf
%-------------------------------------------------------------------------------
% Specify where ATLAS LaTeX style files can be found.
\RequirePackage{latex/atlaslatexpath}
% Comment out the above line if the files are in a central location, e.g. $HOME/texmf.
%-------------------------------------------------------------------------------
\documentclass[PAPER, atlasdraft=true, texlive=2020, UKenglish]{atlasdoc}
% The language of the document must be set: usually UKenglish or USenglish.
% british and american also work!
% Commonly used options:
%  atlasdraft=true|false This document is an ATLAS draft.
%  texlive=YYYY          Specify TeX Live version (2020 is default).
%  coverpage             Create ATLAS draft cover page for collaboration circulation.
%                        See atlas-draft-cover.tex for a list of variables that should be defined.
%  cernpreprint          Create front page for a CERN preprint.
%                        See atlas-preprint-cover.tex for a list of variables that should be defined.
%  NOTE                  The document is an ATLAS note (draft).
%  PAPER                 The document is an ATLAS paper (draft).
%  CONF                  The document is a CONF note (draft).
%  PUB                   The document is a PUB note (draft).
%  BOOK                  The document is of book form, like an LOI or TDR (draft).
%  txfonts=true|false    Use txfonts rather than the default newtx.
%  paper=a4|letter       Set paper size to A4 (default) or letter.

%-------------------------------------------------------------------------------
% Extra packages:
\usepackage{atlaspackage}
% Commonly used options:
%  biblatex=true|false   Use biblatex (default) or bibtex for the bibliography.
%  backend=bibtex        Use the bibtex backend rather than biber.
%  subfigure|subfig|subcaption  to use one of these packages for figures in figures.
%  minimal               Minimal set of packages.
%  default               Standard set of packages.
%  full                  Full set of packages.
%-------------------------------------------------------------------------------
% Style file with biblatex options for ATLAS documents.
\usepackage{atlasbiblatex}

% Useful macros
\usepackage{atlasphysics}
% See doc/atlas_physics.pdf for a list of the defined symbols.
% Default options are:
%   true:  journal, misc, particle, unit, xref
%   false: BSM, hepparticle, hepprocess, hion, jetetmiss, math, process,
%          other, snippets, texmf
% See the package for details on the options.

% Macro to add to-do notes (for several authors). Uses the todonotes package.
% \ATLnote{JS}{Jane}{green!20}{green!50!black!60}
% add macros \JSnote and \JSinote for notes in the margin and inline.
% The first colour is for the body and the second for the border of the note.
% Set output=false in order not to print out the notes.
% Set shift=false to avoid adjustment of margins.
% This package should not be used in the final version of the paper.
\ifthenelse{\boolean{AtlasDraft}}{%
  \usepackage[output=true, shift=true]{atlastodo}
}{}

% Files with references for use with biblatex.
% Note that biber gives an error if it finds empty bib files.
% \addbibresource{atlas-document.bib}
\addbibresource{bib/ATLAS.bib}
\addbibresource{bib/CMS.bib}
\addbibresource{bib/ConfNotes.bib}
\addbibresource{bib/PubNotes.bib}

% Paths for figures - do not forget the / at the end of the directory name.
\graphicspath{{logos/}{figures/}}

% Add your own definitions here (file atlas-document-defs.sty).
\usepackage{atlas-document-defs}

%-------------------------------------------------------------------------------
% Generic document information.
%-------------------------------------------------------------------------------

% Title, abstract and document.
%-------------------------------------------------------------------------------
% This file contains the title, author and abstract.
% It also contains all relevant document numbers used by the different cover pages.
%-------------------------------------------------------------------------------

% Title
\title{Bare bones ATLAS document}

% Author --  Default is ``The ATLAS collaboration''
%\author{The ATLAS Collaboration}

% Authors and list of contributors to the analysis
% \AtlasAuthorContribute also adds the name to the author list
\usepackage{style/atlascontribute}
% Use authblk package if there are multiple authors, which is included by style/atlascontribute
% \usepackage{authblk}
% \renewcommand\Authands{, } % avoid ``. and'' for last author
% \renewcommand\Affilfont{\itshape\small} % affiliation formatting
\AtlasAuthorContributor{First AtlasAuthorContributor}{a}{Author's contribution.}
\AtlasAuthorContributor{Second AtlasAuthorContributor}{b}{Author's contribution.}
\AtlasAuthorContributor{Third AtlasAuthorContributor}{a}{Author's contribution.}
\AtlasContributor{Fourth AtlasContributor}{Contribution to the analysis.}
%\author[a]{First Author}
%\author[a]{Second Author}
%\author[b]{Third Author}
\affil[a]{One Institution}
\affil[b]{Another Institution}

% Date: if not given, uses current date
%\date{\today}

% Draft version:
% Should be 1.0 for the first circulation, and 2.0 for the second circulation.
% If given, adds draft version on front page, a 'DRAFT' box on top of each other page, 
% and line numbers.
% Comment or remove in final version.
\AtlasVersion{0.1}

% ATLAS reference code, to help ATLAS members to locate the paper
\AtlasRefCode{GROUP-2014-XX}

% ATLAS note number. Can be an COM, INT, PUB or CONF note
% \AtlasNote{ATLAS-CONF-2014-XXX}
% \AtlasNote{ATLAS-PUB-2014-XXX}
\AtlasNote{ATL-COM-PHYS-2014-XXX}

% Submission journal
% \AtlasJournal{Phys.\ Lett.\ B.}

% Abstract - % directly after { is important for correct indentation
\AtlasAbstract{%
	This is a bare bones ATLAS document. Put the abstract for the document here.
}

%-------------------------------------------------------------------------------
% The following information is needed for the cover page. The commands are only defined
% if you use the coverpage option in atlasdoc or use the atlascover package
%-------------------------------------------------------------------------------

% List of supporting notes  (leave as null \AtlasCoverSupportingNote{} if you want to skip this option)
% \AtlasCoverSupportingNote{Short title note 1}{https://cds.cern.ch/record/XXXXXXX}
% \AtlasCoverSupportingNote{Short title note 2}{https://cds.cern.ch/record/YYYYYYY}
%
% OR (the 2nd option is deprecated, especially for CONF and PUB notes)
%
% Supporting material TWiki page  (leave as null \AtlasCoverTwikiURL{} if you want to skip this option)
% \AtlasCoverTwikiURL{https://twiki.cern.ch/twiki/bin/view/Atlas/WebHome}

% Comment deadline
% \AtlasCoverCommentsDeadline{DD Month 2014}

% Analysis team members -- Indicate the Contact Editors (usually two) by (*) after their name
% \AtlasCoverAnalysisTeam{Peter Analyser, Contact Editor~(*), Contact Editor2~(*), Alphonse Physicien}

% Editorial Board Members --- Indicate the Chair by a (*) after his/her name
% Give either all members at once (then they appear on one line), or separately
% \AtlasCoverEdBoardMember{EdBoard~Chair~(*), EB~Member~1, EB~Member~2, EB~Member~3}
% \AtlasCoverEdBoardMember{EdBoard~Chair~(*)}
% \AtlasCoverEdBoardMember{EB~Member~1}
% \AtlasCoverEdBoardMember{EB~Member~2}
% \AtlasCoverEdBoardMember{EB~Member~3}

% A PUB note has readers and not an EdBoard -- give their names here (one line or several entries)
% \AtlasCoverReaderMember{Reader~1, Reader~2}
% \AtlasCoverReaderMember{Reader~1}
% \AtlasCoverEdBoardMember{Reader~2}

% Editors egroup
% \AtlasCoverEgroupEditors{atlas-GROUP-2014-XX-editors@cern.ch}

% EdBoard egroup
% \AtlasCoverEgroupEdBoard{atlas-GROUP-2014-XX-editorial-board@cern.ch}


% Author and title for the PDF file.
\hypersetup{pdftitle={ATLAS document},pdfauthor={The ATLAS Collaboration}}

%-------------------------------------------------------------------------------
% Content
%-------------------------------------------------------------------------------
\begin{document}

\maketitle

\tableofcontents

% List of to-do notes.
% \listoftodos

%-------------------------------------------------------------------------------
\section{Introduction}
\label{sec:intro}
%-------------------------------------------------------------------------------

Place your introduction here.

%-------------------------------------------------------------------------------
\section{ATLAS detector}
\label{sec:detector}
%-------------------------------------------------------------------------------

The ATLAS detector~\cite{PERF-2007-01} ...
% % Footnote with ATLAS coordinate system
\newcommand{\AtlasCoordFootnote}{%
ATLAS uses a right-handed coordinate system with its origin at the nominal interaction point (IP)
in the centre of the detector and the $z$-axis along the beam pipe.
The $x$-axis points from the IP to the centre of the LHC ring,
and the $y$-axis points upwards.
Cylindrical coordinates $(r,\phi)$ are used in the transverse plane, 
$\phi$ being the azimuthal angle around the beam pipe.
The pseudorapidity is defined in terms of the polar angle $\theta$ as $\eta = -\ln \tan(\theta/2)$.
Angular distance is measured in units of $\Delta R \equiv \sqrt{\Delta\eta^{2} + \Delta\phi^{2}}$.}

%-------------------------------------------------------------------------------
\subsection{ATLAS detector}
\label{sec:atlas1}
%-------------------------------------------------------------------------------

The ATLAS experiment~\cite{atlas-detector} at the LHC is a multi-purpose particle detector
with a forward-backward symmetric cylindrical geometry and a near $4\pi$ coverage in 
solid angle\footnote{\AtlasCoordFootnote}.
It consists of an inner tracking detector surrounded by a thin superconducting solenoid
providing a \SI{2}{\tesla} axial magnetic field, electromagnetic and hadron calorimeters, and a muon spectrometer.
The inner tracking detector covers the pseudorapidity range $|\eta| < 2.5$.
It consists of silicon pixel, silicon micro-strip, and transition radiation tracking detectors.
Lead/liquid-argon (LAr) sampling calorimeters provide electromagnetic (EM) energy measurements
with high granularity.
A hadron (iron/scintillator-tile) calorimeter covers the central pseudorapidity range ($|\eta| < 1.7$).
The end-cap and forward regions are instrumented with LAr calorimeters
for both EM and hadronic energy measurements up to $|\eta| = 4.9$.
The muon spectrometer surrounds the calorimeters and is based on
three large air-core toroid superconducting magnets with eight coils each.
Its bending power is in the range from \num{2.0} to \SI{7.5}{\tesla\metre}.
It includes a system of precision tracking chambers and fast detectors for triggering.
A three-level trigger system is used to select events.
The first-level trigger is implemented in hardware and uses a subset of the detector information
to reduce the accepted rate to at most \SI{75}{\kilo\hertz}.
This is followed by two software-based trigger levels that
together reduce the accepted event rate to \SI{400}{\hertz} on average
depending on the data-taking conditions during 2012.


%-------------------------------------------------------------------------------
\subsection{ATLAS coordinate system footnote}
\label{sec:coord}
%-------------------------------------------------------------------------------



%-------------------------------------------------------------------------------
\section{Analysis}
\label{sec:analysis}
%-------------------------------------------------------------------------------

You can find some text snippets that can be used in papers in \texttt{latex/atlassnippets.sty}.
To use them, provide the \texttt{snippets} option to \texttt{atlasphysics}.

%-------------------------------------------------------------------------------
\section{Results}
\label{sec:result}
%-------------------------------------------------------------------------------

Place your results here.

% All figures and tables should appear before the summary and conclusion.
% The package placeins provides the macro \FloatBarrier to achieve this.
% \FloatBarrier

%-------------------------------------------------------------------------------
\section{Conclusion}
\label{sec:conclusion}
%-------------------------------------------------------------------------------

Place your conclusion here.

%-------------------------------------------------------------------------------
\section*{Acknowledgements}
%-------------------------------------------------------------------------------

% % Acknowledgements for papers with collision data
% Version 27-Jan-2020

% Standard acknowledgements start here
%----------------------------------------------

We thank CERN for the very successful operation of the LHC, as well as the
support staff from our institutions without whom ATLAS could not be
operated efficiently.

We acknowledge the support of ANPCyT, Argentina; YerPhI, Armenia; ARC, Australia; BMWFW and FWF, Austria; ANAS, Azerbaijan; SSTC, Belarus; CNPq and FAPESP, Brazil; NSERC, NRC and CFI, Canada; CERN; CONICYT, Chile; CAS, MOST and NSFC, China; COLCIENCIAS, Colombia; MSMT CR, MPO CR and VSC CR, Czech Republic; DNRF and DNSRC, Denmark; IN2P3-CNRS and CEA-DRF/IRFU, France; SRNSFG, Georgia; BMBF, HGF and MPG, Germany; GSRT, Greece; RGC and Hong Kong SAR, China; ISF and Benoziyo Center, Israel; INFN, Italy; MEXT and JSPS, Japan; CNRST, Morocco; NWO, Netherlands; RCN, Norway; MNiSW and NCN, Poland; FCT, Portugal; MNE/IFA, Romania; MES of Russia and NRC KI, Russia Federation; JINR; MESTD, Serbia; MSSR, Slovakia; ARRS and MIZ\v{S}, Slovenia; DST/NRF, South Africa; MINECO, Spain; SRC and Wallenberg Foundation, Sweden; SERI, SNSF and Cantons of Bern and Geneva, Switzerland; MOST, Taiwan; TAEK, Turkey; STFC, United Kingdom; DOE and NSF, United States of America. In addition, individual groups and members have received support from BCKDF, CANARIE, Compute Canada and CRC, Canada; ERC, ERDF, Horizon 2020, Marie Sk{\l}odowska-Curie Actions and COST, European Union; Investissements d'Avenir Labex, Investissements d'Avenir Idex and ANR, France; DFG and AvH Foundation, Germany; Herakleitos, Thales and Aristeia programmes co-financed by EU-ESF and the Greek NSRF, Greece; BSF-NSF and GIF, Israel; CERCA Programme Generalitat de Catalunya and PROMETEO Programme Generalitat Valenciana, Spain; G\"{o}ran Gustafssons Stiftelse, Sweden; The Royal Society and Leverhulme Trust, United Kingdom.

The crucial computing support from all WLCG partners is acknowledged gratefully, in particular from CERN, the ATLAS Tier-1 facilities at TRIUMF (Canada), NDGF (Denmark, Norway, Sweden), CC-IN2P3 (France), KIT/GridKA (Germany), INFN-CNAF (Italy), NL-T1 (Netherlands), PIC (Spain), ASGC (Taiwan), RAL (UK) and BNL (USA), the Tier-2 facilities worldwide and large non-WLCG resource providers. Major contributors of computing resources are listed in Ref.~\cite{ATL-GEN-PUB-2016-002}.

%----------------------------------------------
% Created with Glance <Atlas.Glance@cern.ch>


The \texttt{atlaslatex} package contains the acknowledgements that were valid
at the time of the release you are using.
These can be found in the \texttt{acknowledgements} subdirectory.
When your ATLAS paper or PUB/CONF note is ready to be published,
the acknowledgements should be automatically updated by PO-GitLab.
The acknowledgements can be inspected on the web page:
\url{https://atlas-glance.cern.ch/atlas/membership/funding_agencies/history}.

%-------------------------------------------------------------------------------
\clearpage
\appendix
\part*{Appendix}
\addcontentsline{toc}{part}{Appendix}
%-------------------------------------------------------------------------------

In a paper, an appendix is used for technical details that would otherwise disturb the flow of the paper.
Such an appendix should be printed before the Bibliography.

%-------------------------------------------------------------------------------
% If you use biblatex and either biber or bibtex to process the bibliography
% just say \printbibliography here.
\printbibliography
% If you want to use the traditional BibTeX you need to use the syntax below.
% \bibliographystyle{obsolete/bst/atlasBibStyleWoTitle}
% \bibliography{atlas-document,bib/ATLAS,bib/CMS,bib/ConfNotes,bib/PubNotes}
%-------------------------------------------------------------------------------

%-------------------------------------------------------------------------------
% Author list - comment in this line when you are ready to include it.
% \clearpage
% \input{atlas_authlist}
%-------------------------------------------------------------------------------

%-------------------------------------------------------------------------------
% Auxiliary material - comment out the following line if you do not have any.
\include{atlas-document-auxmat}
%-------------------------------------------------------------------------------

%-------------------------------------------------------------------------------
% Extra tables etc. for HepData - comment in the following line if you have any.
% \include{atlas-document-hepdata}
%-------------------------------------------------------------------------------

\end{document}
