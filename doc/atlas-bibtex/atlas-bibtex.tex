%-------------------------------------------------------------------------------
% Examples on how to fill BibTeX entries in the ATLAS Bibliographic Style
% Responsible: Ian Brock (Ian.Brock@cern.ch)
%-------------------------------------------------------------------------------
\documentclass[UKenglish]{style/atlasdoc}

% Use biblatex and biber for the bibliography
\usepackage[minimal,biblatex]{style/atlaspackage}
\usepackage{authblk}
\usepackage{style/atlasphysics}

% Files with references in BibTeX format
\addbibresource{atlas-bibtex.bib}
\addbibresource{atlas-biblatex.bib}

\graphicspath{{../../figures/}}

\newcommand*{\BibTeX}{Bib\TeX}
\newcommand{\File}[1]{\texttt{#1}\xspace}
\newcommand{\Macro}[1]{\texttt{\textbackslash #1}\xspace}
\newcommand{\Option}[1]{\textsf{#1}\xspace}
\newcommand{\Package}[1]{\texttt{#1}\xspace}

%-------------------------------------------------------------------------------
% Generic document information
%-------------------------------------------------------------------------------

% Set author and title for the PDF file
\hypersetup{pdftitle={ATLAS BibTeX guide},pdfauthor={Ian Brock}}

\title{Quick guide to ATLAS \BibTeX\ style}

\author{Ian C. Brock}
\affil{University of Bonn}

\AtlasAbstract{%
	This document illustrates how to use \BibTeX\ for the bibliography of your ATLAS paper.
	Two \BibTeX (\texttt{.bst}) style files have been created that can be used with any of the ATLAS supported journals,
	depending on whether they require the title of the references to be included or not.

	This document was generated using version \ATPackageVersion\ of the ATLAS \LaTeX\ package.
}


%-------------------------------------------------------------------------------
\begin{document} 

%\maketitle

%-------------------------------------------------------------------------------
\section{Instructions}
%-------------------------------------------------------------------------------

The ATLAS Collaboration has specific guidelines as to what constitutes a good bibliographic style. 
For example, a reference to a paper by an LHC Collaboration must not include the first author whereas if the paper is by any other collaboration, it should. 
Also, where available, links to the \texttt{arXiv} entries of the papers must be included. 
To help authors with their paper preparations, a standard ATLAS bibliographic style has been developed which incorporates all of these requirements, 
and, at the same time, is compatible with those of the journals the papers are being submitted to. 

The format of the references in your ATLAS paper depends on the journal to which you are submitting,
but in general we can classify the journal styles in two categories: those which require the title of the references and those which do not. 
To ensure the homogeneity in all ATLAS publications, 
\BibTeX style files are provided for each of these categories along with an example file that illustrates how different types of bibliographic material should be referenced.
Authors must choose between these two style files, depending on the journal to which they wish to submit their paper.

\textcolor{red}{Important:} Please use these files and base your \texttt{.bib} file on the example provided,
as it has the references in the style preferred by the ATLAS Publications Committee.
This will definitely save time in the reviewing process!

These style files have been successfully tested in the framework provided by each of the journals listed in the following sections and with the standard ATLAS document template.

A new implementation of \BibTeX\ is provided by the \texttt{biblatex}~\cite{biblatex} package.
It is planned to move all ATLAS documents to the use of this package.
One major advantage of the package is that it defines quite a few more entry types
that are much more suitable for online documents and things like CONF and PUB notes.
It is also possible to use UTF-8 encoding in the entries, which means that letters such as
ä, é, ß can be included directly in the text.
Adjustment of the style is also much simpler.
It is possible to take a base style and then just apply changes to it rather than
having to learn the details of how \texttt{bst} files are constructed.

\emph{Note that such adjustments still have to be made to the style used in the current documentation.}

You compile a document using \BibTeX\ using the commands:\\
\indent \textbf{PDFLaTeX + BibTeX + PDFLaTeX + PDFLaTeX} 

\noindent For \texttt{biblatex} and \texttt{biber} use:\\
\indent \textbf{PDFLaTeX + BibTeX + PDFLaTeX + PDFLaTeX} 

You can of course use \LaTeX\ rather than PDF\LaTeX, but PDF\LaTeX\ is preferred,
as things like clicking on cross-references and links to publications in the bibliography
works much more reliably with PDF\LaTeX.


%-------------------------------------------------------------------------------
\section{Journals that include the title in the reference}
\label{sec:withtitle}
%-------------------------------------------------------------------------------

Journals:
\begin{itemize}
\item \textbf{JHEP}
\item \textbf{JINST}
\item \textbf{NJP}
\end{itemize}

\noindent BibTeX style file: \\
\indent \textbf{atlasBibStyleWithTitle.bst}

\noindent BibTeX example bibliography file: \\
\indent \textbf{atlasBibStyleExample.bib}

\noindent Include at the end of your \textbf{.tex} file the following lines: \\
\indent \texttt{\textbackslash bibliographystyle\{atlasBibStyleWithTitle\}} \\
\indent \texttt{\textbackslash bibliography\{atlas-bibtex\}}

\noindent To compile: \\
\indent \textbf{LaTeX + BibTeX + LaTeX + LaTeX} 


%-------------------------------------------------------------------------------
\section{Journals that do not include the title in the reference}
\label{sec:wotitle}
%-------------------------------------------------------------------------------

Journals: \\
\indent \textbf{EPJC} \\
\indent \textbf{NPB} \\
\indent \textbf{PLB} \\
\indent \textbf{PRD} \\
\indent \textbf{PRL}

\noindent BibTeX style file: \\
\indent \textbf{atlasBibStyleWoTitle.bst}

\noindent BibTeX example bibliography file: \\
\indent \textbf{atlasBibStyleExample.bib}

\noindent Include at the end of your \textbf{.tex} file the following lines: \\
\indent \texttt{\textbackslash bibliographystyle\{atlasBibStyleWoTitle\}}
\indent \texttt{\textbackslash bibliography\{atlas-bibtex\}}


%-------------------------------------------------------------------------------
\section{ATLAS Notes}
%-------------------------------------------------------------------------------

For ATLAS Notes, the recommended style file is: \\
\indent \textbf{atlasBibStyleWoTitle.bst}.


%-------------------------------------------------------------------------------
\section{\BibTeX tips}
%-------------------------------------------------------------------------------

\begin{itemize}
\item A bibliographic item is created in the \emph{.bib} file as: \\
	\texttt{@Article\{lhcCollaboration:2012 \\ \emph{\% bibliographic information}}
	
	The identifier directly after the document type declaration is how we are going to refer to this item inside the main \texttt{.tex} file:
	\texttt{An LHC Collaboration published a paper\~ \textbackslash cite\{lhcCollaboration:2012\}
	with very interesting result.}
\item When referencing ATLAS CONF notes, the url to the CDS page should be included. For this to work, in the preamble of your \emph{.tex} document add: \\
	\texttt{$\backslash$usepackage\{hyperref\}}
\item If the DOI is filled and the \texttt{hyperref} package loaded, the title of the journal will be highlighted in blue and become a hyperlink to the online paper.
\item If you use \BibTeX\ and want to cite multiple references in the format [A-Z], include the following package in the 	header of your document: \\
	\texttt{$\backslash$usepackage\{cite\}}.
	If you use \texttt{biblatex} the option \texttt{style=numeric-comp} does this for you.
\item When referencing papers from journals like PRD, PLB, etc.,
	one has to be careful not to include the ``D'' or ``B'' as part of the volume but rather in the journal name. 
	Macros have been added to the \textbf{.bst} style files for these journals. Please use these.
\end{itemize}


%-------------------------------------------------------------------------------
\section{Examples}
%-------------------------------------------------------------------------------

\begin{itemize}
	\item LHC Collaboration~\cite{lhcCollaboration:2012}
	\item Other Collaboration~\cite{otherCollaboration:2007}
	\item Individual authors~\cite{authors:2008}
	\item arXiv only~\cite{arxivOnly:2009}
	\item arXiv only submitted to a journal~\cite{arxivSub:2011}
	\item ATLAS CONF Note~\cite{atlasConf:2012} 
\end{itemize}

While the \texttt{collaboration} field is a nice idea, it is not supported by many \BibTeX\ styles.
Hence in \cite{lhcCollaboration:2012}, \texttt{collaboration} has been renamed to \texttt{author} and
the \texttt{author} field has been renamed as \texttt{xauthor}. If you use \texttt{collaboration} and omit
\texttt{author} you will get a warning when you run \texttt{bibtex}. This you see in \cite{atlasConf:2012}.

Note that in Ref.~\cite{atlasConf:2012} the entry type \texttt{@Article} is used and the field \texttt{journal} 
is abused for the conference note number. This is a result of the \BibTeX resrictions on the entry types.
\texttt{biblatex} provides a lot more entry types. It is planned to move to \texttt{biblatex} or the ATLAS templates
in the course of 2014.


%-------------------------------------------------------------------------------
\section*{History}
%-------------------------------------------------------------------------------

\begin{description}
  \item[2013-08-13: Cristina Oropeza Barrera] First version of the document released.
  \item[2014-08-14: Ian Brock] Updated the example references a bit and gave a bit more background information.
\end{description}

%-------------------------------------------------------------------------------
% Print bibliography using biblatex
\printbibliography
%-------------------------------------------------------------------------------
% Old style bibtex bibliography
% \bibliographystyle{../../bibtex/bst/atlasBibStyleWithTitle}
% \bibliography{atlas-bibtex,atlas-biblatex}

\end{document}
