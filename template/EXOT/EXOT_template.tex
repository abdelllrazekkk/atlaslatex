%-------------------------------------------------------------------------------
% This file provides template EXOT group object descriptions and cuts.
% \pdfinclusioncopyfonts=1
% This command may be needed in order to get \ell in PDF plots to appear. Found in
% https://tex.stackexchange.com/questions/322010/pdflatex-glyph-undefined-symbols-disappear-from-included-pdf
%-------------------------------------------------------------------------------
% Specify where ATLAS LaTeX style files can be found.
\newcommand*{\ATLASLATEXPATH}{../../latex/}
% Use this variant if the files are in a central location, e.g. $HOME/texmf.
% \newcommand*{\ATLASLATEXPATH}{}
%-------------------------------------------------------------------------------
\documentclass[NOTE, atlasdraft=true, texlive=2016, USenglish]{\ATLASLATEXPATH atlasdoc}
% The language of the document must be set: usually UKenglish or USenglish.
% british and american also work!
% Commonly used options:
%  atlasdraft=true|false This document is an ATLAS draft.
%  texlive=YYYY          Specify TeX Live version (2016 is default).
%  txfonts=true|false    Use txfonts rather than the default newtx
%  paper=a4|letter       Set paper size to A4 (default) or letter.

%-------------------------------------------------------------------------------
% Extra packages:
\usepackage{\ATLASLATEXPATH atlaspackage}
% Commonly used options:
%  subfigure|subfig|subcaption  to use one of these packages for figures in figures.
%-------------------------------------------------------------------------------
\usepackage{multirow}

% Useful macros
\usepackage[jetetmiss]{\ATLASLATEXPATH atlasphysics}
% See doc/atlas_physics.pdf for a list of the defined symbols.
% Default options are:
%   true:  journal, misc, particle, unit, xref
%   false: BSM, heppparticle, hepprocess, hion, jetetmiss, math, process,
%          other, snippets, texmf
% See the package for details on the options.

% Add you own definitions here (file atlas-document-defs.sty).
% \usepackage{atlas-document-defs}

% Paths for figures - do not forget the / at the end of the directory name.
\graphicspath{{\ATLASLATEXPATH ../logos/}{figures/}}

%-------------------------------------------------------------------------------
% Generic document information
%-------------------------------------------------------------------------------

\AtlasTitle{EXOT group text snippets for INT notes}
\AtlasVersion{0.1}
\author{ATLAS EXOT Group}
\AtlasRefCode{EXOT-2018-XX}
\AtlasAbstract{%
  This note contains text snippets and tables that should be included in supporting notes
  from the EXOT group.

  The templates are in American English.
  If wanted, some adaption to British English could be made. 

  This document was generated using version \ATPackageVersion\ of the ATLAS \LaTeX\ package.

  \emph{2018-10-23: This file is a work in progress (WIP) and will probably be updated.
  Backwards incompatible changes may be made as the examples develop.}
}
% Author and title for the PDF file
\hypersetup{pdftitle={ATLAS EXOT supporting note},pdfauthor={ATLAS EXOT group}}

%-------------------------------------------------------------------------------
% Main document
%-------------------------------------------------------------------------------
\begin{document}

\maketitle

\tableofcontents

\section{Introduction}
\textit{This is a very brief, almost ``abstract-like'' section.  Immediately following is the 
Executive Summary, which should include all of the components that are sometimes in an 
introduction but they are organized in a way that will facilitate review by conveners
since they are in a standard way}


\section{Executive Summary}
\textit{This section, ideally 2-pages (max), should be placed at the beginning of the internal 
note following the more conventional introduction.  It should be split as highlighted here and should 
give a high-level overview of the analysis including (but not limited to):}

\begin{itemize}
\item \textit{Motivation, physics target, and the general characteristics of the signal}
\item \textit{Analysis strategy}
\item \textit{General characteristics of the control, validation, and signal regions}
\item \textit{Background estimation strategy overview}
\item \textit{Highlight major or most important points of the analysis}
\item \textit{Team overview task list including a list of all critical tasks, who is responsible for each task, and what else they are working on outside of this analysis.  This should be presented in the format shown in Table~\ref{tab:Miles_Ahead}.}
\item \textit{List of outstanding items in the analysis that still need to be addressed}
\end{itemize}

\section{Executive Summary}

This section, ideally less than two pages, should be placed at the beginning of the supporting note.
It should give a high-level overview of the analysis including (but not limited to):
\begin{itemize}
\item physics target and the general characteristics of the signal;
\item analysis strategy;
\item general characteristics of the control, validation, and signal regions;
\item a background estimation strategy overview;
\item highlights of major or most important points of the analysis;
\item a table or list of all critical tasks and who is responsible for each.
\end{itemize}

This section should include explicit pointers to the items required for the PAR and FAR.  For reference, those items are listed below. Please feel free to modify these lists into references to document sections.

For a PAR (ed board request), the SUSY group looks for:
\begin{itemize}
\item A definition of the target scenarios
\item A brief run-down of the signal grids, in particular pointing to any production problems or places where the production has not yet begun
\item A discussion of any non-standard object definitions in the analysis, and any on-going development that might affect the object definition
\item A discussion of the derivations: whether any reprocessing is needed, and whether the required samples have been requested
\item Data/MC comparisons in some inclusive regions (demonstrating the technical ability to use the data from all periods in the team's framework)
\item Any signal region definitions that are available, with some preliminary optimization in place and some description of the optimization procedure
\item Plans for further optimization (e.g. optimization for different model space regions, or use of an MVA or multi-bin fit)
\item An outline of the plans to get from here to a paper (plan of work, noting if person power is insufficient for any of the areas), as a part of this summary
\end{itemize}

If you are using SUSYTools, please include the current configuration file in the \href{http://gitlab.cern.ch/atlas-phys-susy-wg/AnalysisSUSYToolsConfigurations}{archive}.

For a FAR, the SUSY group looks for:
\begin{itemize}
\item Definitions of SR, CR, VR, including expected yields with the targeted luminosity for all backgrounds that will be estimated with transfer factors or pure MC. Include signal contamination in CR and VR.
\item Cutflow for the background and for representative signal points.
\item Outline of background estimation strategies, including validation and closure tests for data-driven estimations. Statistical uncertainties for transfer-factor (TF) estimated and data-driven estimated backgrounds.
\item Comparison of data and MC with the relevant dataset in CR and VR. Strategy for mitigation of mis-modelling wherever needed with proof of feasibility.
\item Estimate of the detector level systematic uncertainties through propagation of CP recommendations.
\item A clear statement on how all others systematic uncertainties will be evaluated: all the procedures need to be defined before unblinding.
\item A clear plan for ``discovery regions'' as well as the statistical treatment of the signal regions.
\item Background only fit with pull plot for the nuisance parameters.
\item Estimated exclusion, including depth of exclusion within the normal exclusion contour, with Asimov fit.
\item To-do list for achieving final result and possible bottlenecks (can the rest of the SUSY WG help with anything?)
\item For analyses using machine learning methods, additional diagnostics are required, which are explained \href{https://twiki.cern.ch/twiki/bin/view/AtlasProtected/SusyMachineLearning}{here}.
\end{itemize}

The background forum recommends the following diagnostic plots:
\begin{itemize}
\item Standard occupancy maps and plots that can reveal detector or non-collision background issues - plot for CRs, VRs and SRs of MET phi, the leading jet and/or leading lepton eta vs. phi 2D map, and the leading jet and/or leading lepton phi distributions
\item Selection efficiencies as a function of mu (VR, SR, and CR): to check how dependent the analysis is on pileup (primarily for MC)
\item Run number and data period dependencies: plot lumi-normalized yields in CRs, VRs and SRs as a function of the run number and data period (data only). This is to check for potential temporary issues in the data present only for certain runs, and to reveal potential chunks of data not processed by mistake. You should normalize the per-run yield using the lumi as reported from an independent source (not the in-file metadata!), e.g. simply use \href{https://svnweb.cern.ch/trac/atlasoff/browser/PhysicsAnalysis/SUSYPhys/SUSYTools/trunk/scripts/ilumi2histo.py}{this script in SUSYTools} to build the luminosity-vs-run histogram from your iLumiCalc file.
\item In particular, for Full Run 2 dataset analyses, a plot of data from 2018 vs period, specifically to compare the efficiencies for the period with two dead tile modules to the periods without.
\item Check for missing data: Compare the total number of processed data events and compare to the reference numbers for the combination of GRL and derivation you're running over in \href{https://docs.google.com/spreadsheets/d/1LMioo0nvALkKgoCKRW_ihQThHyVwcOQVRJe4aXiENUs/edit#gid=424865170}{this spreadsheet}. If your total does not match this reference number, feel free to contact BG forum conveners and derivation contacts for help with debugging.
\item Check for duplicated events: several bugs in MC and DAOD production have caused duplicated events to appear in the derivations in the past. This can potentially happen in both MC and data. Please check that there are no duplicated events in your CRs, VRs or SRs. If you notice duplicated events in your derivations, please get in touch with the Background Forum conveners immediately.
\item Comparisons of data and MC in the CRs and VRs, as well as MC in the SRs, for 2015+2016, 2017, and 2018 separately, for key distributions and yields.
\item Debug stream yields in SR and CR. The full name of the debug stream is debugrec\_hlt, and derivation datasets can be found for both 2015 and 2016 data with a query like \tttext{rucio ls --short --filter type=container data*\_13TeV.00*.debugrec\_hlt*DAOD\_SUSY1*p2709*}.
\item Pileup reweighting check: plot the nvtx distribution before and after pileup reweighting (data Vs MC). Purpose: check that the pileup reweighting works as intended.
\end{itemize}

Again, if you are using SUSYTools, please include the current configuration file in the \href{http://gitlab.cern.ch/atlas-phys-susy-wg/AnalysisSUSYToolsConfigurations}{archive}.  Please also take care that you have looked into the \href{https://twiki.cern.ch/twiki/bin/viewauth/Atlas/DataPreparationCheckListForPhysicsAnalysis}{items recommended by DataPrep}.


\section{Data and MC}
\textit{Dataset used with blinding strategy, full list of background samples and details of the signal samples.}


\section{Object selection}
\textit{The supporting notes should now include the following standardized tables of properties: 
each analysis should simply fill them in by writing / replacing the value with the appropriate 
number or by choosing the appropriate option. The idea of these tables is to harmonize some sections 
of the supporting notes as to make review and analysis comparisons simpler.}

\textit{If you use non-standard selections which do not fit in these tables, this should of course 
be noted and discussed in more detail in the text.}

\textit{Object selection tables (following template) and detailed event selection: there may, of 
course, still be some minor open items, as long as they don’t significantly affect the analysis 
strategy, but these should be well defined and clearly indicated (e.g. coloured/bold) in the text 
in this section and in the list of outstanding tasks within the executive summary. Both should 
be updated as the analysis progresses.}
 
\subsection{Electron selection}

\begin{table}[ht]
  \caption{Electron selection criteria.}%
  \label{tab:object:electron}
  \centering
  % \resizebox{\textwidth}{!}{
  \begin{tabular}{ll}
    \toprule
    Feature & \multicolumn{1}{c}{Criterion} \\
    \midrule
    Pseudorapidity range & \(|\eta| <\) X\\
    Energy calibration & \texttt{es2017\_R21\_PRE} (ESModel)\\
    Energy & \(E > \qty[parse-numbers=false]{XX}{\GeV}\) \\
    Transverse energy & \(\ET > \qty[parse-numbers=false]{XX}{\GeV}\) \\
    Transverse momentum & \(\pT > \qty[parse-numbers=false]{XX}{\GeV}\) \\
    \midrule
    \multirow{2}{*}{Object quality} & Not from a bad calorimeter cluster (\texttt{BADCLUSELECTRON})\\ %\cline{2-2}
      & Remove clusters from regions with EMEC bad HV (2016 data only) \\
    \midrule
    \multirow{2}{*}{Track to vertex association} & \(|d_{0}^{\text{BL}}(\sigma)| < X\) \\ %\cline{2-2}
    & \(|\Delta z_{0}^{\text{BL}} \sin{\theta}| < \qty[parse-numbers=false]{X}{\mm}\) \\
    \midrule
    Identification & (\texttt{Loose/Medium/Tight}) \\
    Isolation & \texttt{LooseTrackOnly / Loose / Tight / Gradient / \ldots} \\
      \bottomrule
  \end{tabular}
  % }
\end{table}

Notes:
\begin{itemize}
\item Pseudorapidity: when the calorimeter crack is not excluded, the range can be indicated simply as \enquote{\(|\eta| < 2.47\)}, when the crack is excluded: \enquote{\((|\eta| < 1.37) \quad || \quad (1.52 < |\eta| < 2.47)\)}.
\item Usually only one among \enquote{Energy}, \enquote{Transverse energy} and \enquote{Transverse momentum} criteria is applied --- the \qty{30}{\GeV} value is just an example.
  In special cases energy (i.e.\ calorimeter-based measurement) and momentum (i.e.\ tracking-based measurement) criteria can be required in order to constraint different aspects of the reconstruction.
\item Electron ID\@: 3 working points (Loose/Medium/Tight) are evaluated using the Likelihood-based (LH) method, by the
  \href{https://twiki.cern.ch/twiki/bin/view/AtlasProtected/EGammaIdentificationRun2}{ElectronPhotonSelectorTools}.
\item Energy calibration of electrons is implemented in the\\
  \href{https://twiki.cern.ch/twiki/bin/view/AtlasProtected/ElectronPhotonFourMomentumCorrection}{ElectronPhotonFourMomentumCorrection} tool.
\item Scale Factors for efficiencies for electrons are implemented in the\\
  \href{https://twiki.cern.ch/twiki/bin/view/AtlasProtected/XAODElectronEfficiencyCorrectionTool}{ElectronEfficiencyCorrection} tool.
\item Updated configurations for the EGamma CP tools can be found on this \href{https://twiki.cern.ch/twiki/bin/view/AtlasProtected/EGammaRecommendationsR21}{TWiki} page.
\end{itemize}

\newpage

\subsection{Photon selection}

\begin{table}[ht]
  \caption{Photon selection criteria.}%
  \label{tab:object:photon} 
  \centering
  % \resizebox{\textwidth}{!}{
  \begin{tabular}{ll}
    \toprule
    Feature & \multicolumn{1}{c}{Criterion} \\
    \midrule
    Pseudorapidity range & \(|\eta| <\) X\\
    Energy calibration & \texttt{es2017\_R21\_PRE} (ESModel)\\
    Energy & \(E > \qty[parse-numbers=false]{XX}{\GeV}\) \\
    Transverse energy & \(\ET > \qty[parse-numbers=false]{XX}{\GeV}\) \\
    \midrule
    \multirow{2}{*}{Object quality} & Not from a bad calorimeter cluster (\texttt{BADCLUSELECTRON})\\ %\cline{2-2}
      & Remove clusters from regions with EMEC bad HV (2016 data only) \\
    \midrule
    Photon cleaning & \texttt{passOQquality} \\
    Fudging & Applied for Full sim / not for AtlFastII \\
    \midrule
    Identification & (\texttt{Loose/Tight}) \\
    Isolation &  \texttt{FixedCutTightCaloOnly / FixedCutTight / FixedCutLoose} \\
    \bottomrule
  \end{tabular}
  %  }
\end{table}

Notes:
\begin{itemize}
\item Pseudorapidity: please note that the maximum value for \(|\eta|\) for photon candidates (2.37) is smaller than for electron candidates (2.47). 
  If crack excluded: \enquote{\((|\eta| < 1.37) \quad || \quad (1.52 < |\eta| < 2.37)\)}.
\item Usually only one between \enquote{Energy} and \enquote{Transverse energy} criteria is applied --- the \qty{30}{\GeV} value is just an example.
\item Photon cleaning: a new Photon helper is available to apply the photon cleaning cut 
  (from the \texttt{ElectronPhotonSelectorTools}, tag \(\ge\) 00-02-92-21, release \(\ge\) 2.4.30).
\item Photon ID\@: 2 working points (Loose/Tight) are evaluated using a cut-based method, by the
  \href{https://twiki.cern.ch/twiki/bin/view/AtlasProtected/EGammaIdentificationRun2}{ElectronPhotonSelectorTools}.
\item Energy calibration of photons is implemented in the\\
  \href{https://twiki.cern.ch/twiki/bin/view/AtlasProtected/ElectronPhotonFourMomentumCorrection}{ElectronPhotonFourMomentumCorrection} tool.
\item Scale Factors for efficiencies for photons are implemented in the\\
  \href{https://twiki.cern.ch/twiki/bin/view/AtlasProtected/XAODElectronEfficiencyCorrectionTool}{ElectronEfficiencyCorrection} tool.
\item Updated configurations for the EGamma CP tools can be found on this \href{https://twiki.cern.ch/twiki/bin/view/AtlasProtected/EGammaRecommendationsR21}{TWiki} page.
\end{itemize}

\subsection{Muon selection}

\begin{table}[ht]
  \caption{Muon selection criteria.}%
  \label{tab:object:muon}
  \centering
  % \resizebox{\textwidth}{!}{
  \begin{tabular}[ht]{ll}
    \toprule
    Feature & Criterion \\
    \midrule
    Selection working point & \texttt{Loose/Medium/Tight /High-pT} \\
    Isolation working point & \texttt{LooseTrackOnly/Loose/Tight/Gradient/\ldots}\\
    Momentum calibration & Sagitta correction [used/not used] \\
    \pT Cut & \qty[parse-numbers=false]{X}{\GeV} \\
    \(|\eta|\) cut & \(< X\) \\
    \dzero significance cut & X \\
    \(z_{0}\) cut & \qty[parse-numbers=false]{X}{\mm} \\
    \bottomrule
  \end{tabular}
  % }
\end{table}

The selection criteria are implemented in the \texttt{MuonSelectorTools-XX-XX-XX}\\
with \texttt{MuonMomentumCorrections-XX-XX-XX}, 
isolation in \texttt{IsolationSelection-XX-XX-XX} and \dzero and \(z_{0}\) cuts in \texttt{xAODTracking-XX-XX-XX}.
The muon recommendations can be found in 
\href{https://twiki.cern.ch/twiki/bin/view/AtlasProtected/MCPAnalysisGuidelinesMC16}{MCPAnalysisGuidelinesMC16}.

\subsection{Tau selection}

\begin{table}[ht]
  \caption{Tau selection criteria.}%
  \label{tab:object:tau}
  \centering
  % \resizebox{\textwidth}{!}{
  \begin{tabular}{ll}
  \toprule
  Feature & Criterion \\
  \midrule
  Pseudorapidity range & \(|\eta| < X\) \\
  Track selection & 1 or 3 tracks \\\
  Charge & \(|Q| = 1\) \\
  Tau energy scale & \texttt{MVA TES}\\
  Transverse momentum & \(\pT > \qty[parse-numbers=false]{XX}{\GeV}\) \\
  Jet rejection & BDT-based (\texttt{Loose/Medium/Tight}) \\
  Electron rejection & BDT-based\\
  Muon rejection & Via overlap removal in \(\Delta R < 0.2\) and \(\pT > \qty{2}{\GeV}\).
    Muons must not be Calo-tagged\\
  \bottomrule
  \end{tabular}
  % }
\end{table}

If the crack is excluded: \((|\eta| < 1.37) || (1.52 < |\eta| < 2.5)\)

The selection criteria are all implemented in the \texttt{TauSelectionTool} as part of the \texttt{TauAnalysisTools}.
Documentation can be found in the \href{https://gitlab.cern.ch/atlas/athena/blob/21.2/PhysicsAnalysis/TauID/TauAnalysisTools/doc/README-TauSelectionTool.rst}{README-TauSelectionTool.rst}.

\subsection{Small-$R$ jet selection}

If you want to use variables such as \verb|\fcut| you need to add the option
\texttt{jetetmiss} to \texttt{atlaspackage}.

\begin{table}[ht]
  \caption{Jet reconstruction criteria.}%
  \label{tab:object:jet1}
  \centering
  % \resizebox{\textwidth}{!}{
  \begin{tabular}{ll}
  \toprule
  Feature & Criterion \\
  \midrule
  Algorithm & \Antikt  \\
  \(R\)-parameter & 0.4 \\
  Input constituent & EMTopo \\
  Analysis release number & 21.2.10 \\
  %Calibration tag & JetCalibTools-00-04-76 \\
  \texttt{CalibArea} tag & 00-04-81 \\
  Calibration configuration & \texttt{JES\_data2017\_2016\_2015\_Recommendation\_Feb2018\_rel21.config} \\
  Calibration sequence (Data) & \texttt{JetArea\_Residual\_EtaJES\_GSC\_Insitu} \\
  Calibration sequence (MC) & \texttt{JetArea\_Residual\_EtaJES\_GSC} \\
  %Calibration sequence (AFII) & \texttt{JetArea\_Residual\_EtaJES\_GSC} \\
  \midrule
  \multicolumn{2}{c}{Selection requirements} \\
  \midrule
  Observable & Requirement \\
  \midrule
  Jet cleaning & \texttt{LooseBad} \\
  BatMan cleaning & No \\
  \pT & \(> \SI[parse-numbers=false]{XX}{\GeV}\) \\
  \(|\eta|\) & \(< X\) \\
  JVT & (\emph{Update if needed}) \(>0.59\) for \(\pT < \SI{60}{\GeV}\), \(|\eta| < 0.4\)\\
  \bottomrule
  \end{tabular}
  % }
\end{table}


\clearpage
\subsection{Large-$R$ jet selection}

\begin{table}[ht]
  \caption{Large-\(R\) jet reconstruction criteria.}%
  \label{tab:object:jet2}
  \centering
  % \resizebox{\textwidth}{!}{
  \begin{tabular}{ll}
    \toprule
    Feature & Criterion \\ 
    \midrule
    Algorithm & \antikt  \\
    R-parameter & 1.0 \\
    Input constituent & \texttt{LCTopo} \\
    Grooming algorithm & Trimming \\ 
    \fcut & 0.05 \\
    \(R_{\text{trim}}\) & 0.2 \\
    Analysis release number & 21.2.10 \\
    %Calibration tag & JetCalibTools-00-04-76 \\
    \texttt{CalibArea} tag & 00-04-81 \\
    Calibration configuration & \texttt{JES\_MC16recommendation\_FatJet\_JMS\_comb\_19Jan2018.config} \\
    Calibration sequence (Data) & \texttt{EtaJES\_JMS\_Insitu} \\
    Calibration sequence (MC) & \texttt{EtaJES\_JMS} \\
    \bottomrule
    \multicolumn{2}{c}{Selection requirements} \\
    \midrule
    Observable & Requirement \\
    \midrule
    \pT  & \(> \SI[parse-numbers=false]{XX}{\GeV}\) \\
    \(|\eta|\) & \(< X\) \\
    Mass & \(> \SI[parse-numbers=false]{XX}{\GeV}\) \\
    \bottomrule
    \multicolumn{2}{c}{Boosted object tagger} \\
    \midrule
    Object  & Working point \\
    \midrule
    \(W\) / \(Z\) / top & 50\% / 80\% \\
    \(X\rightarrow bb\) & single/double \btag with/without loose/tight mass \\
    \bottomrule
  \end{tabular}
  % }
\end{table}


\subsection{\MET selection}

\begin{table}[ht]
  \caption{\MET reconstruction criteria.}%
  \label{tab:object:met}
  \centering
  \begin{tabular}{ll}
    \toprule
    Parameter & Value \\ 
    \midrule
    Algorithm & Calo-based \\
    Soft term & Track-based (TST) \\ 
    MET operating point & \texttt{Tight} \\
    Analysis release & 21.2.16 \\
    Calibration tag & \texttt{METUtilities-00-02-46} \\
    \bottomrule
    \multicolumn{2}{c}{Selection requirements} \\
    \midrule
    Observable & Requirement \\
    \midrule
    \MET & \(> \SI[parse-numbers=false]{XX}{\GeV}\) \\
    \(\sum{\ET} / \MET\)  & \(< X\) \\
    Object-based \MET significance & \(> X\) \\
    \bottomrule
  \end{tabular}
\end{table}



\subsection{Jet flavour tagging selection}


\begin{table}[ht]
\begin{center}
\resizebox{\textwidth}{!}{
\begin{tabular}{|l|c|}\hline
  \multicolumn{2}{|c|}{b-tagging selection} \\\hline

				& EM Topo Jets / Track jets / VR jets \\\hline\hline
Jet collection		& AntiKt4EMTopo / AntiKt2PV0 / AntiKtVR30Rmax4Rmin02 \\\hline
Jet selection 		& $\pT >$ X GeV    \\				            	& $|\eta| <$ Y \\				
                 & $JVT$ cut if applicable \\
                 \hline
Algorithm 		& MV2c10 / MV2c10mu / MV2c10rnn / DL1 / DL1mu /DL1rnn 	\\\hline
Operating point		& Hybrid /  Fixed \\
                     & Eff = 60 / 70 / 77 / 85 
   \\\hline                  
CDI                & 2017-21-13TeV-MC16-CDI-2017-12-22\_v1 \\\hline
\end{tabular}}
\end{center}
\label{tab:bstar}
\end{table}%


\subsection{Track selection}

If you use tracks as particular objects on which you cut in your analysis.

\begin{table}[ht]
  \caption{\texttt{TrackParticle} object selection criteria.}%
  \label{tab:object:track}
  \centering
  % \resizebox{\textwidth}{!}{
  \begin{tabular}{ll}
    \toprule
    Tracking algorithm								    & Primary / Large Radius Tracking / Custom \\
    Track quality selection (official)    & \texttt{Loose/Tight} \\
    \pT                                   & \(> \SI[parse-numbers=false]{XX}{\GeV}\) \\
    \(|\eta|\)                            & \(< X\) \\
    Track-vertex association criteria     & \texttt{Loose/Tight} \\
    Track-to-tet association method       & Ghost Matched / \(\Delta R\) \\
    \bottomrule
  \end{tabular}
  % }
\end{table}

\subsection{Overlap Removal}
The reconstruction of the same energy deposits as multiple objects is resolved using the standard overlap removal tools, AssociationUtils, documented \href{https://gitlab.cern.ch/atlas/athena/blob/21.2/PhysicsAnalysis/AnalysisCommon/AssociationUtils/README.rst}{here}

The (Standard/Heavy-flavor/Boosted/Boosted+Heavy-flavor/Lapton-favored) working point is used corresponding to:

\begin{table}[ht]
\begin{center}
\small
\resizebox{\textwidth}{!}{
\begin{tabular}{|c|c|c|}
\hline
 Reject & Against & Criteria \\\hline
 electron & electron & shared track, $p_{T,1} < p_{T,2}$ \\
 tau      & electron & $\Delta R <$ 0.2 \\
 tau      & muon     & $\Delta R <$ 0.2 \\
 muon     & electron & is calo-muon and shared ID track \\
 electron & muon     & shared ID track \\
 photon   & electron & $\Delta R <$ 0.4 \\
 photon   & muon     & $\Delta R <$ 0.4 \\
 jet      & electron & [$\Delta R <$ 0.2/Not a bjet and $\Delta R <$ 0.2] \\
 electron & jet      & [$\Delta R <$ 0.4/$\Delta R <$ min(0.4, 0.04 + 10GeV/ElePt)/None] \\
 jet      & muon     & [NumTrack $<$ 3 and (ghost-associated or $\Delta R <$ 0.2) \\
  && / Not a bjet and NumTrack $<$ 3 and (ghost-associated or $\Delta R <$ 0.2)] \\
 muon     & jet      & [$\Delta R <$ 0.4/$\Delta R <$ min(0.4, 0.04 + 10GeV/MuPt)/None] \\
 jet      & tau      & $\Delta R <$ 0.2 \\
 photon   & jet      & $\Delta R <$ 0.4 \\
 fat-jet  & electron & $\Delta R <$ 1.0 \\
 jet      & fat-jet  & $\Delta R <$ 1.0 \\
\hline
\end{tabular}}
\end{center}
\end{table}

$\Delta R$ is calculated using rapidity by default.





\section{Event selection}
\textit{The following items should also be filled in for the event selection.  There may, of course, 
still be some minor open items, as long as they don’t significantly affect the analysis strategy, 
but these should be well defined and clearly indicated (e.g. coloured/bold) in the text in this 
section and in the list of outstanding tasks within the executive summary. 
Both should be updated as the analysis progresses.}

\subsection{Event cleaning}
Following the \href{https://twiki.cern.ch/twiki/bin/viewauth/Atlas/DataPreparationCheckListForPhysicsAnalysis}{recommendations of the DataPrep group}, the following event-level requirements are made.

We use the official GRL\@:
 \begin{verbatim} FILL IN HERE \end{verbatim}
 
The following event-level vetos are made to reject bad / corrupt events:
 \begin{itemize}
  \item LAr noise burst and data corruption (\verb|xAOD::EventInfo::LAr|),
  \item Tile corrupted events (\verb|xAOD::EventInfo::Tile|),
  \item events affected by the SCT recovery procedure for single event upsets (\verb|xAOD::EventInfo::SCT|),
  \item incomplete events (\verb|xAOD::EventInfo::Core|).
 \end{itemize}
 
 Debug stream events [have/have not] been included.
 
 Checks [have/have not] been done to remove duplicate events.
 
Events are required to have a primary vertex with at least two associated tracks.
The primary vertex is selected as the one with the largest \(\Sigma \pT^2\),
where the sum is over all tracks with transverse momentum \(\pT > \qty{0.4}{\GeV}\) that are associated with the vertex.
 
 


\section{Background Modelling}
\textit{After outlining the object and event selection, noting possible outstanding points that still
need to be addressed to freeze the selection, you should demonstrate that you can analyze
the dataset that you intend to publish.  This should include CR/VR plots for
the main backgrounds with the full data (full run-2 analyses) or at least a representative
majority of the data (analyses during data-taking); for the more minor backgrounds this may
still be in progress but an outline of the planned method should be present.}


\section{Systematic Uncertainties}

\textit{Several systematics may still be missing but the note should include a proposed plan listing the CP systematics 
you will need to consider in this analysis (+ timescale on which they will be available if not already) and 
an outline of how the systematics on the backgrounds are proposed to be determined. If not statistics-limited, 
the most dominant systematic(s) should be present.}

Systematic uncertainties arise from the reconstruction of the various physics objects and 
from theoretical and/or modelling uncertainties affecting the predictions for both the backgrounds and signals. 
These uncertainties manifest themselves as uncertainties both in the overall yield and 
shape of the final observable.

\subsection{Experimental}

A summary of the experimental systematic uncertainties taken into account in this analysis is given in
Table~\ref{tab:syst_summary_sources}, along with the shorthand name of the uncertainty used throughout 
the analysis. 

\textit{Include also subsections for each of the individual descriptions of the uncertainty groups and the
source where they come from.  If the recommendation is not available at this time, state that in the section.}

\begin{table}[h]
	\centering
	\scriptsize
	\begin{center}
		\begin{tabular}{ll}
			\toprule \midrule
			Systematic uncertainty        & Short description                                                                          \\ \midrule
			\multicolumn{2}{c}{\textbf{Event}}  \\ \midrule
			Luminosity  & uncertainty on the total integrated luminosity  \\ \midrule
			\multicolumn{2}{c}{\textbf{Electrons}}  \\ \midrule
			EL\_EFF\_Trigger\_TOTAL\_1NPCOR\_PLUS\_UNCOR   & trigger efficiency uncertainty    \\
			EL\_EFF\_Reco\_TOTAL\_1NPCOR\_PLUS\_UNCOR    & reconstruction efficiency uncertainty          \\ 
			EL\_EFF\_ID\_TOTAL\_1NPCOR\_PLUS\_UNCOR    & ID efficiency uncertainty               \\ 
			EL\_EFF\_Iso\_TOTAL\_1NPCOR\_PLUS\_UNCOR    & isolation efficiency uncertainty       \\
			EG\_SCALE\_ALL & energy scale uncertainty  \\ 
			EG\_RESOLUTION\_ALL & energy resolution uncertainty  \\ \midrule
			\multicolumn{2}{c}{\textbf{Muons}}  \\ \midrule
			mu20\_iloose\_L1MU15\_OR\_HLT\_mu40\_MUON\_EFF\_Trig   & \multirow{ 2}{*}{trigger efficiency uncertainties (2 muon selection)}       \\ 
			mu24\_ivarmed\_OR\_HLT\_mu40\_MU\_EFF\_TrigStat        &         \\ 
			mu24\_ivarmed\_OR\_HLT\_mu50\_MU\_EFF\_TrigStat        &         \\ 
			mu26\_ivarmed\_OR\_HLT\_mu50\_MU\_EFF\_TrigStat        &         \\ 
			MUON\_EFF\_RECO\_STAT        & \multirow{ 2}{*}{reconstruction uncertainty for \pt > 15 GeV}        \\ % todo: guess that includes the ID uncertainty? PR.
			MUON\_EFF\_RECO\_SYS        &        \\ 
			MUON\_EFF\_RECO\_STAT\_LOWPT        & \multirow{ 2}{*}{reconstruction and ID efficiency uncertainty for \pt < 15 GeV}        \\ 
			MUON\_EFF\_RECO\_SYS \_LOWPT       &        \\ 
			MUON\_ISO\_STAT        & \multirow{ 2}{*}{isolation efficiency uncertainty}        \\ 
			MUON\_ISO\_SYS        &        \\ 
			MUON\_TTVA\_STAT        & \multirow{ 2}{*}{track-to-vertex association efficiency uncertainty}        \\ 
			MUON\_TTVA\_SYS        &        \\ 
			MUONS\_SCALE        & energy scale uncertainty                                                           \\ 
			MUONS\_SAGITTA\_RHO        & variations in the scale of the momentum (charge dependent)                  \\ 
			MUONS\_SAGITTA\_RESBIAS        & variations in the scale of the momentum (charge dependent)              \\ 
			MUONS\_ID        & energy resolution uncertainty from inner detector                              \\ 
			MUONS\_MS & energy resolution uncertainty from muon system  \\ \midrule
			\multicolumn{2}{c}{\textbf{Small-R Jets}}  \\ \midrule
			JET\_GroupedNP  & energy scale uncertainty split into 3 components              \\ 
			JET\_SR1\_JET\_EtaIntercalibration\_NonClosure & non-closure in the jet response at $2.4<|\eta|<2.5$   \\
			JET\_SR1\_JER\_SINGLE\_NP  & energy resolution uncertainty  \\
			JvtEfficiency & JVT efficiency uncertainty  \\
			%
			%
			% todo: decide on b-tagging EV prescription. So far we only have the simplest prescription, but we should switch.
			%
			%
			FT\_EFF\_EIGEN\_B & $b$-tagging efficiency uncertainties ("BTAG\_MEDIUM):   \\
			FT\_EFF\_EIGEN\_C & \\ % \multirow{2}{*}{3 components for $b$-jets, 4 for $c$-jets and 5 for light jets} \\
			FT\_EFF\_EIGEN\_L &   \\ 
			FT\_EFF\_EIGEN\_extrapolation &  $b$-tagging efficiency uncertainty on the extrapolation on high \pt-jets  \\ 
			FT\_EFF\_EIGEN\_extrapolation\_from\_charm & $b$-tagging efficiency uncertainty on $\tau$-jets   \\ \midrule
			\multicolumn{2}{c}{\textbf{Large-R Jets}}  \\ \midrule
			%
			%
			%
			FATJET\_JMR   & mass resolution uncertainty                                                                 \\ 
			FATJET\_JER    & energy resolution uncertainty                                                               \\ 
			JET\_Comb\_Baseline\_Kin &  \multirow{4}{*}{energy scale uncertainties (\pt and mass scales fully correlated)}  \\
			JET\_Comb\_Modelling\_Kin &  \\
			JET\_Comb\_TotalStat\_Kin &  \\
			JET\_Comb\_Tracking\_Kin   &  \\ 
			\midrule % TODO: which uncertainties do we need for large-R jets?
			% calo mass 
			%JET\_Medium\_JET\_Rtrk\_Baseline\_Kin &  \multirow{4}{*}{energy scale uncertainties (\pt and mass scales fully correlated)}  \\
			%JET\_Medium\_JET\_Rtrk\_Modelling\_Kin &  \\
			%JET\_Medium\_JET\_Rtrk\_TotalStat\_Kin &  \\
			%JET\_Medium\_JET\_Rtrk\_Tracking\_Kin   &  \\ \midrule
			%
			%
			% todo: decide on b-tagging EV prescription also for track jets. So far we only have the simplest prescription, but we should switch.
			%
			%
			\multicolumn{2}{c}{\textbf{Track-Jets}}  \\ \midrule
			FT\_EFF\_EIGEN\_B & $b$-tagging efficiency uncertainties ("BTAG\_MEDIUM):   \\
			FT\_EFF\_EIGEN\_C & \\ % \multirow{2}{*}{3 components for $b$-jets, 4 for $c$-jets and 5 for light jets} \\
			FT\_EFF\_EIGEN\_L &   \\ 
			FT\_EFF\_EIGEN\_extrapolation &  $b$-tagging efficiency uncertainty on the extrapolation on high \pt-jets  \\ 
			FT\_EFF\_EIGEN\_extrapolation\_from\_charm & $b$-tagging efficiency uncertainty on $\tau$-jets   \\ \midrule
			\multicolumn{2}{c}{\textbf{\MET-Trigger and \MET-Terms}}  \\ \midrule
			METTrigStat    & \multirow{2}{*}{trigger efficiency uncertainty} -  \\
			METTrigSyst &    \\
			MET\_SoftTrk\_ResoPerp           & track-based soft term related to transversal resolution uncertainty                  \\ 
			MET\_SoftTrk\_ResoPara           & track-based soft term related to longitudinal resolution uncertainty                  \\ 
			MET\_SoftTrk\_Scale                  & track-based soft term related to longitudinal scale uncertainty                          \\ 
			MET\_JetTrk\_Scale                   & track MET scale uncertainty due to tracks in jets                                            \\ \midrule
			PRW\_DATASF         & uncertainty on data SF used for the computation of pileup reweighting      \\
			
			\bottomrule
		\end{tabular}
	\end{center}
	\caption{ Qualitative summary of the experimental systematic uncertainties considered in this analysis.}
	\label{tab:syst_summary_sources}
\end{table}


\subsection{Theory/Modelling}

\textit{Modelling uncertainties can be analysis specific in the case of the background.  However, you should 
have a clear idea for your analysis how you will estimate these uncertainties.  If you are doing 
an MC-based background estimation, describe the sources of these uncertainties and the comparisons that
you will make.  If you are doing a data driven estimation, describe the sources.}

\textit{For signal yield uncertainties, these uncertainties are evaluated in a standard way and should 
include PDF variations and renormalization/factorization scale variations.  There is more information
provided on the PMG TWiki pages for this.}


\clearpage


\section{Statistical Model/Results}

\textit{An overview of the final fit setup including the final discriminating variables(s),
the (SR/CR) regions to be included in the fit and the floating normalisation parameters.
Some rough first  expected limits/discovery sensitivity plots are useful if you have them but
not necessary. In this case the binning of the final variable(s) and the systematics
smoothing/pruning should be indicated.}


\end{document}
