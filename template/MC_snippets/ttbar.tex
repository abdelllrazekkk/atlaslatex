% !TEX root = MC_snippets.tex

%%%%%%%%%%%%%%%%%%%%%%%%%%%%%%%%%%%%%%%%%%%
%%%              ttbar                  %%%
%%%%%%%%%%%%%%%%%%%%%%%%%%%%%%%%%%%%%%%%%%%
\section[\ttbar production]{\ttbar production}
\label{subsec:ttbar}

This section describes the MC samples used for the modelling of \ttbar production.
\Cref{subsubsec:ttbar_PP8} describes the \POWPY[8] samples,
\cref{subsubsec:ttbar_PH7} describes the \POWHER[7] samples,
\cref{subsubsec:ttbar_aMCP8} describes the \MGNLOPY[8] samples,
and finally \cref{subsubsec:ttbar_sherpa} describes the \SHERPA samples.

The reference cross-section values are extracted from Ref.~\cite{LHCTopWGttbarXsec}.
Studies of MC simulation performance including comparisons with unfolded data
are collected in the PUB notes~\cite{ATL-PHYS-PUB-2018-009,ATL-PHYS-PUB-2017-007,ATL-PHYS-PUB-2016-020}.

\subsection[Powheg+Pythia8]{\POWPY[8]}
\label{subsubsec:ttbar_PP8}

\paragraph{Samples}
%\label{par:ttbar_PP8_samples}

The descriptions below correspond to the samples in \cref{tab:ttbar_PP8,tab:ttbar_PP8_addRad,tab:ttbar_PP8_hdampvar}.

\begin{table}[htbp]
  \caption{Nominal \ttbar samples produced with \POWPY[8].
    The \hdamp\ value is set to 1.5\,\mtop.}%
  \label{tab:ttbar_PP8}
  \centering
  \begin{tabular}{l l}
    \toprule
    DSID range & Description \\
    \midrule
    410470 & \ttbar non-all-hadronic \\
    410471 & \ttbar dileptonic \\
    410472 & \ttbar all-hadronic \\
    \bottomrule
  \end{tabular}
\end{table}

\begin{table}[htbp]
  \caption{\ttbar samples produced with \POWPY[8]
    used to estimate initial-state radiation systematic uncertainties.
    The \hdamp value is set to 3.0\,\mtop.}%
  \label{tab:ttbar_PP8_addRad}
  \centering
  \begin{tabular}{l l}
    \toprule
    DSID range & Description \\
    \midrule
    410480 & \ttbar single lepton \\
    410481 & \ttbar all-hadronic \\
    410482 & \ttbar dileptonic \\
    \bottomrule
  \end{tabular}
\end{table}

\begin{table}[htbp]
  \caption{\ttbar samples produced with \POWPY[8] with alternative \hdamp values
    which can be used to estimate the uncertainty due to the \POWPY[8] matching scheme.}%
  \label{tab:ttbar_PP8_hdampvar}
  \centering
  \begin{tabular}{l l l l}
    \toprule
    DSID (1.3\,\mtop) & DSID (1.8\,\mtop) & DSID (2.0\,\mtop) & Description \\
    \midrule
    411350 & 411353 & 411356 & \ttbar single lepton \\
    411351 & 411354 & 411357 & \ttbar all-hadronic \\
    411352 & 411355 & 411358 & \ttbar dileptonic \\
    \bottomrule
  \end{tabular}
\end{table}

\paragraph{Short description:}

The production of \ttbar events was modelled using the
\POWHEGBOX[v2]~\cite{Frixione:2007nw,Nason:2004rx,Frixione:2007vw,Alioli:2010xd}
generator at NLO with the \NNPDF[3.0nlo]~\cite{Ball:2014uwa} PDF set
and the \hdamp parameter\footnote{The
  \hdamp parameter is a resummation damping factor and one of the
  parameters that controls the matching of \POWHEG matrix elements to
  the parton shower and thus effectively regulates the
  high-\pT radiation against which the \ttbar system recoils.} set
to 1.5\,\mtop~\cite{ATL-PHYS-PUB-2016-020}.  The events were interfaced
to \PYTHIA[8.230]~\cite{Sjostrand:2014zea} to model the parton shower,
hadronisation, and underlying event, with parameters set according
to the A14 tune~\cite{ATL-PHYS-PUB-2014-021} and using the \NNPDF[2.3lo]
set of PDFs~\cite{Ball:2012cx}. The decays of bottom and charm hadrons
were performed by \EVTGEN[1.6.0]~\cite{Lange:2001uf}.

The uncertainty due to initial-state radiation (ISR) was estimated by
simultaneously varying the \hdamp parameter and the \muR and
\muF scales, and choosing the Var3c up/down variants of the A14 tune
as described in Ref.~\cite{ATL-PHYS-PUB-2017-007}. The impact of
final-state radiation (FSR) was evaluated by varying the renormalisation scale 
for emissions from the parton shower up or down by a factor two.



\paragraph{Long description:}

The production of \ttbar events was modelled using the
\POWHEGBOX[v2]~\cite{Frixione:2007nw,Nason:2004rx,Frixione:2007vw,Alioli:2010xd}
generator, which provided matrix elements at next-to-leading
order~(NLO) in the strong coupling constant \alphas, and the
\NNPDF[3.0nlo]~\cite{Ball:2014uwa} parton distribution function~(PDF).
The \hdamp parameter, which controls the matching in \POWHEG and
effectively regulates the high-\pT radiation against which the
\ttbar system recoils, was set to 1.5\,\mtop~\cite{ATL-PHYS-PUB-2016-020}.
The functional form of the renormalisation and factorisation scales was
set to the default scale \(\sqrt{\mtop^{2} + \pTX[2]}\).
The events were interfaced with
\PYTHIA[8.230]~\cite{Sjostrand:2014zea} for the parton shower and
hadronisation, using the A14 set of tuned
parameters~\cite{ATL-PHYS-PUB-2014-021} and the \NNPDF[2.3lo]
set of PDFs~\cite{Ball:2012cx}.
The decays of bottom and charm hadrons were simulated using the
\EVTGEN[1.6.0] program~\cite{Lange:2001uf}.

The \ttbar sample was normalised to the cross-section prediction at next-to-next-to-leading order (NNLO)
in QCD including the resummation of next-to-next-to-leading logarithmic (NNLL) soft-gluon terms calculated using
\TOPpp[2.0]~\cite{Beneke:2011mq,Cacciari:2011hy,Baernreuther:2012ws,Czakon:2012zr,Czakon:2012pz,Czakon:2013goa,Czakon:2011xx}.
For proton--proton collisions at a centre-of-mass energy of \(\rts = \qty{13}{\TeV}\), this cross-section corresponds to
\(\sigma(\ttbar)_{\text{NNLO+NNLL}} = 832 \pm 51\,\unit{\pb}\) using a top-quark mass of \(\mtop = \qty{172.5}{\GeV}\).
The uncertainties in the cross-section due to the PDF and \alphas were calculated using the \PDFforLHC[15] prescription~\cite{Butterworth:2015oua}
with the \MSTW[nnlo]~\cite{Martin:2009iq,Martin:2009bu}, \CT[10nnlo]~\cite{Lai:2010vv,Gao:2013xoa}
and \NNPDF[2.3lo]~\cite{Ball:2012cx} PDF sets in the five-flavour scheme, and were added in quadrature to the effect of the scale uncertainty.

The uncertainty due to initial-state radiation (ISR) was estimated by
comparing the nominal \ttbar sample with two additional
samples~\cite{ATL-PHYS-PUB-2017-007}.  To simulate higher parton
radiation, the factorisation and renormalisation scales were reduced by
a factor of 0.5 while simultaneously increasing the \hdamp value to
3.0\,\mtop and using the Var3c up variation from the A14 tune. For
lower parton radiation, \muR and \muF were increased by a factor of two
while keeping the \hdamp value set to 1.5\,\mtop and using the Var3c down
variation in the parton shower.  The Var3c A14 tune
variation~\cite{ATL-PHYS-PUB-2014-021} largely corresponds to the variation of
\alphas for ISR in the A14 tune. 
The impact of final-state radiation (FSR) was evaluated by
varying the renormalisation scale for emissions from the
parton shower up and down by a factor of two.


The \NNPDF[3.0lo] replicas were used to evaluate the PDF uncertainties for the nominal PDF.
In addition, the central value of this PDF was compared with the central values of the
\CT[14nnlo]~\cite{Dulat:2015mca} and \MMHT[nnlo]~\cite{Harland-Lang:2014zoa} PDF sets.


\subsection[Powheg+Herwig7.04]{\POWHER[7.04]}
\label{subsubsec:ttbar_PH7}

\paragraph{Samples}
%\label{par:ttbar_PH7_samples}

The descriptions below correspond to the samples in \cref{tab:ttbar_PH7}.

\begin{table}[htbp]
  \caption{\ttbar samples produced with \POWHER[7].}%
  \label{tab:ttbar_PH7}
  \centering
  \begin{tabular}{l l}
    \toprule
    DSID range & Description \\
    \midrule
    410557 & \ttbar single lepton  \\
    410558 & \ttbar dileptonic  \\
    410559 & \ttbar all-hadronic  \\
    \bottomrule
  \end{tabular}
\end{table}

\paragraph{Short description:}

The impact of using a different parton shower and hadronisation model was evaluated
by comparing the nominal \ttbar sample with another event sample produced with the
\POWHEGBOX[v2]~\cite{Frixione:2007nw,Nason:2004rx,Frixione:2007vw,Alioli:2010xd}
generator using the \NNPDF[3.0nlo]~\cite{Ball:2014uwa} parton distribution function~(PDF).
Events in the latter sample were interfaced with \HERWIG[7.04]~\cite{Bahr:2008pv,Bellm:2015jjp},
using the H7UE set of tuned parameters~\cite{Bellm:2015jjp} and the
\MMHT[lo] PDF set~\cite{Harland-Lang:2014zoa}.
The decays of bottom and charm hadrons
were simulated using the \EVTGEN[1.6.0] program~\cite{Lange:2001uf}.

\paragraph{Long description:}

The impact of using a different parton shower and hadronisation model was evaluated
by comparing the nominal \ttbar sample with an event sample also produced with the
\POWHEGBOX[v2]~\cite{Frixione:2007nw,Nason:2004rx,Frixione:2007vw,Alioli:2010xd}
generator but interfaced with \HERWIG[7.04]~\cite{Bahr:2008pv,Bellm:2015jjp}, using the H7UE set
of tuned parameters~\cite{Bellm:2015jjp} and the \MMHT[lo] PDF set~\cite{Harland-Lang:2014zoa}.
\POWHEGBOX provided matrix elements at next-to-leading order~(NLO) in the
strong coupling constant \alphas, and used the \NNPDF[3.0nlo]~\cite{Ball:2014uwa}
parton distribution function~(PDF) and an \hdamp parameter value of \(1.5\,\mtop\)~\cite{ATL-PHYS-PUB-2016-020}.
The functional form of the renormalisation and factorisation scales was
set to the default scale \(\sqrt{\mtop^{2} + \pT^2}\).
The decays of bottom and charm hadrons
were simulated using the \EVTGEN[1.6.0] program~\cite{Lange:2001uf}.


\subsection[Powheg+Herwig7.13]{\POWHER[7.13]}
%\label{subsubsec:ttbar_PH713}

\paragraph{Samples}
%\label{par:ttbar_PH713_samples}

The descriptions below correspond to the samples in \cref{tab:ttbar_PH713}.

\begin{table}[htbp]
  \caption{\ttbar samples produced with \POWHER[7.13].}%
  \label{tab:ttbar_PH713}
  \centering
  \begin{tabular}{l l}
    \toprule
    DSID range & Description \\
    \midrule
    411233 & \ttbar single lepton  \\
    411234 & \ttbar dileptonic  \\
    411316 & \ttbar all-hadronic  \\
    \bottomrule
  \end{tabular}
\end{table}

\paragraph{Short description:}

The impact of using a different parton shower and hadronisation model was evaluated
by comparing the nominal \ttbar sample with another event sample produced with the
\POWHEGBOX[v2]~\cite{Frixione:2007nw,Nason:2004rx,Frixione:2007vw,Alioli:2010xd}
generator using the \NNPDF[3.0nlo]~\cite{Ball:2014uwa} parton distribution function~(PDF).
Events in the latter sample were interfaced with \HERWIG[7.13]~\cite{Bahr:2008pv,Bellm:2015jjp},
using the \HERWIG[7.1] default set of tuned parameters~\cite{Bellm:2015jjp,Bellm:2017jjp}
and the \MMHT[lo] PDF set~\cite{Harland-Lang:2014zoa}.
The decays of bottom and charm hadrons
were simulated using the \EVTGEN[1.6.0] program~\cite{Lange:2001uf}.


\paragraph{Long description:}

The impact of using a different parton shower and hadronisation model was evaluated
by comparing the nominal \ttbar sample with an event sample also produced with the
\POWHEGBOX[v2]~\cite{Frixione:2007nw,Nason:2004rx,Frixione:2007vw,Alioli:2010xd}
generator but interfaced with \HERWIG[7.13]~\cite{Bahr:2008pv,Bellm:2015jjp}, using the
\HERWIG[7.1] default set of tuned parameters~\cite{Bellm:2015jjp,Bellm:2017jjp} and the
\MMHT[lo] PDF set~\cite{Harland-Lang:2014zoa}.
\POWHEGBOX provided matrix elements at next-to-leading
order~(NLO) in the strong coupling constant \alphas, and used the
\NNPDF[3.0nlo]~\cite{Ball:2014uwa} parton distribution function~(PDF) and
an \hdamp parameter value of 1.5\,\mtop~\cite{ATL-PHYS-PUB-2016-020}.
The functional form of the renormalisation and factorisation scales was
set to the default scale \(\sqrt{\mtop^{2} + \pTX[2]}\).
The decays of bottom and charm hadrons
were simulated using the \EVTGEN[1.6.0] program~\cite{Lange:2001uf}.


\subsection[MadGraph5\_aMC@NLO+Pythia8]{\MGNLOPY[8]}
\label{subsubsec:ttbar_aMCP8}

\paragraph{Samples}
%\label{par:ttbar_aMCP8_samples}

The descriptions below correspond to the samples in \cref{tab:ttbar_aMCP8}.
\begin{table}[htbp]
  \caption{\ttbar samples produced with \MGNLOPY[8].}%
  \label{tab:ttbar_aMCP8}
  \centering
  \begin{tabular}{l l}
  \toprule
  DSID range & Description \\
  \midrule
  410464 & \ttbar single lepton \\
  410465 & \ttbar dileptonic \\
  410466 & \ttbar all-hadronic \\
  \bottomrule
  \end{tabular}
\end{table}

\paragraph{Short description:}

To assess the uncertainty in the matching of NLO matrix elements to the
parton shower, the \POWHEG sample was compared with a sample of events
generated with \MGNLO[2.6.0]~\cite{Alwall:2014hca} interfaced with
\PYTHIA[8.230]~\cite{Sjostrand:2014zea}. The \MGNLO calculation used the
\NNPDF[3.0nlo] set of PDFs~\cite{Ball:2014uwa} and \PYTHIA[8] used
the A14 set of tuned parameters~\cite{ATL-PHYS-PUB-2014-021} and
the \NNPDF[2.3lo] set of PDFs~\cite{Ball:2012cx}.
The decays of bottom and charm hadrons
were simulated using the \EVTGEN[1.6.0] program~\cite{Lange:2001uf}.

\paragraph{Long description:}

To assess the uncertainty due to the choice of matching scheme,
the \POWHEG sample was compared with a sample generated by
\MGNLOPY[8]. For the calculation of the hard-scattering,
\MGNLO[2.6.0]~\cite{Alwall:2014hca} with the \NNPDF[3.0nlo]~\cite{Ball:2014uwa} PDF set was
used. The events were interfaced with
\PYTHIA[8.230]~\cite{Sjostrand:2014zea}, using the A14 set of tuned
parameters~\cite{ATL-PHYS-PUB-2014-021} and the \NNPDF[2.3lo] set of PDFs~\cite{Ball:2012cx}.
Top quarks were decayed at LO using
\MADSPIN~\cite{Frixione:2007zp,Artoisenet:2012st} to preserve spin
correlations. The decays of bottom and charm hadrons were simulated
using the \EVTGEN[1.6.0] program~\cite{Lange:2001uf}.  The parton-shower starting
scale had the functional form \(\muQ = \HT/2\)~\cite{ATL-PHYS-PUB-2017-007},
where \HT is defined as the scalar sum of the \pT of all outgoing partons.
The renormalisation and factorisation scale choice was the same as for the
\POWHEGBOX set-up.

\subsection[MadGraph5\_aMC@NLO+Herwig7.13]{\MGNLOHER[7.13]}
%\label{subsubsec:ttbar_aMCH713}

\paragraph{Samples}
%\label{par:ttbar_aMCH713_samples}

The descriptions below correspond to the samples in \cref{tab:ttbar_aMCH713}.
\begin{table}[htbp]
  \caption{\ttbar samples produced with \MGNLOHER[7.13].}%
  \label{tab:ttbar_aMCH713}
  \centering
  \begin{tabular}{l l}
    \toprule
    DSID range & Description \\
    \midrule
    412116 & \ttbar single lepton \\
    412117 & \ttbar dileptonic \\
    412175 & \ttbar all-hadronic \\
    \bottomrule
  \end{tabular}
\end{table}

\paragraph{Short description:}

To assess the uncertainty in the matching of NLO matrix elements to the
parton shower, a sample produced with the
\POWHEGBOX[v2]
generator was compared with a sample generated with \MGNLO[2.6.0]~\cite{Alwall:2014hca}, both using the
\NNPDF[3.0nlo]~\cite{Ball:2014uwa} parton distribution function~(PDF) and interfaced with
\HERWIG[7.13]~\cite{Bahr:2008pv,Bellm:2015jjp}, using the \HERWIG[7.1] default set of
tuned parameters~\cite{Bellm:2015jjp} and the \MMHT[lo] PDF set~\cite{Harland-Lang:2014zoa}.
The decays of bottom and charm hadrons
were simulated using the \EVTGEN[1.6.0] program~\cite{Lange:2001uf}.


\paragraph{Long description:}

To assess the uncertainty in the matching of NLO matrix elements to the parton shower,
a \POWHEG sample was compared with a sample generated by \MGNLO~\cite{Alwall:2014hca}.
The first sample was produced with the same hard-scatter set-up as the nominal sample using the
\POWHEGBOX[v2]~\cite{Frixione:2007nw,Nason:2004rx,Frixione:2007vw,Alioli:2010xd}
generator, which provided matrix elements at next-to-leading
order~(NLO) in the strong coupling constant \alphas, with the
\NNPDF[3.0nlo]~\cite{Ball:2014uwa} parton distribution function~(PDF) and
the \hdamp parameter set to 1.5\,\mtop~\cite{ATL-PHYS-PUB-2016-020}.
The functional form of the renormalisation and factorisation scales was
set to the default scale \(\sqrt{\mtop^{2} + \pTX[2]}\).
The second sample used \MGNLO[2.6.0] with the
\NNPDF[3.0nlo]~\cite{Ball:2014uwa} PDF set for the calculation of the hard-scattering.
Top quarks were decayed at LO using
\MADSPIN~\cite{Frixione:2007zp,Artoisenet:2012st} to preserve spin
correlations. The parton-shower starting
scale had the functional form \(\muQ = \HT/2\)~\cite{ATL-PHYS-PUB-2017-007},
where \HT is defined
as the scalar sum of the \pT of all outgoing partons.
The events from both generators were interfaced with
\HERWIG[7.13]~\cite{Bahr:2008pv,Bellm:2015jjp}, using the \HERWIG[7.1] default set
of tuned parameters~\cite{Bellm:2015jjp} and the \MMHT[lo] PDF set
\cite{Harland-Lang:2014zoa}.
The renormalisation and factorisation scale choice in the \MGNLO set-up was the same as for the
\POWHEGBOX set-up. The decays of bottom and charm hadrons were simulated
using the \EVTGEN[1.6.0] program~\cite{Lange:2001uf} in both set-ups.


\subsection[Sherpa 2.2.1]{\SHERPA[2.2.1]}
\label{subsubsec:ttbar_sherpa}

\paragraph{Samples}
%\label{par:ttbar_sherpa_samples}

The descriptions below correspond to the samples in \cref{tab:ttbar_sherpa}.
\begin{table}[htbp]
  \caption{\ttbar samples produced with \SHERPA[2.2.1].}%
  \label{tab:ttbar_sherpa}
  \centering
  \begin{tabular}{l l}
  \toprule
  DSID range & Description \\
  \midrule
  410249 & \ttbar all-hadronic  \\
  410250 & \ttbar single lepton  \\
  410251 & \ttbar single lepton  \\
  410252 & \ttbar dileptonic  \\
  \bottomrule
  \end{tabular}
\end{table}

\paragraph{Short description:}

Additional samples of \ttbar events were produced with the
\SHERPA[2.2.1]~\cite{Bothmann:2019yzt} generator using NLO-accurate
matrix elements for up to one additional parton, and LO-accurate
matrix elements for up to four additional partons calculated with the
Comix~\cite{Gleisberg:2008fv} and
\OPENLOOPS~\cite{Buccioni:2019sur,Cascioli:2011va,Denner:2016kdg} libraries. They were
matched with the \SHERPA parton shower~\cite{Schumann:2007mg} using
the \MEPSatNLO
prescription~\cite{Hoeche:2011fd,Hoeche:2012yf,Catani:2001cc,Hoeche:2009rj}
and the set of tuned parameters developed by the \SHERPA authors
to match the \NNPDF[3.0nnlo] set of PDFs~\cite{Ball:2014uwa}.

\textbf{Additional information}: The central scale had the functional
form \(\mu^{2} = \mtop^{2} + 0.5\times(\pTX[2][t]p + \pTX[2][\bar{t}])\).
The CKKW matching scale of the
additional emissions was set to \qty{30}{\GeV}.
