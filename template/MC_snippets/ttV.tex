%%%%%%%%%%%%%%%%%%%%%%%%%%%%%%%%%%%%%%%%%%%
%%%              ttV                    %%%
%%%%%%%%%%%%%%%%%%%%%%%%%%%%%%%%%%%%%%%%%%%
\subsection[\ttV production]{\ttV production}
\label{subsec:ttV}

This section describes the MC samples used for the modelling of \ttV ($V=W/Z$) production.
Section~\ref{subsubsec:ttV_aMCP8} describes the \MGNLOPY[8] samples,
and Section~\ref{subsubsec:ttV_sherpa} describes the \SHERPA samples.
(NOTE: this section is not frozen as the \SHERPA samples are likely to be updated and become the nominal samples in the near future.)

\subsubsection[MadGraph5\_aMC@NLO+Pythia8]{\MGNLOPY[8]}
\label{subsubsec:ttV_aMCP8}

\paragraph{Samples}
%\label{par:ttV_aMCP8_samples}

The descriptions below correspond to the samples in Tables~\ref{tab:ttV_aMCP8} and \ref{tab:ttV_aMCP8_addRad}.

\begin{table}[htbp]
\begin{center}
\caption{Nominal \ttV samples produced with \MGNLOPY[8].} 
\label{tab:ttV_aMCP8}
\begin{tabular}{ l | l }
\hline
DSID range & Description \\
\hline
410155 & \ttW \\
410156 & \ttZnunu \\
410157 & \ttZqq \\
410218 & \ttee, $m_{\ell\ell} > 5$~\GeV \\
410219 & \ttmumu, $m_{\ell\ell} > 5$~\GeV \\
410220 & \tttautau, $m_{\ell\ell} > 5$~\GeV \\
410276 & \ttee, $m_{\ell\ell} \in [1,5]$~\GeV\\
410277 & \ttmumu, $m_{\ell\ell} \in [1,5]$~\GeV\\
410278 & \tttautau, $m_{\ell\ell} \in [1,5]$~\GeV\\
\hline
\end{tabular}
\end{center}
\end{table}

\begin{table}[htbp]
\begin{center}
\caption{\ttV samples produced with \MGNLOPY[8] used to estimate initial-state radiation systematic uncertainties.}
\label{tab:ttV_aMCP8_addRad}
\begin{tabular}{ l | l }
\hline
DSID range & Description \\
\hline
410376 & \ttW\ A14Var3c up \\
410377 & \ttW\ A14Var3c down \\
410378 & \ttZnunu\ A14Var3c up \\
410379 & \ttZnunu\ A14Var3c down \\
410380 & \ttZqq\ A14Var3c up \\
410381 & \ttZqq\ A14Var3c down \\
410370 & \ttee\ A14Var3c up \\
410371 & \ttee\ A14Var3c down \\
410372 & \ttmumu\ A14Var3c up \\
410373 & \ttmumu\ A14Var3c down \\
410374 & \tttautau\ A14Var3c up \\
410375 & \tttautau\ A14Var3c down \\
\hline
\end{tabular}
\end{center}
\end{table}


\paragraph{Short description:}

The production of \ttV\ events was modelled using the
\MGNLO[2.3.3]~\cite{Alwall:2014hca} generator at NLO with the
\NNPDF[3.0nlo]~\cite{Ball:2014uwa} parton distribution function~(PDF).
The events were interfaced to \PYTHIA[8.210]~\cite{Sjostrand:2014zea}~
using the A14 tune~\cite{ATL-PHYS-PUB-2014-021} and the
\NNPDF[2.3lo]~\cite{Ball:2014uwa} PDF set. The decays of bottom and charm
hadrons were simulated using the \EVTGEN[1.2.0] program~\cite{Lange:2001uf}.

The uncertainty due to initial-state radiation (ISR) was estimated by
comparing the nominal event sample with two samples where the Var3c
up/down variations of the A14 tune were employed.

\paragraph{Long description:}

The production of \ttV events was modelled using the \MGNLO[2.3.3]
\cite{Alwall:2014hca} generator, which provided matrix elements at
next-to-leading order~(NLO) in the strong coupling constant \alphas
with the \NNPDF[3.0nlo]~\cite{Ball:2014uwa} parton distribution
function~(PDF). The functional form of the renormalisation and
factorisation scales was set to the default of $0.5 \times \sum_i
\sqrt{m^2_i+p^2_{\text{T},i}}$, where the sum runs over all the particles
generated from the matrix element calculation. Top quarks were decayed
at LO using \MADSPIN~\cite{Frixione:2007zp,Artoisenet:2012st} to
preserve spin correlations. The events were interfaced with
\PYTHIA[8.210]~\cite{Sjostrand:2014zea} for the parton shower and
hadronisation, using the A14 set of tuned
parameters~\cite{ATL-PHYS-PUB-2014-021} and the
\NNPDF[2.3lo]~\cite{Ball:2014uwa} PDF set.  
The decays of bottom and charm hadrons were simulated using the \EVTGEN[1.2.0] program~\cite{Lange:2001uf}. 

% Note : The cross-section is supplemented with an off-shell (down to 5~\GeV) correction calculated in Ref.~\cite{ATLASPMGttVNLOXsec}. 
The cross-sections were calculated at NLO QCD and NLO EW accuracy using
\MGNLO as reported in Ref.~\cite{deFlorian:2016spz}.
In the case of \ttll the cross-section was scaled by 
an off-shell correction estimated at one-loop level in \alphas.
(\emph{Optionally:})
The predicted values at \rts~=~\SI{13}{\TeV} are 0.88$^{+0.09}_{-0.11}$\,pb and
0.60$^{+0.08}_{-0.07}$\,pb for \ttZ and \ttW, respectively, where the
uncertainties were estimated from variations of \alphas and the renormalisation and
factorisation scales.

The uncertainty due to initial-state radiation (ISR) was estimated by
comparing the nominal \ttV sample with two additional samples, which
have the same settings as the nominal one, but employed the Var3c up or down
variation of the A14 tune, which corresponds
to the variation of \alphas for initial-state radiation
(ISR) in the A14 tune.

Uncertainties due to missing higher-order corrections were evaluated
by simultaneously varying the renormalisation and factorisation scales
by factors of 2.0 and 0.5. Uncertainties in the PDFs
were evaluated using the 100 replicas of the \NNPDF[3.0nlo] set.

\subsubsection[Sherpa]{\SHERPA}
\label{subsubsec:ttV_sherpa}

\paragraph{Samples}
%\label{par:ttV_sherpa_samples}

The descriptions below correspond to the samples in Table~\ref{tab:ttV_sherpa}.
\begin{table}[htbp]
\begin{center}
\caption{\ttV samples produced with \SHERPA.}
\label{tab:ttV_sherpa}
\begin{tabular}{ l | l }
\hline
DSID range & Description \\
\hline
410142 & \ttll \\
410143 & \ttZqq, \ttZnunu \\
410144 & \ttW \\
\hline
\end{tabular}
\end{center}
\end{table}


\paragraph{Description:}

Additional \ttV samples were produced with the
\SHERPA[2.2.0]~\cite{Bothmann:2019yzt} generator at LO accuracy, using
the \MEPSatLO set-up~\cite{Catani:2001cc,Hoeche:2009rj} with up to one
additional parton for the \ttll sample and two additional partons for the
others. A dynamic renormalisation scale was used and is defined
similarly to that of the nominal \ttV samples. The CKKW matching
scale of the additional emissions was set to 30\,\GeV. The default
\SHERPA[2.2.0] parton shower was used along with the
\NNPDF[3.0nnlo]~\cite{Ball:2014uwa} PDF set.
