%-------------------------------------------------------------------------------
% This file provides a skeleton ATLAS document.
%-------------------------------------------------------------------------------
% \pdfoutput=1
% The \pdfoutput command is needed by arXiv/JHEP/JINST to ensure use of pdflatex.
% It should be included in the first 5 lines of the file.
%-------------------------------------------------------------------------------
% Specify where ATLAS LaTeX style files can be found.
\newcommand*{\ATLASLATEXPATH}{../../latex/}
% Use this variant if the files are in a central location, e.g. $HOME/texmf.
% \newcommand*{\ATLASLATEXPATH}{}
%-------------------------------------------------------------------------------
\documentclass[UKenglish,texlive=2016,txfonts=true]{\ATLASLATEXPATH atlasdoc}
% The language of the document must be set: usually UKenglish or USenglish.
% british and american also work!
% Commonly used options:
%  texlive=YYYY          Specify TeX Live version (2016 is default).
%  atlasstyle=true|false Use ATLAS style for document (default).
%  coverpage             Create ATLAS draft cover page for collaboration circulation.
%                        See atlas-draft-cover.tex for a list of variables that should be defined.
%  cernpreprint          Create front page for a CERN preprint.
%                        See atlas-preprint-cover.tex for a list of variables that should be defined.
%  PAPER                 The document is an ATLAS paper (draft).
%  CONF                  The document is a CONF note (draft).
%  PUB                   The document is a PUB note (draft).
%  BOOK                  The document is of book form, like an LOI or TDR (draft)
%  txfonts=true|false    Use txfonts rather than the default newtx - needed for arXiv submission.
%  paper=a4|letter       Set paper size to A4 (default) or letter.

%-------------------------------------------------------------------------------
% Extra packages:
\usepackage{\ATLASLATEXPATH atlaspackage}
% Commonly used options:
%  biblatex=true|false   Use biblatex (default) or bibtex for the bibliography.
%  backend=biber         Use the biber backend rather than bibtex.
%  subfigure|subfig|subcaption  to use one of these packages for figures in figures.
%  minimal               Minimal set of packages.
%  default               Standard set of packages.
%  full                  Full set of packages.
%-------------------------------------------------------------------------------
% Style file with biblatex options for ATLAS documents.
\usepackage{\ATLASLATEXPATH atlasbiblatex}

% Package for creating list of authors and contributors to the analysis.
\usepackage{\ATLASLATEXPATH atlascontribute}

% Useful macros
\usepackage{\ATLASLATEXPATH atlasphysics}
% See doc/atlas_physics.pdf for a list of the defined symbols.
% Default options are:
%   true:  journal, misc, particle, unit, xref
%   false: BSM, heppparticle, hepprocess, hion, jetetmiss, math, process, other, texmf
% See the package for details on the options.

% Files with references for use with biblatex.
% Note that biber gives an error if it finds empty bib files.
%\addbibresource{MC-snippets.bib}

\addbibresource{../../bib/ATLAS.bib}
\addbibresource{../../bib/ATLAS-errata.bib}
\addbibresource{../../bib/ATLAS-useful.bib}
\addbibresource{../../bib/CMS.bib}
\addbibresource{../../bib/ConfNotes.bib}
\addbibresource{../../bib/PubNotes.bib}

% Paths for figures - do not forget the / at the end of the directory name.
\graphicspath{{../../logos/}{figures/}}

% Add you own definitions here (file MC-snippets-defs.sty).
\usepackage{MC-snippets-defs}

%-------------------------------------------------------------------------------
% Generic document information
%-------------------------------------------------------------------------------

% Title, abstract and document 
%-------------------------------------------------------------------------------
% This file contains the title, author and abstract.
% It also contains all relevant document numbers used by the different cover pages.
%-------------------------------------------------------------------------------

% Title
\AtlasTitle{References and set-up descriptions for the MC16 campaign}

% Author - this does not work with revtex (add it after \begin{document})
\author{The ATLAS Collaboration}

% Authors and list of contributors to the analysis
% \AtlasAuthorContributor also adds the name to the author list
% Include package latex/atlascontribute to use this
% Use authblk package if there are multiple authors, which is included by latex/atlascontribute
% \usepackage{authblk}
% Use the following 3 lines to have all institutes on one line
% \makeatletter
% \renewcommand\AB@affilsepx{, \protect\Affilfont}
% \makeatother
% \renewcommand\Authands{, } % avoid ``. and'' for last author
% \renewcommand\Affilfont{\itshape\small} % affiliation formatting
% \AtlasAuthorContributor{First AtlasAuthorContributor}{a}{Author's contribution.}
% \AtlasAuthorContributor{Second AtlasAuthorContributor}{b}{Author's contribution.}
% \AtlasAuthorContributor{Third AtlasAuthorContributor}{a}{Author's contribution.}
% \AtlasContributor{Fourth AtlasContributor}{Contribution to the analysis.}
% \author[a]{First Author}
% \author[a]{Second Author}
% \author[b]{Third Author}
% \affil[a]{One Institution}
% \affil[b]{Another Institution}

% If a special author list should be indicated via a link use the following code:
% Include the two lines below if you do not use atlasstyle:
% \usepackage[marginal,hang]{footmisc}
% \setlength{\footnotemargin}{0.5em}
% Use the following lines in all cases:
% \usepackage{authblk}
% \author{The ATLAS Collaboration%
% \thanks{The full author list can be found at:\newline
%   \url{https://atlas.web.cern.ch/Atlas/PUBNOTES/ATL-PHYS-PUB-2016-007/authorlist.pdf}}
% }

% Draft version:
% Should be 1.0 for the first circulation, and 2.0 for the second circulation.
% If given, adds draft version on front page, a 'DRAFT' box on top of each other page, 
% and line numbers.
% Comment or remove in final version.
\AtlasVersion{2.0}

% ATLAS reference code, to help ATLAS members to locate the paper
%\AtlasRefCode{GROUP-2016-XX}

% ATLAS note number. Can be an COM, INT, PUB or CONF note
% \AtlasNote{ATLAS-CONF-2016-XXX}
% \AtlasNote{ATL-PHYS-PUB-2016-XXX}
% \AtlasNote{ATL-COM-PHYS-2016-XXX}

% CERN preprint number
% \PreprintIdNumber{CERN-PH-2016-XX}

% ATLAS date - arXiv submission; usually filled in by the Physics Office
% \AtlasDate{\today}

% ATLAS heading - heading at top of title page. Set for TDR etc.
% \AtlasHeading{ATLAS ABC TDR}

% arXiv identifier
% \arXivId{14XX.YYYY}

% HepData record
% \HepDataRecord{ZZZZZZZZ}

% Submission journal and final reference
% \AtlasJournal{Phys.\ Lett.\ B.}
% \AtlasJournalRef{\PLB 789 (2014) 123}
% \AtlasDOI{}

% Abstract - % directly after { is important for correct indentation
\AtlasAbstract{%
  This document is intended as a template for referencing ATLAS Monte Carlo
set-ups for Standard Model processes produced in the MC16
campaign. It includes descriptions of the generator configurations and
the appropriate references.}

%-------------------------------------------------------------------------------
% The following information is needed for the cover page. The commands are only defined
% if you use the coverpage option in atlasdoc or use the atlascover package
%-------------------------------------------------------------------------------

% List of supporting notes  (leave as null \AtlasCoverSupportingNote{} if you want to skip this option)
% \AtlasCoverSupportingNote{Short title note 1}{https://cds.cern.ch/record/XXXXXXX}
% \AtlasCoverSupportingNote{Short title note 2}{https://cds.cern.ch/record/YYYYYYY}
%
% OR (the 2nd option is deprecated, especially for CONF and PUB notes)
%
% Supporting material TWiki page  (leave as null \AtlasCoverTwikiURL{} if you want to skip this option)
% \AtlasCoverTwikiURL{https://twiki.cern.ch/twiki/bin/view/Atlas/WebHome}

% Comment deadline
% \AtlasCoverCommentsDeadline{DD Month 2016}

% Analysis team members - contact editors should no longer be specified
% as there is a generic email list name for the editors
% \AtlasCoverAnalysisTeam{Peter Analyser, Susan Editor1, Jenny Editor2, Alphonse Physicien}

% Editorial Board Members - indicate the Chair by a (chair) after his/her name
% Give either all members at once (then they appear on one line), or separately
% \AtlasCoverEdBoardMember{EdBoard~Chair~(chair), EB~Member~1, EB~Member~2, EB~Member~3}
% \AtlasCoverEdBoardMember{EdBoard~Chair~(chair)}
% \AtlasCoverEdBoardMember{EB~Member~1}
% \AtlasCoverEdBoardMember{EB~Member~2}
% \AtlasCoverEdBoardMember{EB~Member~3}

% A PUB note has readers and not an EdBoard -- give their names here (one line or several entries)
% \AtlasCoverReaderMember{Reader~1, Reader~2}
% \AtlasCoverReaderMember{Reader~1}
% \AtlasCoverEdBoardMember{Reader~2}

% Editors egroup
% \AtlasCoverEgroupEditors{atlas-GROUP-2016-XX-editors@cern.ch}

% EdBoard egroup
% \AtlasCoverEgroupEdBoard{atlas-GROUP-2016-XX-editorial-board@cern.ch}


% Author and title for the PDF file
\hypersetup{pdftitle={ATLAS document},pdfauthor={The ATLAS Collaboration}}

%-------------------------------------------------------------------------------
% Content
%-------------------------------------------------------------------------------
\begin{document}

\maketitle

\tableofcontents

% List of contributors - print here or after the Bibliography.
%\PrintAtlasContribute{0.30}
%\clearpage

%-------------------------------------------------------------------------------
% Intro
\section{Introduction}

This document is a collection of short descriptions of the baseline Standard Model processes
produced as part of the ATLAS MC16 production campaign. Often a short and a long description 
is provided, depending on whether a sample is used as a background or a signal sample in an
analysis, respectively.


It is assumed that paper editors will make a final pass through the wording, e.g.\ to avoid
acronyms being introduced multiple times.
The descriptions contain the appropriate citations which are included by default in 
the \texttt{atlaslatex} package as well.
These citations often reflect decades of theory work and would have typically been agreed upon 
with the generator developers, who rely on them to secure funding for future generator development.
PMG therefore strongly encourages \emph{keeping all recommended citations} for any given snippet.


Please note that the generator versions can generally change from sample to sample. 
A change in the third digit typically indicates some sort of technical bug fix that
does not affect the physics modelling otherwise. 
In order to save CPU time, samples are often regenerated only when they are affected 
by a (sufficiently severe) bug and so even within a set of final states 
of any given process, the generator version may differ. 


%-------------------------------------------------------------------------------
% Common generator details
%\include{MC}
%-------------------------------------------------------------------------------
% Minbias
\chapter{\Pileup overlay}

\paragraph{Description:}

The effect of multiple interactions in the same and neighbouring bunch
crossings (\pileup) was modelled by overlaying the simulated hard-scattering event with
inelastic proton--proton (\(pp\)) events generated with \PYTHIA[8.186]~\cite{Sjostrand:2007gs}
using the \NNPDF[2.3lo] set of parton distribution functions (PDF)~\cite{Ball:2012cx} and the
A3 set of tuned parameters~\cite{ATL-PHYS-PUB-2016-017}.

\paragraph{Optional description:}

The Monte Carlo (MC) events were weighted to reproduce the
distribution of the average number of interactions per bunch crossing
(\(\left<\mu \right>\)) observed in the data. The \(\left<\mu \right>\)
value in data was rescaled by a factor of \(1.03\pm 0.04\) to improve
agreement between data and simulation in the visible inelastic
proton--proton (\(pp\)) cross-section~\cite{STDM-2015-05}.


%-------------------------------------------------------------------------------
% V+jets
% !TEX root = MC_snippets.tex

\chapter{Single-boson processes}
%\label{sec:VJ}

In the following paragraphs, the set-up of the current ATLAS single-boson baseline samples is described.
Details of the full process configuration are given in the PUB note~\cite{ATL-PHYS-PUB-2017-006}. In the case of \SHERPA samples,
a minimal description of built-in systematic uncertainties is also given.

%In addition, \SHERPA[2.1.1] is still used for cross checks and in the evaluation of theoretical uncertainties.
%Samples using this version have been generated with the \CT[10nlo] set of PDFs~\cite{Lai:2010vv}, in conjunction
%with the dedicated set of tuned parton-shower parameters developed by the \SHERPA authors for this generator version.

\section[Sherpa MEPS@NLO]{\SHERPA (\MEPSatNLO)}
%\label{sec:vjets-sherpa}

\subsection{QCD \(V+\)jets}
%\label{sec:vjets-sherpa-vjets}

\subsection*{Samples}
%\label{sec:vjets-sherpa-samples}

The descriptions below correspond to the samples in \cref{tab:vjets-sherpa}.

\begin{table}[!htbp]
  \caption{\(V\)+jets samples with \SHERPA.}%
  \label{tab:vjets-sherpa}
  \centering
  \begin{tabular}{l l}
    \toprule
    DSID range & Description \\
    \midrule
    364100--364113    & \(Z\to\mu\mu\)   \\
    364198--364203    & \(Z\to \mu\mu\) (\(\SI{10}{\GeV} < m_{\ell\ell} < \SI{40}{\GeV}\)) \\
    364359, 364362, 364281  &  \(Z\to\mu\mu\) (very low mass)   \\
    364114--364127    & \(Z\to ee\)    \\
    364204--364209    & \(Z\to ee\) (\(\SI{10}{\GeV} < m_{\ell\ell} < \SI{40}{\GeV}\)) \\
    364358, 364361, 364282 & \(Z\to ee\) (very low mass) \\
    364128--364141    & \(Z\to \tau\tau\)    \\
    364210--364215    & \(Z\to \tau\tau\) (\(\SI{10}{\GeV} < m_{\ell\ell} < \SI{40}{\GeV}\)) \\
    364282, 364360, 364363 &  \(Z\to \tau\tau\) \\
    364142--364155    & \(Z\to \nu\nu\)   \\
    364156--364169    & \(W\to \mu\nu\)   \\
    364170--364183    & \(W\to e\nu\)    \\
    364184--364197    & \(W\to \tau\nu\)  \\
    364216--364229    & \(Z\to\ell\ell,W\to\ell\nu\) (high \pT)\\
    \bottomrule
  \end{tabular}
\end{table}

\paragraph{Short description:}

The production of \(V+\)jets was simulated with the
\SHERPA[2.2.1]~\cite{Bothmann:2019yzt}
generator using next-to-leading-order (NLO) matrix elements (ME) for up to two partons, and leading-order (LO) matrix elements
for up to four partons calculated with the Comix~\cite{Gleisberg:2008fv}
and \OPENLOOPS~\cite{Buccioni:2019sur,Cascioli:2011va,Denner:2016kdg} libraries. They
were matched with the \SHERPA parton shower~\cite{Schumann:2007mg} using the \MEPSatNLO
prescription~\cite{Hoeche:2011fd,Hoeche:2012yf,Catani:2001cc,Hoeche:2009rj}
using the set of tuned parameters developed by the \SHERPA authors.
The \NNPDF[3.0nnlo] set of PDFs~\cite{Ball:2014uwa} was used and the samples
were normalised to a next-to-next-to-leading-order (NNLO)
prediction~\cite{Anastasiou:2003ds}.


\paragraph{Long description:}

The production of \(V+\)jets was simulated with the \SHERPA[2.2.1]~\cite{Bothmann:2019yzt}
generator. In this set-up, NLO-accurate matrix elements for up to two partons, and LO-accurate matrix elements for up
to four partons were calculated with the Comix~\cite{Gleisberg:2008fv} and
\OPENLOOPS~\cite{Buccioni:2019sur,Cascioli:2011va,Denner:2016kdg} libraries.
The default \SHERPA parton shower~\cite{Schumann:2007mg} based on
Catani--Seymour dipole factorisation and the cluster hadronisation model~\cite{Winter:2003tt}
were used. They employed the dedicated set of tuned parameters developed by the
\SHERPA authors and the \NNPDF[3.0nnlo] PDF set~\cite{Ball:2014uwa}.

The NLO matrix elements for a given jet multiplicity were matched to the parton
shower (PS) using a colour-exact variant of the MC@NLO
algorithm~\cite{Hoeche:2011fd}. Different jet multiplicities were then merged
into an inclusive sample using an improved CKKW matching
procedure~\cite{Catani:2001cc,Hoeche:2009rj} which was extended to NLO
accuracy using the \MEPSatNLO prescription~\cite{Hoeche:2012yf}. The merging threshold
was set to \SI{20}{\GeV}.

Uncertainties from missing higher orders were
evaluated~\cite{Bothmann:2016nao} using seven variations of the QCD
factorisation and renormalisation scales in the matrix elements by
factors of \(0.5\) and \(2\), avoiding variations in opposite directions.

Uncertainties in the nominal PDF set were evaluated using 100 replica
variations. Additionally, the results were cross-checked using the
central values of the \CT[14nnlo]~\cite{Dulat:2015mca} and
\MMHT[nnlo]~\cite{Harland-Lang:2014zoa} PDF sets. The effect of the uncertainty
in the strong coupling constant \(\alphas\) was assessed by variations of \(\pm 0.001\).






The \(V\)+jets samples were normalised to a next-to-next-to-leading-order (NNLO)
prediction~\cite{Anastasiou:2003ds}.


\subsection{Electroweak \(Vjj\) (VBF)}
%\label{sec:vjets-sherpa-vjj}

The descriptions below correspond to the samples in
\cref{tab:ewkvjets-sherpa}.  Samples include the VBF and \(V\)-strahlung diagrams, but
they do not include semileptonic \(VV\) diagrams and do not overlap with QCD \(V+\)jets samples.

\begin{table}[!htbp]
  \caption{Electroweak \(Vjj\) samples with \SHERPA.}%
  \label{tab:ewkvjets-sherpa}
  \centering
  \begin{tabular}{l l}
    \toprule
    DSID range & Description \\
    \midrule
    308092--308096 & EWK \(Vjj\) \\
    \bottomrule
  \end{tabular}
\end{table}

\paragraph{Description:}

Electroweak production of \(\ell\ell jj\), \(\ell\nu jj\) and \(\nu\nu jj\) final states
was simulated with \SHERPA[2.2.1]~\cite{Bothmann:2019yzt} using
leading-order (LO) matrix elements with up to two additional parton emissions.
The matrix elements were merged with the \SHERPA parton
shower~\cite{Schumann:2007mg} following the \MEPSatLO
prescription~\cite{Catani:2001cc} and using the set of tuned
parameters developed by the \SHERPA authors.  The \NNPDF[3.0nnlo] set of
PDFs~\cite{Ball:2014uwa} was employed. The samples were produced
using the VBF approximation, which avoids overlap  with semileptonic
diboson topologies by requiring a \(t\)-channel colour-singlet exchange.

%These samples are generated in the \(G_\mu\) scheme using, ensuring an
%optimal description of pure electroweak interactions at the
%electroweak scale.


\section[MadGraph5 (CKKW-L)]{\MADGRAPH (CKKW-L)}
%\label{sec:vjets-mg5py8_ckkwl}

\subsection*{Samples}
%\label{sec:vjets-mg5py8_ckkwl-samples}

The descriptions below correspond to the samples in
\cref{tab:vjets-mg5py8_ckkwl}. The set-ups of \(N_\text{parton}\)- and
\HT-sliced samples differ slightly between the two slicing schemes
with regard to the matrix element PDF, the jet-clustering radius parameter
and the scale used in the evaluation of  \alphas to determine the weight of
each splitting. The short description merges the two set-ups and requires
the paper editors to select the appropriate PDF set (or gracefully describe
both); the long description is left unmerged.

\begin{table}[!htbp]
  \caption{\(V\)+jets samples with \MGPY[8] using CKKW-L merging.}%
  \label{tab:vjets-mg5py8_ckkwl}
  \centering
  \begin{tabular}{l l}
    \toprule
    DSID range & Description \\
    \midrule
    363123--363146 & \HT-sliced \(Z\to\mu\mu\)   \\
    363147--363170 & \HT-sliced \(Z\to ee\)     \\
    361510--361514 & \(N_\text{parton}\)-sliced \(Z\to\tau\tau\) \\
    361515--361519 & \(N_\text{parton}\)-sliced \(Z\to\nu\nu\)   \\
    363624--363647 & \HT-sliced \(W\to \mu\nu\)   \\
    363600--363623 & \HT-sliced \(W\to e\nu\)    \\
    363648--363671 & \HT-sliced \(W\to\tau\nu\)  \\
    \bottomrule
  \end{tabular}
\end{table}

\paragraph{Short description for \HT-sliced and \(N_\text{parton}\)-sliced \(V\)+jets:}

QCD \(V\)+jets production was simulated with \MGNLO[2.2.2]~\cite{Alwall:2014hca},
using LO-accurate matrix elements (ME) with up to four final-state partons.
The ME calculation employed the \NNPDF[3.0nlo] set of PDFs~\cite{Ball:2014uwa}
(\HT-sliced) / \NNPDF[2.3lo] set of PDFs~\cite{Ball:2012cx} (\(N_\text{parton}\)-sliced).
Events were interfaced to \PYTHIA[8.186]~\cite{Sjostrand:2007gs} for the modelling
of the parton shower, hadronisation, and  underlying event. The overlap between
matrix element and parton shower emissions was removed using the CKKW-L
merging procedure~\cite{Lonnblad:2001iq,Lonnblad:2011xx}. The A14
tune~\cite{ATL-PHYS-PUB-2014-021} of \PYTHIA[8] was used with the
\NNPDF[2.3lo] PDF set~\cite{Ball:2012cx}.
The decays of bottom and charm
hadrons were performed by \EVTGEN[1.2.0]~\cite{Lange:2001uf}.
The \(V\)+jets samples were normalised to a next-to-next-to-leading-order (NNLO)
prediction~\cite{Anastasiou:2003ds}.


\paragraph{\HT-sliced long description:}

QCD \(V\)+jets production was simulated with LO-accurate matrix elements (ME)
for up to four partons with \MGNLO[2.2.2]~\cite{Alwall:2014hca}. The ME calculation was interfaced with
\PYTHIA[8.186]~\cite{Sjostrand:2007gs} for the modelling of the parton
shower, hadronisation, and underlying event. To remove overlap between the matrix
element and the parton shower the CKKW-L merging
procedure~\cite{Lonnblad:2001iq,Lonnblad:2011xx} was applied with a
merging scale of \SI{30}{\GeV} and a jet-clustering radius parameter of
\(0.2\). In order to better model the region of large jet \pT, the
strong coupling constant \alphas was evaluated at the scale of each splitting to
determine the weight. The matrix element calculation was performed with
the \NNPDF[3.0nlo] PDF set~\cite{Ball:2014uwa} with \(\alphas= 0.118\). The calculation was done
in the five-flavour number scheme with massless \(b\)- and
\(c\)-quarks. Quark masses were reinstated in the \PYTHIA[8] parton shower.
The renormalisation and factorisation scales were set to the \MADGRAPH default
values, based on a clustering of the event. The A14
tune~\cite{ATL-PHYS-PUB-2014-021} of \PYTHIA[8] was used with the
\NNPDF[2.3lo] PDF set~\cite{Ball:2012cx} with \(\alphas=0.13\).
The decays of bottom and charm hadrons were performed by \EVTGEN[1.2.0]~\cite{Lange:2001uf}.


\paragraph{\(N_\text{parton}\)-sliced long description:}

QCD \(V\)+jets production was simulated with LO-accurate matrix elements (ME) for up to four partons with
\MGNLO[2.2.2]~\cite{Alwall:2014hca}. The ME calculation was interfaced
with \PYTHIA[8.186]~\cite{Sjostrand:2007gs} for the modelling of the parton
shower and underlying event. To remove overlap between the matrix
element and the parton shower the CKKW-L merging
procedure~\cite{Lonnblad:2001iq,Lonnblad:2011xx} was applied with a
merging scale of \SI{30}{\GeV} and a jet-clustering radius parameter of
\(0.4\). In order to better model the region of large jet \pT, the
strong coupling constant \alphas was evaluated at the scale of each splitting to
determine the weight. The matrix element calculation was performed with
the \NNPDF[2.3lo] PDF set~\cite{Ball:2012cx} with \(\alphas= 0.13\). The calculation was done
in the five-flavour number scheme with massless \(b\)- and
\(c\)-quarks. Quark masses were reinstated in the \PYTHIA[8] parton shower.
The renormalisation and factorisation scales were set to the \MADGRAPH default
values, based on a clustering of the event. The A14 tune~\cite{ATL-PHYS-PUB-2014-021}
of \PYTHIA[8] was used with the \NNPDF[2.3lo] PDF set~\cite{Ball:2012cx} with \(\alphas=0.13\).
The decays of bottom and charm hadrons were performed by \EVTGEN[1.2.0]~\cite{Lange:2001uf}.

% No usable aMC@NLO \FXFX samples exist in ATLAS central production, therefore
% the description is commented out to avoid confusion

%% \section{\MGNLO (\FXFX)}
%% \label{sec:vjets-mg5-fxfx}

%% \begin{table}[!htbp]
%% \caption{Samples with \MGNLOPYTHIA[8] using CKKW-L merging.}%
%% \label{tab:mg5-fxfx}
%% \centering
%% \begin{tabular}{l l}
%% \toprule
%% DSID range & Description \\
%% \midrule
%% 999999-888888 & V+jets HT-sliced\\
%% \bottomrule
%% \end{tabular}
%% \end{center}
%% \end{table}

%% Samples have also been generated using the \MGNLO program to
%% generate matrix elements for \(V\) + 0, 1 and 2 partons at NLO accuracy.
%% The showering and subsequent hadronisation has been performed using \PYTHIA[8.210] with the A14 tune, using the \NNPDF[2.3lo]
%% PDF set with \(\alphas = 0.13\).
%% The different jet multiplicities are merged using the
%% \FXFX prescription~\cite{Frederix:2012ps} implemented in the \MGNLO program
%% (version 2.3.3 is used here).

%% \MGNLO performs a 5FNS calculation with massless \(b\)- and \(c\)-quarks in the matrix element, and massive quarks in the
%% \PYTHIA shower.
%% The PDF input set in use for event generation is \NNPDF[2.3nlo] PDF set with \(\alphas = 0.119\)
%% and the samples have been generated with additional weights for the PDF replicas as well as scale variations of
%% the renormalisation and the factorisation scales~\cite{Frederix:2011ss}.

%% The impact of various merging scales (\muQ) has been studied,
%% analysing three different values: \SI{20}{\GeV} (down variation), \SI{25}{\GeV} (nominal value) and \SI{50}{\GeV} (up variation).
%% At the event-generation level, the minimum jet transverse momentum, \(\pT^j\) is required to be at least
%% \SI{8}{\GeV} with no restriction on the absolute value of the jet pseudorapidity (jet \(|\eta|\)).
%% %The cut on the \(\pT^j\) should be at least half the value used for the merging scale,
%% %in order to not introduce any bias in the MC generation.
%% The samples have been generated using LHAPDF-6.1.5~\cite{Butterworth:2014efa} and FastJet-3.1.0~\cite{Cacciari:2006sm}.


\section[Inclusive Powheg \(V\)]{Inclusive \POWHEG \(V\)}
%\label{sec:v-powheg}

The descriptions below correspond to the samples in \cref{tab:v-powheg}.

\begin{table}[!htbp]
  \caption{Inclusive \(V\) samples with \POWHEG.}%
  \label{tab:v-powheg}
  \centering
  \begin{tabular}{l l}
    \toprule
    DSID range & Description \\
    \midrule
    361100--361108    & \(W^+,W^-,Z/\gamma^\ast\) with \(e,\mu,\tau\) decays\\
    301000--301178, 344722    & high-mass slices: \(W^+,W^-,Z\) with \(e,\mu,\tau\) decays \\
    361664--361669 & \(Z/\gamma^\ast\) low-mass slices (\(m=6\)--10--60\,\si{\GeV})\\
    426335--426336 & \(Z/\gamma^\ast\) high-\(\pTX[][\ell\ell] > \SI{150}{\GeV}\) slices \\
    \bottomrule
  \end{tabular}
\end{table}

\paragraph{Description:}

The \POWHEGBOX[v1] MC generator~\cite{Nason:2004rx,Frixione:2007vw,Alioli:2010xd,Alioli:2008gx}
was used for the simulation at NLO accuracy of the hard-scattering processes of \(W\)
and \(Z\) boson production and decay in the electron, muon, and \(\tau\)-lepton
channels. It was interfaced to \PYTHIA[8.186]~\cite{Sjostrand:2007gs}
for the modelling of the parton shower, hadronisation, and underlying
event, with parameters set according to the \AZNLO
tune~\cite{STDM-2012-23}. The \CT[10nlo] PDF set~\cite{Lai:2010vv} was used
for the hard-scattering processes, whereas the \CTEQ[6L1] PDF
set~\cite{Pumplin:2002vw} was used for the parton shower. The effect of
QED final-state radiation was simulated with \PHOTOSpp[3.52]~\cite{Golonka:2005pn,Davidson:2010ew}.
The \EVTGEN[1.2.0] program~\cite{Lange:2001uf} was used to decay bottom and charm hadrons.

% No usable \POWHEG \MINLO samples exist in ATLAS central production, therefore
% the description is commented out to avoid confusion

%% \section{\POWHEG \MINLO}
%% \label{sec:vjets-minlo}

%% Predictions from \POWHEG \MINLO~\cite{Alioli:2010xd,Hamilton:2012np,Hamilton:2012rf}
%% interfaced to \PYTHIA[8.210]~\cite{Sjostrand:2007gs} with the
%% AZNLO tune~\cite{STDM-2012-23} are obtained to produce
%% \(V\)+jets events, and will be referred to in the remainder of the note
%% as \POWHEG \MINLO+\PYTHIA[8]. The \EVTGEN[1.2.0] program~\cite{Lange:2001uf}
%% is used as an afterburner to better simulate the decays of bottom and charm hadrons.
%% The \PHOTOSpp[3.61] program~\cite{Golonka:2005pn,Davidson:2010ew} is interfaced to \PYTHIA to accurately
%% describe the QED final state radiation, and the emission of lepton pairs is activated in this setup.
%% The PDF set used in \POWHEG is \CT[14nnlo]~\cite{Dulat:2015mca} whereas the PDF set used in the parton shower
%% is the \CTEQ[6L1]~\cite{Pumplin:2002vw} leading order set.

%% \POWHEG \MINLO is an improvement of the CKKW matching procedure to reach NLO accuracy for
%% boson production in association one jet, and smoothly
%% merges to NLO single boson production at low boson \pT, using no external merging scale.
%% NLO contributions from the CKKW Sudakov form factors are subtracted to the virtual corrections provided by \POWHEGBOX.
%% Thanks to the optimal scale and to the Sudakov form factors, the
%% \POWHEG \MINLO samples do not need any Born suppression cut, and their cross-sections are well-behaved at small boson \pT,
%% contrary to fixed order predictions that diverge in this kinematic phase space.

%-------------------------------------------------------------------------------
% Multiboson
\section{Multiboson processes}
%\label{sec:MB}

In the following paragraphs, the set-ups of the current ATLAS multiboson baseline samples are described. 
Details of the full process configuration are given in the PUB note~\cite{ATL-PHYS-PUB-2017-005}.


\subsection[Inclusive Powheg]{Inclusive \POWHEG}
%\label{sec:MB-powheg}

The descriptions below correspond to the samples in Table~\ref{tab:MB-powheg}.

\begin{table}[htbp]
  \caption{Inclusive $VV$ samples with \POWHEG.}%
  \label{tab:MB-powheg}
  \centering
  \begin{tabular}{l l}
    \toprule
    DSID range & Description \\
    \midrule
    361600--361605 & inclusive $WW(2\ell 2\nu), WZ(3\ell\nu), ZZ(4\ell), ZZ(4\nu), ZZ(2\ell 2\nu)$ (all lepton flavours)\\
    361606--361611 & inclusive $WW/WZ/ZZ$ semileptonic decays ($\ell\ell qq$, $\ell\nu qq$, $\nu\nu qq$, all lepton flavours)\\
    \bottomrule
  \end{tabular}
\end{table}

\paragraph{Description:}

The \POWHEGBOX[v2]~\cite{Nason:2004rx,Frixione:2007vw,Alioli:2010xd} generator
was used to simulate the $WW$, $WZ$ and $ZZ$~\cite{Nason:2013ydw} production
processes at NLO accuracy in QCD. The effect of singly resonant
amplitudes and interference effects due to $Z/\gamma^*$ and
same-flavour lepton combinations in the final state were included, where appropriate. 
Interference effects between $WW$ and $ZZ$ for same-flavour charged leptons and
neutrinos were ignored.
Events were interfaced to \PYTHIA[8.186]~\cite{Sjostrand:2007gs}
for the modelling of the parton shower, hadronisation, and underlying
event, with parameters set according to the \AZNLO
tune~\cite{STDM-2012-23}. The \CT[10] PDF set~\cite{Lai:2010vv} was used
for the hard-scattering processes, whereas the \CTEQ[6L1] PDF
set~\cite{Pumplin:2002vw} was used for the parton shower. The \EVTGEN[1.2.0]
program~\cite{Lange:2001uf} was used to decay bottom and charm hadrons.

The factorisation and renormalisation scales were set to the invariant mass of
the boson pair. An invariant mass of $m_{\ell\ell} > \SI{4}{\GeV}$ was required
at matrix-element level for any pair of same-flavour charged leptons.


\subsection[Sherpa]{\SHERPA}
%\label{sec:MB-sherpa}

\subsubsection{Fully leptonic, semileptonic and loop-induced \texorpdfstring{$VV$}{VV}}
%\label{sec:MB-sherpa-fully}

The descriptions below correspond to the samples in
Table~\ref{tab:MB-sherpa}. They describe the almost
identical set-ups of fully leptonic (including loop-induced $VV$ production)
and semileptonic $VV$ decays. For loop-induced processes $gg\to VV$, 
the description below assumes that the `nominal' samples with 
Higgs contributions are used. If you are using specialised set-ups that
exclude the Higgs component, the description should be modified appropriately.
(Get in touch with the PMG Weak Boson Processes subgroup if you are unsure.)

\begin{table}[htbp]
  \caption{$VV$ samples with \SHERPA: fully leptonic, semileptonic,
    loop-induced fully leptonic, loop-induced semileptonic. SFOS stands for
    \enquote{same flavour opposite-charge sign}.}%
  \label{tab:MB-sherpa}
  \centering
  \begin{tabular}{l l}
    \toprule
    DSID range & Description \\
    \midrule
    364250--364255, 363494   & $4\ell, 3\ell\nu, 2\ell 2\nu, 4\nu$ with $m_{\ell\ell}\mathrm{(SFOS)} > \SI{4}{\GeV}$, $\pt^\ell(1,2) > \SI{5}{\GeV}$\\
    364288--364290 & fully leptonic low $m_{\ell\ell}$ and $\pt^\ell$ complement\\
    \midrule
    345705--345727   & loop-induced leptonic\\
    \midrule
    363355-363360, 363489   & semileptonic, on-shell diboson production with factorised decays\\
    \midrule
    364302-364305   & loop-induced semileptonic, using factorised on-shell decays)\\
    \bottomrule
  \end{tabular}
\end{table}

\paragraph{Description:}

Samples of diboson final states ($VV$) were simulated with the
\SHERPA[2.2.1] or 2.2.2~\cite{Bothmann:2019yzt} generator depending on the process,
%~\footnote{This is an admixture of 2.2.1 and above versions, so the version should be kept generic to avoid confusions. As an
%alternative, the sentence can be modified indicating that samples are simulated with the \SHERPA[2.2.1] or 2.2.2 depending on the process.} 
including off-shell effects and Higgs boson contributions, where appropriate.
Fully leptonic final states and semileptonic final states, where one boson
decays leptonically and the other hadronically, were generated using
matrix elements at NLO accuracy in QCD for up to one additional parton
and at LO accuracy for up to three additional parton
emissions. Samples for the loop-induced processes $gg \to VV$ were
generated using LO-accurate matrix elements for up to one
additional parton emission for both the cases of fully leptonic and
semileptonic final states. The matrix element calculations were matched
and merged with the \SHERPA parton shower based on Catani--Seymour
dipole factorisation~\cite{Gleisberg:2008fv,Schumann:2007mg} using the MEPS@NLO
prescription~\cite{Hoeche:2011fd,Hoeche:2012yf,Catani:2001cc,Hoeche:2009rj}.
The virtual QCD corrections were provided by the
\OPENLOOPS library~\cite{Buccioni:2019sur,Cascioli:2011va,Denner:2016kdg}. The
\NNPDF[3.0nnlo] set of PDFs was used~\cite{Ball:2014uwa}, along with the
dedicated set of tuned parton-shower parameters developed by the
\SHERPA authors.


\paragraph{Additional description:}

The ME+PS matching~\cite{Hoeche:2011fd} was employed for different jet
multiplicities which were then merged into an inclusive sample using an
improved CKKW matching procedure~\cite{Catani:2001cc,Hoeche:2009rj}
which is extended to NLO accuracy using the MEPS@NLO
prescription~\cite{Hoeche:2012yf}. The virtual QCD corrections for
matrix elements at NLO accuracy were provided by the
\OPENLOOPS library~\cite{Buccioni:2019sur,Cascioli:2011va,Denner:2016kdg}. The
calculation was performed in the $G_\mu$ scheme, ensuring an
optimal description of pure electroweak interactions at the
electroweak scale.

Uncertainties from missing higher orders were
evaluated~\cite{Bothmann:2016nao} using seven variations of the QCD
factorisation and renormalisation scales in the matrix elements by
factors of \(0.5\) and \(2\), avoiding variations in opposite directions.

Uncertainties in the nominal PDF set were evaluated using 100 replica
variations. Additionally, the results were cross-checked using the
central values of the \CT[14nnlo]~\cite{Dulat:2015mca} and
\MMHT[nnlo]~\cite{Harland-Lang:2014zoa} PDF sets. The effect of the uncertainty
in the strong coupling constant \(\alphas\) was assessed by variations of \(\pm 0.001\).







\subsubsection[Electroweak VVjj]{Electroweak $VVjj$}
%\label{sec:MB-sherpa-vvjj}

The descriptions below correspond to the samples in
Table~\ref{tab:MB-sherpa-vvjj}.

\begin{table}[htbp]
  \caption{Electroweak $VVjj$ samples with \SHERPA.}%
  \label{tab:MB-sherpa-vvjj}
  \centering
  \begin{tabular}{l l}
    \toprule
    DSID range & Description \\
    \midrule
    364283--364284   & $\ell\ell\ell\ell jj$, $\ell\ell\ell\nu jj$\\
    364285   &  $\ell\ell\nu\nu jj$ opposite-sign\\
    364287   &  $\ell\ell\nu\nu jj$ same-sign\\
    366086--366089   &  $\ell\ell\ell\ell jj$, $\ell\ell\ell\nu jj$, $\ell\ell\nu\nu jj$, with the triboson contributions removed\\
    \bottomrule
  \end{tabular}
\end{table}

\paragraph{Description:}

Electroweak production of a diboson in association with two jets
($VVjj$) was simulated with the \SHERPA[2.2.2]~\cite{Bothmann:2019yzt}
generator. The LO-accurate matrix elements were matched to a parton
shower based on Catani--Seymour dipole factorisation~\cite{Gleisberg:2008fv,Schumann:2007mg} using the MEPS@LO
prescription~\cite{Hoeche:2011fd,Hoeche:2012yf,Catani:2001cc,Hoeche:2009rj}.
Samples were generated using the \NNPDF[3.0nnlo] PDF set~\cite{Ball:2014uwa},
along with the dedicated set of tuned parton-shower parameters
developed by the \SHERPA authors.

\paragraph{Additional description:} The calculation was performed in the
$G_\mu$ scheme, ensuring an optimal description of pure
electroweak interactions at the electroweak scale.

Uncertainties from missing higher orders were
evaluated~\cite{Bothmann:2016nao} using seven variations of the QCD
factorisation and renormalisation scales in the matrix elements by
factors of \(0.5\) and \(2\), avoiding variations in opposite directions.

Uncertainties in the nominal PDF set were evaluated using 100 replica
variations. Additionally, the results were cross-checked using the
central values of the \CT[14nnlo]~\cite{Dulat:2015mca} and
\MMHT[nnlo]~\cite{Harland-Lang:2014zoa} PDF sets. The effect of the uncertainty
in the strong coupling constant \(\alphas\) was assessed by variations of \(\pm 0.001\).







\subsubsection[Vgamma (NLO, biased)]{$V\gamma$ (NLO, biased)}
%\label{sec:MB-sherpa-vgamma-nlo-biased}

The descriptions below correspond to the samples in
Table~\ref{tab:MB-sherpa-vynlo-ew}.

\begin{table}[htbp]
  \caption{NLO $V\gamma$ samples with \SHERPA.}%
  \label{tab:MB-sherpa-vynlo-ew}
  \centering
  \begin{tabular}{l l}
    \toprule
    DSID range & Description \\
    \midrule
    700011--700017 & biased in $\log_{10}(\max[\pT(V), \pT(\gamma)])$\\
    \bottomrule
  \end{tabular}
\end{table}

\paragraph{Description:}

The production of $V\gamma$ final states was simulated with the
\SHERPA[2.2.8]~\cite{Bothmann:2019yzt} generator.  Matrix elements at
NLO QCD accuracy for up to one additional parton and LO accuracy for
up to three additional parton emissions were matched and merged with
the \SHERPA parton shower based on Catani--Seymour
dipole factorisation~\cite{Gleisberg:2008fv,Schumann:2007mg} using the MEPS@NLO
prescription~\cite{Hoeche:2011fd,Hoeche:2012yf,Catani:2001cc,Hoeche:2009rj}.
The virtual QCD corrections for matrix elements at NLO accuracy were provided by 
the \OPENLOOPS[2] library~\cite{Buccioni:2019sur,Cascioli:2011va,Buccioni:2017yxi,Denner:2016kdg}.
Samples were generated using the \NNPDF[3.0nnlo] PDF set~\cite{Ball:2014uwa}, along with
the dedicated set of tuned parton-shower parameters developed by the \SHERPA authors.


\paragraph{Additional description:}

The ME+PS matching~\cite{Hoeche:2011fd} was employed for different jet
multiplicities which were then merged into an inclusive sample 
using an improved CKKW matching
procedure~\cite{Catani:2001cc,Hoeche:2009rj} which was extended to NLO
accuracy using the MEPS@NLO prescription~\cite{Hoeche:2012yf}. 
The virtual QCD corrections for matrix elements at NLO accuracy were provided by 
the \OPENLOOPS[2] library~\cite{Buccioni:2019sur,Cascioli:2011va,Buccioni:2017yxi,Denner:2016kdg}.
Multijet merging at NLO accuracy in the electroweak coupling was based on 
the \NLOEWvirt approach~\cite{Kallweit:2014xda,Kallweit:2015dum}.
The calculation was performed in the $G_\mu$ scheme, ensuring an optimal 
description of pure electroweak interactions at the electroweak scale.

Uncertainties from missing higher orders were
evaluated~\cite{Bothmann:2016nao} using seven variations of the QCD
factorisation and renormalisation scales in the matrix elements by
factors of \(0.5\) and \(2\), avoiding variations in opposite directions.

Uncertainties in the nominal PDF set were evaluated using 100 replica
variations. Additionally, the results were cross-checked using the
central values of the \CT[14nnlo]~\cite{Dulat:2015mca} and
\MMHT[nnlo]~\cite{Harland-Lang:2014zoa} PDF sets. The effect of the uncertainty
in the strong coupling constant \(\alphas\) was assessed by variations of \(\pm 0.001\).







\subsubsection[Vgamma (NLO, sliced)]{$V\gamma$ (NLO, sliced)}
%\label{sec:MB-sherpa-vgamma-nlo}

The descriptions below correspond to the samples in
Table~\ref{tab:MB-sherpa-vynlo}.

\begin{table}[htbp]
  \caption{NLO $V\gamma$ samples with \SHERPA.}%
  \label{tab:MB-sherpa-vynlo}
  \centering
  \begin{tabular}{l l}
    \toprule
    DSID range & Description \\
    \midrule
    364500--364535   & sliced in $p_{\text{T}}^{\gamma}$\\ 
    345887--345900   & sliced in $m_{\ell\ell}$\\
    \bottomrule
  \end{tabular}
\end{table}

\paragraph{Description:}

The production of $V\gamma$ final states was simulated with the
\SHERPA[2.2.2]~\cite{Bothmann:2019yzt} generator.  Matrix elements at
NLO QCD accuracy for up to one additional parton and LO accuracy for
up to three additional parton emissions were matched and merged with
the \SHERPA parton shower based on Catani--Seymour
dipole factorisation~\cite{Gleisberg:2008fv,Schumann:2007mg} using the \MEPSatNLO
prescription~\cite{Hoeche:2011fd,Hoeche:2012yf,Catani:2001cc,Hoeche:2009rj}.
The virtual QCD corrections for matrix elements at NLO accuracy were
provided by the
\OPENLOOPS library~\cite{Buccioni:2019sur,Cascioli:2011va,Denner:2016kdg}. Samples
were generated using the \NNPDF[3.0nnlo] PDFset~\cite{Ball:2014uwa}, along with
the dedicated set of tuned parton-shower parameters developed by the
\SHERPA authors.


\paragraph{Additional description:}

The ME+PS matching~\cite{Hoeche:2011fd} was employed for different jet
multiplicities which were then merged into an inclusive sample 
using an improved CKKW matching
procedure~\cite{Catani:2001cc,Hoeche:2009rj} which was extended to NLO
accuracy using the \MEPSatNLO prescription~\cite{Hoeche:2012yf}. 
The virtual QCD corrections for matrix elements at NLO accuracy were
provided by the \OPENLOOPS library~\cite{Buccioni:2019sur,Cascioli:2011va,Denner:2016kdg}. 
The calculation was performed in the $G_\mu$ scheme, ensuring an optimal 
description of pure electroweak interactions at the electroweak scale.

Uncertainties from missing higher orders were
evaluated~\cite{Bothmann:2016nao} using seven variations of the QCD
factorisation and renormalisation scales in the matrix elements by
factors of \(0.5\) and \(2\), avoiding variations in opposite directions.

Uncertainties in the nominal PDF set were evaluated using 100 replica
variations. Additionally, the results were cross-checked using the
central values of the \CT[14nnlo]~\cite{Dulat:2015mca} and
\MMHT[nnlo]~\cite{Harland-Lang:2014zoa} PDF sets. The effect of the uncertainty
in the strong coupling constant \(\alphas\) was assessed by variations of \(\pm 0.001\).







\subsubsection[Vgamma (LO)]{$V\gamma$ (LO)}
%\label{sec:MB-sherpa-vgamma-lo}

The descriptions below correspond to the samples in
Table~\ref{tab:MB-sherpa-vylo}. 

\begin{table}[htbp]
  \caption{LO $V\gamma$ samples with \SHERPA.}%
  \label{tab:MB-sherpa-vylo}
  \centering
  \begin{tabular}{l l}
    \toprule
    DSID range & Description \\
    \midrule
    366140--366154   & \SHERPA[2.2.4], $Z(\to ee/\mu\mu/\tau\tau)\gamma$, sliced in $p_{\mathrm{T}}^{\gamma}$\\
    \bottomrule
  \end{tabular}
\end{table}

\paragraph{Description:}

The production of $V\gamma$ final states was simulated with the
\SHERPA[2.2.4]~\cite{Bothmann:2019yzt} generator. Matrix elements at LO
accuracy in QCD for up to three additional parton emissions were
matched and merged with the \SHERPA parton shower based on
Catani--Seymour dipole factorisation~\cite{Gleisberg:2008fv,Schumann:2007mg} 
using the \MEPSatLO
prescription~\cite{Hoeche:2011fd,Hoeche:2012yf,Catani:2001cc,Hoeche:2009rj}.
Samples were generated using the \NNPDF[3.0nnlo] PDF set~\cite{Ball:2014uwa},
along with the dedicated set of tuned parton-shower parameters
developed by the \SHERPA authors.


\paragraph{Additional description:}

The ME+PS matching~\cite{Hoeche:2011fd} was employed for different jet
multiplicities which were then merged into an inclusive sample using an improved CKKW matching
procedure~\cite{Catani:2001cc,Hoeche:2009rj}. 
The calculation was performed in the $G_\mu$ scheme, ensuring an optimal 
description of pure electroweak interactions at the electroweak scale.

Uncertainties from missing higher orders were
evaluated~\cite{Bothmann:2016nao} using seven variations of the QCD
factorisation and renormalisation scales in the matrix elements by
factors of \(0.5\) and \(2\), avoiding variations in opposite directions.

Uncertainties in the nominal PDF set were evaluated using 100 replica
variations. Additionally, the results were cross-checked using the
central values of the \CT[14nnlo]~\cite{Dulat:2015mca} and
\MMHT[nnlo]~\cite{Harland-Lang:2014zoa} PDF sets. The effect of the uncertainty
in the strong coupling constant \(\alphas\) was assessed by variations of \(\pm 0.001\).







\subsubsection{Tribosons (NLO)}
%\label{sec:MB-sherpa-vvv-nlo}

The descriptions below correspond to the samples in
Table~\ref{tab:MB-sherpa-vvvnlo}.

\begin{table}[htbp]
  \caption{NLO $VVV$ samples (factorised decays) with \SHERPA.}%
  \label{tab:MB-sherpa-vvvnlo}
  \centering
  \begin{tabular}{l l}
    \toprule
    DSID range & Description \\
    \midrule
    363507--363509   & $3\ell1\nu2j$, $4\ell2j$\\
    364242--364249   & fully leptonic decays\\
    364336--364339   & $WWW\to 2\ell2\nu jj$\\
    \bottomrule
  \end{tabular}
\end{table}

\paragraph{Description:}

The production of triboson ($VVV$) events was simulated with the 
\SHERPA[2.2.2]~\cite{Bothmann:2019yzt} generator using factorised gauge-boson decays. 
Matrix elements, accurate to NLO for the inclusive process and to LO for up to 
two additional parton emissions, were matched and merged with the \SHERPA parton 
shower based on Catani--Seymour dipole factorisation~\cite{Gleisberg:2008fv,Schumann:2007mg} 
using the \MEPSatNLO prescription~\cite{Hoeche:2011fd,Hoeche:2012yf,Catani:2001cc,Hoeche:2009rj}. 
The virtual QCD corrections for matrix elements at NLO accuracy were 
provided by the \OPENLOOPS library~\cite{Buccioni:2019sur,Cascioli:2011va,Denner:2016kdg}.
Samples were generated using the \NNPDF[3.0nnlo] PEF set~\cite{Ball:2014uwa}, along with
the dedicated set of tuned parton-shower parameters developed by the \SHERPA authors.


\paragraph{Additional description:}

The ME+PS matching~\cite{Hoeche:2011fd} was employed for different jet
multiplicities which were then merged into an inclusive sample 
using an improved CKKW matching procedure~\cite{Catani:2001cc,Hoeche:2009rj} which is extended to NLO
accuracy using the \MEPSatNLO prescription~\cite{Hoeche:2012yf}.  The
calculation was performed in the $G_\mu$ scheme, ensuring an optimal description of pure electroweak interactions at the
electroweak scale.

Uncertainties from missing higher orders were
evaluated~\cite{Bothmann:2016nao} using seven variations of the QCD
factorisation and renormalisation scales in the matrix elements by
factors of \(0.5\) and \(2\), avoiding variations in opposite directions.

Uncertainties in the nominal PDF set were evaluated using 100 replica
variations. Additionally, the results were cross-checked using the
central values of the \CT[14nnlo]~\cite{Dulat:2015mca} and
\MMHT[nnlo]~\cite{Harland-Lang:2014zoa} PDF sets. The effect of the uncertainty
in the strong coupling constant \(\alphas\) was assessed by variations of \(\pm 0.001\).







\subsubsection{Tribosons (LO)}
%\label{sec:MB-sherpa-vvv-lo}

The descriptions below correspond to the samples in
Table~\ref{tab:MB-sherpa-vvvlo}.

\begin{table}[htbp]
  \caption{LO $VVV$ samples with \SHERPA.}%
  \label{tab:MB-sherpa-vvvlo}
  \centering
  \begin{tabular}{l l}
    \toprule
    DSID range & Description \\
    \midrule
    407311--407315   & on- and off-shell contributions to 6-lepton production\\
    \bottomrule
  \end{tabular}
\end{table}

\paragraph{Description:}

The production of triboson ($VVV$) events was simulated with the
\SHERPA[2.2.1]~\cite{Bothmann:2019yzt} generator. Matrix elements accurate to LO in QCD 
for up to one additional parton emission were matched and merged with the \SHERPA parton shower based on Catani--Seymour
dipole factorisation~\cite{Gleisberg:2008fv,Schumann:2007mg} using the \MEPSatLO
prescription~\cite{Hoeche:2011fd,Hoeche:2012yf,Catani:2001cc,Hoeche:2009rj}.
Samples were generated using the \NNPDF[3.0nnlo] PDF set~\cite{Ball:2014uwa},
along with the dedicated set of tuned parton-shower parameters
developed by the \SHERPA authors.

\paragraph{Additional description:}

The ME+PS matching~\cite{Hoeche:2011fd} was employed for different jet
multiplicities which were then merged into an inclusive sample 
using an improved CKKW matching
procedure~\cite{Catani:2001cc,Hoeche:2009rj}.  The calculation was
performed in the $G_\mu$ scheme, ensuring an optimal description
of pure electroweak interactions at the electroweak scale.

Uncertainties from missing higher orders were
evaluated~\cite{Bothmann:2016nao} using seven variations of the QCD
factorisation and renormalisation scales in the matrix elements by
factors of \(0.5\) and \(2\), avoiding variations in opposite directions.

Uncertainties in the nominal PDF set were evaluated using 100 replica
variations. Additionally, the results were cross-checked using the
central values of the \CT[14nnlo]~\cite{Dulat:2015mca} and
\MMHT[nnlo]~\cite{Harland-Lang:2014zoa} PDF sets. The effect of the uncertainty
in the strong coupling constant \(\alphas\) was assessed by variations of \(\pm 0.001\).






%-------------------------------------------------------------------------------
% Higgs processes                                                                                             
\section{Higgs boson processes}
%\label{sec:higgs}

In this section, the set-up of the current ATLAS samples for Higgs boson production in gluon--gluon fusion, 
vector-boson fusion and Higgs-strahlung processes is described.

\subsection{$H$ via gluon--gluon fusion}
%\label{sec:higgs-ggH}

\subsubsection*{\texorpdfstring{\POWPY[8]}{Powheg+Pythia8} samples}
%\label{sec:higgs-ggH-samples}

%HGam: https://twiki.cern.ch/twiki/bin/view/Sandbox/RuggeroTurraSandbox
%ZZ: https://twiki.cern.ch/twiki/bin/view/AtlasProtected/HZZllllRun2MCSamplesMC16#mc16
%WW: https://twiki.cern.ch/twiki/bin/view/AtlasProtected/HWWMC
%Hlep:https://twiki.cern.ch/twiki/bin/view/AtlasProtected/HSG4MCrequests2015
%Hbb: https://twiki.cern.ch/twiki/bin/view/AtlasProtected/HbbMC

The descriptions below correspond to the samples in Table~\ref{tab:higgs-ggH-samples}.
\begin{table}[!htbp]
\begin{center}
\caption{\POWHEG Higgs gluon--gluon fusion samples with \POWPY[8] for different Higgs boson decay channels.} \label{tab:higgs-ggH-samples}
\begin{tabular}{ l | l | l}
\hline
DSID   & Decay channel & Additional comment \\
\hline
343981 & $H\to\gamma\gamma$&   \\
345316 & $H\to Z\gamma$ & \\
345060 & $H\to ZZ^{*} \to 4\ell $ & $\ell=e,\mu,\tau$   \\
345324 & $H\to WW^{*} \to 2\ell2\nu $ &  $p_{{\textrm T},\ell1} > 15$~\GeV\  and $p_{{\textrm T},\ell2} > 5$~\GeV   \\
308284 & $H\to ZZ^{*}\to 4\nu$ & $E_\text{T}^\text{miss} > 75$~\GeV\\
345342 & $H\to b \bar{b}$ & \\
345097 & $H\to\mu\mu$      &     \\
345120 & $H\to\tau\tau\to \ell^{+}\ell^{-}$      & $p_{{\textrm T},\ell1} > 13$~\GeV\  and $p_{{\textrm T},\ell2} > 7$~\GeV \\
345121 & $H\to\tau\tau\to h^{+}\ell^{-}$      & $p_{{\textrm T},\ell} > 15$~\GeV\  and $p_{{\textrm T},h} > 20$~\GeV\\
345122 & $H\to\tau\tau\to \ell^{+}h^{-}$      & $p_{{\textrm T},\ell} > 15$~\GeV\  and $p_{{\textrm T},h} > 20$~\GeV\\
345123 & $H\to\tau\tau\to h^{+}h^{-}$      & $p_{{\textrm T},h1} > 30$~\GeV\  and $p_{{\textrm T},h2} > 20$~\GeV \\
345124 & $H\to\tau\tau\to \mu\tau$      &\\ %$p_{\textrm T},h1 > 30$~GeV  and $p_{\textrm T},h2 > 20$~GeV 
345125 & $H\to\tau\tau\to e \tau$      & \\ %$p_{\textrm T},h1 > 30$~GeV  and $p_{\textrm T},h2 > 20$~GeV 
\hline
\end{tabular}
\end{center}
\end{table}






\paragraph{Short description:}

Higgs boson production via gluon--gluon fusion was simulated at next-to-next-to-leading-order (NNLO) accuracy in 
QCD using \POWHEGBOX[v2]~\cite{Hamilton:2013fea,Hamilton:2015nsa,Alioli:2010xd,Nason:2004rx,Frixione:2007vw}. 
The simulation achieved NNLO accuracy for arbitrary inclusive $gg\to H$ observables by reweighting the Higgs boson 
rapidity spectrum in \textsc{Hj}-\MINLO~\cite{Hamilton:2012np,Campbell:2012am,Hamilton:2012rf} to that of HNNLO~\cite{Catani:2007vq}.
%The transverse momentum spectrum of the Higgs boson obtained with this sample was found to be compatible with the fixed-order HNNLO calculation and the Hres2.3 calculation~\cite{Bozzi:2005wk,deFlorian:2011xf} performing resummation at next-to-next-to- leading-logarithm accuracy matched to a NNLO fixed-order calculation (NNLL+NNLO).
The \PDFforLHC[15nnlo] PDF set~\cite{Butterworth:2015oua} and the \AZNLO tune~\cite{STDM-2012-23} 
of \PYTHIA[8]~\cite{Sjostrand:2014zea} were used.

The gluon--gluon fusion prediction from the Monte Carlo samples was normalised to the 
next-to-next-to-next-to-leading-order cross-section in QCD plus electroweak corrections 
at next-to-leading order 
(NLO)~\cite{deFlorian:2016spz,Anastasiou:2016cez,Anastasiou:2015ema,Dulat:2018rbf,Harlander:2009mq,Harlander:2009bw,Harlander:2009my,Pak:2009dg,Actis:2008ug,Actis:2008ts,Bonetti:2018ukf,Bonetti:2018ukf}. The decays of bottom and charm hadrons
were performed by \EVTGEN~\cite{Lange:2001uf}.
The normalisation of all Higgs boson
samples accounts for the decay branching ratio calculated with HDECAY~\cite{Djouadi:1997yw,Spira:1997dg,Djouadi:2006bz}
and \PROPHECY~\cite{Bredenstein:2006ha,Bredenstein:2006rh,Bredenstein:2006nk}.


\paragraph{Long description:}

Higgs boson production via gluon--gluon fusion was simulated at next-to-next-to-leading-order (NNLO) accuracy in QCD using 
\POWHEGBOX[v2]~\cite{Hamilton:2013fea,Hamilton:2015nsa,Alioli:2010xd,Nason:2004rx,Frixione:2007vw}. The simulation achieved NNLO
accuracy for arbitrary inclusive $gg\to H$ observables by reweighting the Higgs boson rapidity spectrum 
in \textsc{Hj}-\MINLO~\cite{Hamilton:2012np,Campbell:2012am,Hamilton:2012rf} to that of HNNLO~\cite{Catani:2007vq}. The transverse momentum
spectrum of the Higgs boson obtained with this sample was found to be compatible with the fixed-order HNNLO calculation and the
\HRES[2.3] calculation~\cite{Bozzi:2005wk,deFlorian:2011xf} performing resummation at next-to-next-to-leading-logarithm accuracy
matched to a NNLO fixed-order calculation (NNLL+NNLO). 
Top- and bottom-quark mass effects were included up to next-to-leading order (NLO).

The renormalisation and factorisation scales were set to half of the Higgs boson mass and the \PDFforLHC[15nnlo] PDF
set~\cite{Butterworth:2015oua} was used. The matrix elements were matched to the parton shower of \PYTHIA[8]~\cite{Sjostrand:2014zea}
which uses the AZNLO tune~\cite{STDM-2012-23}. The decays of bottom and charm hadrons
were performed by \EVTGEN~\cite{Lange:2001uf}.

The QCD scale uncertainties were obtained by a three-point scale variation of $\muR=\muF$ by a factor of two about 
the central value for the NNLO part, and using a nine-point variation for the \POWHEGBOX scale. 
PDF and \alphas uncertainties were estimated using the \PDFforLHC[15nlo] set of eigenvectors.
The envelope of the resulting 27 scale variations was taken to estimate the QCD scale uncertainty. 
Uncertainties were also provided for switching off bottom- and top-quark mass effects.
%From YR4

The prediction from the Monte Carlo samples was normalised to the next-to-next-to-next-to-leading-order 
cross-section in QCD in the infinite top-quark mass 
limit~\cite{deFlorian:2016spz,Anastasiou:2016cez,Anastasiou:2015ema,Dulat:2018rbf,Aglietti:2004nj} and including exact 
corrections for all finite quark-mass effects at NLO in QCD as well as NLO electroweak 
effects~\cite{Actis:2008ug,Bonetti:2018ukf}. Additionally, corrections to the inverse of the top-quark mass were taken 
into account at NNLO~\cite{Harlander:2009mq,Harlander:2009bw,Harlander:2009my,Pak:2009dg}. 
The normalisation of all Higgs boson samples accounts for the decay branching ratio calculated with 
HDECAY~\cite{Djouadi:1997yw,Spira:1997dg,Djouadi:2006bz}
and \PROPHECY~\cite{Bredenstein:2006ha,Bredenstein:2006rh,Bredenstein:2006nk}.


\subsection{$H$ via vector-boson fusion}
%\label{sec:higgs-VBF}

\subsubsection*{\texorpdfstring{\POWPY[8]}{Powheg+Pythia8} samples}
%\label{sec:higgs-VBF-samples}
The descriptions below correspond to the samples in Table~\ref{tab:higgs-VBF-samples}.
\begin{table}[!htbp]
\begin{center}
\caption{\POWPY[8] Higgs vector-boson fusion samples for different Higgs boson decay channels.} \label{tab:higgs-VBF-samples}
\begin{tabular}{ l | l | l}
\hline
DSID   & Decay channel & Additional comment \\
\hline
346317 & $H\to$ all  & \\
346214 & $H\to\gamma\gamma$&   \\
345833 & $H\to Z\gamma$ & $Z\to \ell^{+}\ell^{-}$\\
345834 & $H\to \gamma\gamma^{*}$ & $\gamma^{*}\to \ell^{+}\ell^{-}$\\
346228 & $H\to ZZ^{*} \to 4\ell $ & $\ell=e,\mu$   \\
450576 & $H\to ZZ^{*} \to 2\ell2b $ &   \\
345948& $H\to WW^{*} \to 2\ell2\nu $ &  $p_{{\textrm T},\ell1} > 15$~\GeV\  and $p_{{\textrm T},\ell2} > 5$~\GeV   \\
346600& $H\to ZZ^{*}\to 4\nu$ & $\MET>75$~\GeV\\
345949 & $H\to b \bar{b}$ & \\
% & $H\to\mu\mu$      &     \\
346190& $H\to\tau\tau\to \ell^{+}\ell^{-}$      & $p_{{\textrm T},\ell1} > 13$~\GeV\  and $p_{{\textrm T},\ell2} > 7$~\GeV \\
346191 & $H\to\tau\tau\to h^{+}\ell^{-}$      & $p_{{\textrm T},\ell} > 15$~\GeV\  and $p_{{\textrm T},h} > 20$~\GeV\\
346192 & $H\to\tau\tau\to \ell^{+}h^{-}$      & $p_{{\textrm T},\ell} > 15$~\GeV\  and $p_{{\textrm T},h} > 20$~\GeV\\
346193 & $H\to\tau\tau\to h^{+}h^{-}$      & $p_{{\textrm T},h1} > 30$~\GeV\  and $p_{{\textrm T},h2} > 20$~\GeV \\
346194 & $H\to\tau\tau\to e\tau$      & \\%$p_{\textrm T},l1 > 10$~GeV  and $p_{\textrm T},h1 > 20$~GeV \\
346195 & $H\to\tau\tau\to \mu\tau$      & \\%$p_{\textrm T},l1 > 10$~GeV  and $p_{\textrm T},h1 > 20$~GeV \\
\hline
\end{tabular}
\end{center}
\end{table}


\paragraph{Short description:}
Higgs boson production via vector-boson fusion was simulated with
\POWHEGBOX[v2]~\cite{Nason:2009ai,Alioli:2010xd,Nason:2004rx,Frixione:2007vw} 
and interfaced with \PYTHIA[8]~\cite{Sjostrand:2014zea} for parton shower and non-perturbative effects,
with parameters set according to the \AZNLO tune~\cite{STDM-2012-23}.
The \POWHEGBOX prediction is accurate to next-to-leading order (NLO) and uses
the \PDFforLHC[15nlo] PDF set~\cite{Butterworth:2015oua}. 
%It was tuned to match calculations with effects due to finite heavy-quark masses 
%and soft-gluon resummation up to NNLL.
It was normalised to an approximate-NNLO QCD cross-section 
with NLO electroweak corrections~\cite{Ciccolini:2007jr,Ciccolini:2007ec,Bolzoni:2010xr}.
The decays of bottom and charm hadrons were performed by \EVTGEN~\cite{Lange:2001uf}.
The normalisation of all Higgs boson samples accounts for the decay branching ratio calculated 
with \textsc{HDECAY}~\cite{Djouadi:1997yw,Spira:1997dg,Djouadi:2006bz} 
and \PROPHECY~\cite{Bredenstein:2006ha,Bredenstein:2006rh,Bredenstein:2006nk}.




\paragraph{Long description:}
Higgs boson production via vector-boson fusion was simulated with
\POWHEGBOX[v2]~\cite{Nason:2009ai,Alioli:2010xd,Nason:2004rx,Frixione:2007vw}. A factorised approximation, where cross-talk between the
fermion lines is neglected, was used. The implementation is based on the respective NLO QCD calculations for genuine $W/Z$ vector-boson
fusion topologies (VBF approximation). Quark--antiquark annihilation and interference contributions 
between $t$- and $u$-channel contributions were disregarded.
%External fermion masses are neglected throughout.  For the choice of renormalisation andfactorisation scales, various options are available.  For this report, fixed scales,μF=μR=MW, areused, and contributions from external bottom and top quarks are entirely disregarded.

The renormalisation and factorisation scales were set to the $W$ boson mass and the \PDFforLHC[15nlo] PDF set~\cite{Butterworth:2015oua}
was used. The matrix elements were matched to the parton shower of \PYTHIA[8]~\cite{Sjostrand:2014zea} which uses the \AZNLO
tune~\cite{STDM-2012-23}. A dipole-recoil strategy was used for the parton shower.The decays of bottom and charm hadrons
were performed by \EVTGEN~\cite{Lange:2001uf}.%, which provides a better agreement than other recoil schemes with the NNLO predictions of proVBFH (not included in the HiggsReference twiki). % cite https://arxiv.org/pdf/1710.00391.pdf ?,


%The samples include weight variations evaluated on-the-fly for the estimation of QCD scale, PDF- and \alphas-induced uncertainties. 
The QCD scales \muR and \muF were varied independently by factors of 0.5 and 2.0, both in the matrix element 
and in the parton shower.

The prediction from the \POWHEGBOX sample was normalised to the next-to-next-to-leading-order cross-section in QCD 
using the VBF approximation~\cite{Ciccolini:2007jr,Ciccolini:2007ec,Bolzoni:2010xr}. Relative 
next-to-leading-order electroweak corrections were also taken into account for the $t$- and $u$-channel contribution 
considered in the VBF approximation. The normalisation of all Higgs boson samples accounts for the decay branching 
ratio calculated with \textsc{HDECAY}~\cite{Djouadi:1997yw,Spira:1997dg,Djouadi:2006bz} and 
\PROPHECY~\cite{Bredenstein:2006ha,Bredenstein:2006rh,Bredenstein:2006nk}. 


\subsection{$VH$}
%\label{sec:higgs-VH}

This section describes the generation details of nominal samples for Higgs-strahlung production.

\subsubsection*{\texorpdfstring{\POWPY[8]}{Powheg+Pythia8} samples}
%\label{sec:higgs-VH-samples}
The descriptions below correspond to the samples in Tables~\ref{tab:higgs-ZH-samples} to \ref{tab:higgs-ggZH-samples}.

\begin{table}[!htbp]
\begin{center}
\caption{\POWPY[8] samples of Higgs boson production in association with a $Z$ boson for different Higgs boson decay channels} 
\label{tab:higgs-ZH-samples}
\begin{tabular}{ l | l | l}
\hline
DSID   & Decay channel & Additional comment \\
\hline
345038 & $H\to ZZ^{*} \to 4\ell$ & $Z\to$ all\\
345319 & $H\to \gamma\gamma$ & $Z\to$ all\\
345322 & $H\to Z\gamma$ & $Z\to$ all\\
345103 & $H\to\mu\mu$ & $Z\to$ all\\ 
345217 & $H\to\tau\tau$& $Z\to$ all\\ 
345218 & $H\to e\tau$ &$Z\to$ all\\
345219 & $H\to \mu\tau$&$Z\to$ all\\
345445 &$H\to WW^{*} \to 2\ell2\nu $ & $Z\to$ all \\
345876 & $H\to ee$ &$Z\to$ all \\
345965 &  $H\to \gamma\gamma^{*}$ & $\gamma^{*}\to \ell^{+}\ell^{-}$; $Z\to$ all\\
346310 & $H\to$ all&$Z\to$ all\\
346607 & $H\to ZZ^{*} \to 4\nu $ &$Z\to \ell^{+}\ell^{-}$ \\
345055 & $H\to b\bar{b}$ & $p_{{\textrm T},Z}$ enhancement; $Z\to \ell^{+}\ell^{-}$\\
345111 & $H\to c\bar{c}$ & $p_{{\textrm T},Z}$ enhancement; $Z\to \ell^{+}\ell^{-}$\\
345337 & $H\to WW^{*} \to 2\ell2\nu $ &$p_{{\textrm T},Z}$ enhancement; $Z\to \ell^{+}\ell^{-}$\\
346326 & $H\to\tau\tau$& $p_{{\textrm T},Z}$ enhancement; $Z\to \ell^{+}\ell^{-}$\\
346693 & $H\to ZZ^{*} \to 4\nu $ & $p_{{\textrm T},Z}$ enhancement; $Z\to \ell^{+}\ell^{-}$\\
345056 & $H\to b\bar{b}$ & $p_{{\textrm T},Z}$ enhancement; $Z\to \nu\bar{\nu}$\\
345112 & $H\to c\bar{c}$ & $p_{{\textrm T},Z}$ enhancement; $Z\to \nu\bar{\nu}$\\
345445 & $H\to WW^{*} \to 2\ell2\nu$  & $p_{{\textrm T},Z}$ enhancement; $Z\to \nu\bar{\nu}$\\
\hline
\end{tabular}
\end{center}
\end{table}


\begin{table}[!htbp]
\begin{center}
\caption{ \POWPY[8] samples of Higgs boson production in association with a $W^{+}$ boson for different Higgs boson decay channels} 
%\label{tab:higgs-WpH-samples}
\begin{tabular}{ l | l | l}
\hline
DSID   & Decay channel & Additional comment \\
\hline
345039 & $H\to ZZ^{*} \to 4\ell$ & $W^{+}\to$ all\\
345318 & $H\to \gamma\gamma$ & $W^{+}\to$ all\\
345104 & $H\to \mu\mu$ &  $W^{+}\to$ all\\
345212 & $H\to \tau\tau$ &  $W^{+}\to$ all\\
345214 & $H\to e \tau $ & $W^{+}\to$ all\\
345216 & $H\to \mu \tau $ & $W^{+}\to$ all\\
345321 & $H\to Z\gamma$ & $W^{+}\to$ all\\
345325 & $H\to WW^{*} \to 2\ell2\nu $ & $W^{+} \to q\bar{q}$ \\
345877 & $H\to e^{+}e^{-}$ & $W^{+}\to$ all\\
345964 & $H\to \gamma\gamma^{*}$ & $\gamma^{*}\to \ell^{+}\ell^{-}$; $W^{+}\to$ all\\
346311 & $H\to$ all & $W^{+}\to$ all\\ 
346605 & $H\to ZZ^{*} \to 4\nu $ & $W^{+}\to$ all\\
346699 & $H \to 4\ell$ & interfaced to PROPHECY4F \\
346705 & $H \to 4\ell$ & interfaced to Hto4l \\
345054 & $H\to b\bar{b}$ & $p_{{\textrm T},W}$ enhancement; $W^{+}\to \ell^{+}\nu$ \\
345110 & $H\to c\bar{c}$ & $p_{{\textrm T},W}$ enhancement; $W^{+}\to \ell^{+}\nu$ \\
345327 & $H\to WW^{*} \to 2\ell2\nu$ & $p_{{\textrm T},W}$ enhancement; $W^{+}\to \ell^{+}\nu$ \\ 
346561 & $H\to WW^{*} \to q\bar{q}\ell\nu$ & $p_{{\textrm T},W}$ enhancement; $W^{+}\to \ell^{+}\nu$ \\
346325 & $H\to \tau\tau$ & $p_{{\textrm T},W}$ enhancement; $W^{+}\to \ell^{+}\nu$ \\
346729 & $H\to ZZ^{*} \to 4\nu$ & $p_{{\textrm T},W}$ enhancement; $W^{+}\to \ell^{+}\nu$ \\
\hline
\end{tabular}
\end{center}
\end{table}

\begin{table}[!htbp]
\begin{center}
\caption{\POWPY[8] samples of Higgs boson production in association with a $W^{-}$ boson for different Higgs boson decay channels} 
%\label{tab:higgs-WmH-samples}
\begin{tabular}{ l | l | l}
\hline
DSID   & Decay channel & Additional comment \\
\hline
345040 & $H\to ZZ^{*} \to 4\ell$ & $W^{-}\to$ all\\
345317 & $H\to \gamma\gamma$ & $W^{-}\to$ all\\
345105 & $H\to \mu\mu$ &  $W^{-}\to$ all\\
345211 & $H\to \tau\tau$ &  $W^{-}\to$ all\\
345213 & $H\to e \tau $ & $W^{-}\to$ all\\
345215 & $H\to \mu \tau $ & $W^{-}\to$ all\\
345320 & $H\to Z\gamma$ & $W^{-}\to$ all\\
345333 & $H\to WW^{*} \to 2\ell2\nu $ & $W^{-} \to q\bar{q}$ \\
345878 & $H\to e^{+}e^{-}$ & $W^{-}\to$ all\\
345963 & $H\to \gamma\gamma^{*}$ & $\gamma^{*}\to \ell^{+}\ell^{-}$; $W^{-}\to$ all\\
346312 & $H\to$ all & $W^{-}\to$ all\\ 
346606 & $H\to ZZ^{*} \to 4\nu $ & $W^{-}\to$ all\\
346700 & $H \to 4\ell$ & interfaced to PROPHECY4F \\
346706 & $H \to 4\ell$ & interfaced to Hto4l \\
345053 & $H\to b\bar{b}$ & $p_{{\textrm T},W}$ enhancement; $W^{-}\to \ell^{-}\bar{\nu}$ \\
345109 & $H\to c\bar{c}$ & $p_{{\textrm T},W}$ enhancement; $W^{-}\to \ell^{-}\bar{\nu}$  \\
345326 & $H\to WW^{*} \to 2\ell2\nu$ & $p_{{\textrm T},W}$ enhancement;  $W^{-}\to \ell^{-}\bar{\nu}$ \\
346560 & $H\to WW^{*} \to q\bar{q}\ell\nu$ & $p_{{\textrm T},W}$ enhancement; $W^{-}\to \ell^{-}\bar{\nu}$ \\
346324 & $H\to \tau\tau$ & $p_{{\textrm T},W}$ enhancement; $W^{-}\to \ell^{-}\bar{\nu}$ \\
346730 & $H\to ZZ^{*} \to 4\nu$ & $p_{{\textrm T},W}$ enhancement; $W^{-}\to \ell^{-}\bar{\nu}$ \\
\hline
\end{tabular}
\end{center}
\end{table}

\begin{table}[!htbp]
\begin{center}
\caption{\POWPY[8] samples of loop-induced Higgs boson production in association with a $Z$ boson for different
Higgs boson decay channels} 
\label{tab:higgs-ggZH-samples}
\begin{tabular}{ l | l | l}
\hline
DSID   & Decay channel & Additional comment \\
\hline
345061 & $H\to \gamma\gamma$ & $Z\to$ all \\
345066 & $H\to ZZ^{*} \to 4\ell$ & $Z\to$ all \\
345098 & $H\to \mu\mu$ & $Z\to$ all \\
345596 & $H\to ZZ^{*} \to 4\nu$ & $Z\to$ all \\
346524 & $H\to WW^{*} \to 2\ell2\nu $ & $Z\to$ all \\
346697 & $H\to ZZ^{*} \to 4\nu$ & interfaced to PROPHECY4F \\
346703 & $H\to ZZ^{*} \to 4\nu$ & interfaced to Hto4l \\
345057 & $H\to b\bar{b}$ &  $Z\to \ell^{+}\ell^{-}$\\ 
345113 &  $H\to c\bar{c}$ & $Z\to \ell^{+}\ell^{-}$\\
345446 & $H\to WW^{*} \to 2\ell2\nu$ &  $Z\to \ell^{+}\ell^{-}$\\
346329 & $H\to \tau\tau$ &  $Z\to \ell^{+}\ell^{-}$\\
346694 & $H\to ZZ^{*} \to 4\nu$ &  $Z\to \ell^{+}\ell^{-}$\\
345058 & $ H\to b\bar{b}$ & $Z\to \nu\bar{\nu}$\\
345114 & $ H\to c\bar{c}$ & $Z\to \nu\bar{\nu}$\\
\hline
\end{tabular}
\end{center}
\end{table}





\paragraph{Short description:}

Higgs boson production in association with a vector boson was simulated using
\POWHEGBOX[v2]~\cite{Nason:2009ai,Alioli:2010xd,Nason:2004rx,Frixione:2007vw} and interfaced with \PYTHIA[8]~\cite{Sjostrand:2014zea} for
parton shower and non-perturbative effects. The \POWHEGBOX prediction is accurate to next-to-leading order for $VH$ boson plus one-jet production. 
The loop-induced $gg\to ZH$ process was generated separately at leading order. The \PDFforLHC[15nlo] PDF
set~\cite{Butterworth:2015oua} and the \AZNLO tune~\cite{STDM-2012-23} of \PYTHIA[8]~\cite{Sjostrand:2014zea} were used. The decays of bottom and charm hadrons
were performed by \EVTGEN~\cite{Lange:2001uf}. The Monte Carlo
prediction was normalised to cross-sections calculated at NNLO in QCD with NLO electroweak corrections for $q\bar{q}/qg \to VH$ and at NLO
and next-to-leading-logarithm accuracy in QCD for $gg \to
ZH$~\cite{Ciccolini:2003jy,Brein:2003wg,Brein:2011vx,Altenkamp:2012sx,Denner:2014cla,Brein:2012ne,Harlander:2014wda}. The
normalisation of all Higgs boson samples accounts for the decay branching ratio calculated with 
HDECAY~\cite{Djouadi:1997yw,Spira:1997dg,Djouadi:2006bz} and \PROPHECY~\cite{Bredenstein:2006ha,Bredenstein:2006rh,Bredenstein:2006nk}.



\paragraph{Long description:}
Higgs boson production in association with a vector boson was simulated using
\POWHEGBOX[v2]~\cite{Nason:2009ai,Alioli:2010xd,Nason:2004rx,Frixione:2007vw}. The computation was carried out using the
\MINLO~\cite{Luisoni:2013cuh} prescription, which achieves NLO accuracy for the $VH$ and $VH$ boson plus one-jet production
inclusive distributions and dictates the choice of renormalisation and factorisation scales. Virtual amplitudes were constructed
through the interface to the \GOSAM package~\cite{Cullen:2011ac}. The loop-induced $gg\to ZH$ process was generated separately at
leading order with \POWHEGBOX. In all cases, the \PDFforLHC[15nlo] PDF set~\cite{Butterworth:2015oua} was used.

The matrix elements were matched to the parton shower of \PYTHIA[8]~\cite{Sjostrand:2014zea} which uses the \AZNLO
tune~\cite{STDM-2012-23}. The decays of bottom and charm hadrons were performed by \EVTGEN~\cite{Lange:2001uf}.
%The samples include weight variations evaluated on-the-fly for the estimation of QCD scale, PDF- and \alphas-induced uncertainties. 
The QCD scales \muR and \muF were varied independently by factors of 0.5 and 2.0 to account for their uncertainties.

The predictions from \POWHEGBOX were normalised to the best available theoretical prediction. The  $q\bar{q}/qg \to VH$ cross-sections
were calculated at NNLO in QCD with NLO electroweak corrections and the $gg \to ZH$ cross-sections were calculated at NLO and 
next-to-leading-logarithm accuracy~\cite{Ciccolini:2003jy,Brein:2003wg,Brein:2011vx,Altenkamp:2012sx,Denner:2014cla,Brein:2012ne,Harlander:2014wda}. 
The normalisation of the $q\bar{q}\to ZH$ samples was extracted from the subtraction of the latter from the former. 
Differential NLO EW corrections were available from the \textsc{HAWK} program~\cite{Denner:2014cla} to be applied to 
$q\bar{q}$-initiated $VH$ production as a function of the vector boson's transverse momentum.
The normalisation of all Higgs boson samples accounts for the decay branching ratio calculated with
\textsc{HDECAY}~\cite{Djouadi:1997yw,Spira:1997dg,Djouadi:2006bz} and 
\PROPHECY~\cite{Bredenstein:2006ha,Bredenstein:2006rh,Bredenstein:2006nk}.



%-------------------------------------------------------------------------------
% Top
% !TEX root = MC_snippets.tex

\chapter{Top-quark processes}
%\label{sec:TQ}

This chapter describes the samples used for top-quark processes.
The \ttbar\ samples are described in \cref{subsec:ttbar}.
Single-top samples are described in \cref{subsec:tW} for $tW$ associated production,
in \cref{subsec:tchan} for $t$-channel production,
and in \cref{subsec:schan} for $s$-channel production.
Finally, \ttHF\ samples are described in \cref{subsec:ttHF}.

% !TEX root = MC_snippets.tex

%%%%%%%%%%%%%%%%%%%%%%%%%%%%%%%%%%%%%%%%%%%
%%%              ttbar                  %%%
%%%%%%%%%%%%%%%%%%%%%%%%%%%%%%%%%%%%%%%%%%%
\section[\ttbar production]{\ttbar production}
\label{subsec:ttbar}

This section describes the MC samples used for the modelling of \ttbar production.
\Cref{subsubsec:ttbar_PP8} describes the \POWPY[8] samples,
\cref{subsubsec:ttbar_PH7} describes the \POWHER[7] samples,
\cref{subsubsec:ttbar_aMCP8} describes the \MGNLOPY[8] samples,
and finally \cref{subsubsec:ttbar_sherpa} describes the \SHERPA samples.

The reference cross-section values are extracted from Ref.~\cite{LHCTopWGttbarXsec}.
Studies of MC simulation performance including comparisons with unfolded data 
are collected in the PUB notes~\cite{ATL-PHYS-PUB-2018-009,ATL-PHYS-PUB-2017-007,ATL-PHYS-PUB-2016-020}.

\subsection[Powheg+Pythia8]{\POWPY[8]}
\label{subsubsec:ttbar_PP8}

\paragraph{Samples}
%\label{par:ttbar_PP8_samples}

The descriptions below correspond to the samples in \cref{tab:ttbar_PP8,tab:ttbar_PP8_addRad,tab:ttbar_PP8_hdampvar}.

\begin{table}[htbp]
  \caption{Nominal \ttbar samples produced with \POWPY[8].
    The \hdamp\ value is set to 1.5\,\mtop.}%
  \label{tab:ttbar_PP8}
  \centering
  \begin{tabular}{l l}
    \toprule
    DSID range & Description \\
    \midrule
    410470 & \ttbar non-all-hadronic \\
    410471 & \ttbar dileptonic \\
    410472 & \ttbar all-hadronic \\
    \bottomrule
  \end{tabular}
\end{table}

\begin{table}[htbp]
  \caption{\ttbar samples produced with \POWPY[8]
    used to estimate initial-state radiation systematic uncertainties. 
    The \hdamp value is set to 3.0\,\mtop.}%
  \label{tab:ttbar_PP8_addRad}
  \centering
  \begin{tabular}{l l}
    \toprule
    DSID range & Description \\
    \midrule
    410480 & \ttbar single lepton \\
    410481 & \ttbar all-hadronic \\
    410482 & \ttbar dileptonic \\
    \bottomrule
  \end{tabular}
\end{table}

\begin{table}[htbp]
  \caption{\ttbar samples produced with \POWPY[8] with alternative \hdamp values
    which can be used to estimate the uncertainty due to the \POWPY[8] matching scheme.}%
  \label{tab:ttbar_PP8_hdampvar}
  \centering
  \begin{tabular}{l l l l}
    \toprule
    DSID (1.3\,\mtop) & DSID (1.8\,\mtop) & DSID (2.0\,\mtop) & Description \\
    \midrule
    411350 & 411353 & 411356 & \ttbar single lepton \\
    411351 & 411354 & 411357 & \ttbar all-hadronic \\
    411352 & 411355 & 411358 & \ttbar dileptonic \\
    \bottomrule
  \end{tabular}
\end{table}

\paragraph{Short description:}

The production of \ttbar events was modelled using the
\POWHEGBOX[v2]~\cite{Frixione:2007nw,Nason:2004rx,Frixione:2007vw,Alioli:2010xd}
generator at NLO with the \NNPDF[3.0nlo]~\cite{Ball:2014uwa} PDF set
and the \hdamp parameter\footnote{The
  \hdamp parameter is a resummation damping factor and one of the
  parameters that controls the matching of \POWHEG matrix elements to
  the parton shower and thus effectively regulates the
  high-\pT radiation against which the \ttbar system recoils.} set
to 1.5\,\mtop~\cite{ATL-PHYS-PUB-2016-020}.  The events were interfaced
to \PYTHIA[8.230]~\cite{Sjostrand:2014zea} to model the parton shower,
hadronisation, and underlying event, with parameters set according
to the A14 tune~\cite{ATL-PHYS-PUB-2014-021} and using the \NNPDF[2.3lo]
set of PDFs~\cite{Ball:2012cx}. The decays of bottom and charm hadrons
were performed by \EVTGEN[1.6.0]~\cite{Lange:2001uf}.

The uncertainty due to initial-state radiation (ISR) was estimated by
simultaneously varying the \hdamp parameter and the \muR and
\muF scales, and choosing the Var3c up/down variants of the A14 tune
as described in Ref.~\cite{ATL-PHYS-PUB-2017-007}. The impact of
final-state radiation (FSR) was evaluated by varying the renormalisation scale 
for emissions from the parton shower up or down by a factor two.



\paragraph{Long description:}

The production of \ttbar events was modelled using the
\POWHEGBOX[v2]~\cite{Frixione:2007nw,Nason:2004rx,Frixione:2007vw,Alioli:2010xd}
generator, which provided matrix elements at next-to-leading
order~(NLO) in the strong coupling constant \alphas, and the
\NNPDF[3.0nlo]~\cite{Ball:2014uwa} parton distribution function~(PDF). 
The \hdamp parameter, which controls the matching in \POWHEG and
effectively regulates the high-\pT radiation against which the
\ttbar system recoils, was set to 1.5\,\mtop~\cite{ATL-PHYS-PUB-2016-020}.
The functional form of the renormalisation and factorisation scales was
set to the default scale $\sqrt{\mtop^{2} + \pTX[2]}$.
The events were interfaced with
\PYTHIA[8.230]~\cite{Sjostrand:2014zea} for the parton shower and
hadronisation, using the A14 set of tuned
parameters~\cite{ATL-PHYS-PUB-2014-021} and the \NNPDF[2.3lo]
set of PDFs~\cite{Ball:2012cx}.
The decays of bottom and charm hadrons were simulated using the
\EVTGEN[1.6.0] program~\cite{Lange:2001uf}.

The \ttbar sample was normalised to the cross-section prediction at next-to-next-to-leading order (NNLO)
in QCD including the resummation of next-to-next-to-leading logarithmic (NNLL) soft-gluon terms calculated using
\TOPpp[2.0]~\cite{Beneke:2011mq,Cacciari:2011hy,Baernreuther:2012ws,Czakon:2012zr,Czakon:2012pz,Czakon:2013goa,Czakon:2011xx}.
For proton--proton collisions at a centre-of-mass energy of $\rts = \SI{13}{\TeV}$, this cross-section corresponds to
$\sigma(\ttbar)_{\text{NNLO+NNLL}} = 832 \pm 51\,\si{\fb}$ using a top-quark mass of $\mtop = \SI{172.5}{\GeV}$.
The uncertainties in the cross-section due to the PDF and \alphas were calculated using the \PDFforLHC[15] prescription~\cite{Butterworth:2015oua}
with the \MSTW[nnlo]~\cite{Martin:2009iq,Martin:2009bu}, \CT[10nnlo]~\cite{Lai:2010vv,Gao:2013xoa} 
and \NNPDF[2.3lo]~\cite{Ball:2012cx} PDF sets in the five-flavour scheme, and were added in quadrature to the effect of the scale uncertainty.

The uncertainty due to initial-state radiation (ISR) was estimated by
comparing the nominal \ttbar sample with two additional
samples~\cite{ATL-PHYS-PUB-2017-007}.  To simulate higher parton
radiation, the factorisation and renormalisation scales were reduced by
a factor of 0.5 while simultaneously increasing the \hdamp value to
3.0\,\mtop and using the Var3c up variation from the A14 tune. For
lower parton radiation, \muR and \muF were increased by a factor of two
while keeping the \hdamp value set to 1.5\,\mtop and using the Var3c down
variation in the parton shower.  The Var3c A14 tune
variation~\cite{ATL-PHYS-PUB-2014-021} largely corresponds to the variation of
\alphas for ISR in the A14 tune. 
The impact of final-state radiation (FSR) was evaluated by
varying the renormalisation scale for emissions from the
parton shower up and down by a factor of two.


The \NNPDF[3.0lo] replicas were used to evaluate the PDF uncertainties for the nominal PDF.
In addition, the central value of this PDF was compared with the central values of the 
\CT[14nnlo]~\cite{Dulat:2015mca} and \MMHT[nnlo]~\cite{Harland-Lang:2014zoa} PDF sets.


\subsection[Powheg+Herwig7.04]{\POWHER[7.04]}
\label{subsubsec:ttbar_PH7}

\paragraph{Samples}
%\label{par:ttbar_PH7_samples}

The descriptions below correspond to the samples in \cref{tab:ttbar_PH7}.

\begin{table}[htbp]
  \caption{\ttbar samples produced with \POWHER[7].}%
  \label{tab:ttbar_PH7}
  \centering
  \begin{tabular}{l l}
    \toprule
    DSID range & Description \\
    \midrule
    410557 & \ttbar single lepton  \\
    410558 & \ttbar dileptonic  \\
    410559 & \ttbar all-hadronic  \\
    \bottomrule
  \end{tabular}
\end{table}

\paragraph{Short description:}

The impact of using a different parton shower and hadronisation model was evaluated
by comparing the nominal \ttbar sample with another event sample produced with the
\POWHEGBOX[v2]~\cite{Frixione:2007nw,Nason:2004rx,Frixione:2007vw,Alioli:2010xd}
generator using the \NNPDF[3.0nlo]~\cite{Ball:2014uwa} parton distribution function~(PDF). 
Events in the latter sample were interfaced with \HERWIG[7.04]~\cite{Bahr:2008pv,Bellm:2015jjp}, 
using the H7UE set of tuned parameters~\cite{Bellm:2015jjp} and the 
\MMHT[lo] PDF set~\cite{Harland-Lang:2014zoa}.
The decays of bottom and charm hadrons
were simulated using the \EVTGEN[1.6.0] program~\cite{Lange:2001uf}. 

\paragraph{Long description:}

The impact of using a different parton shower and hadronisation model was evaluated
by comparing the nominal \ttbar sample with an event sample also produced with the
\POWHEGBOX[v2]~\cite{Frixione:2007nw,Nason:2004rx,Frixione:2007vw,Alioli:2010xd} 
generator but interfaced with \HERWIG[7.04]~\cite{Bahr:2008pv,Bellm:2015jjp}, using the H7UE set
of tuned parameters~\cite{Bellm:2015jjp} and the \MMHT[lo] PDF set~\cite{Harland-Lang:2014zoa}. 
\POWHEGBOX provided matrix elements at next-to-leading order~(NLO) in the 
strong coupling constant \alphas, and used the \NNPDF[3.0nlo]~\cite{Ball:2014uwa} 
parton distribution function~(PDF) and an \hdamp parameter value of $1.5\,\mtop$~\cite{ATL-PHYS-PUB-2016-020}.
The functional form of the renormalisation and factorisation scales was
set to the default scale $\sqrt{\mtop^{2} + \pT^2}$.
The decays of bottom and charm hadrons
were simulated using the \EVTGEN[1.6.0] program~\cite{Lange:2001uf}. 


\subsection[Powheg+Herwig7.13]{\POWHER[7.13]}
%\label{subsubsec:ttbar_PH713}

\paragraph{Samples}
%\label{par:ttbar_PH713_samples}

The descriptions below correspond to the samples in \cref{tab:ttbar_PH713}.

\begin{table}[htbp]
  \caption{\ttbar samples produced with \POWHER[7.13].}%
  \label{tab:ttbar_PH713}
  \centering
  \begin{tabular}{l l}
    \toprule
    DSID range & Description \\
    \midrule
    411233 & \ttbar single lepton  \\
    411234 & \ttbar dileptonic  \\
    411316 & \ttbar all-hadronic  \\
    \bottomrule
  \end{tabular}
\end{table}

\paragraph{Short description:}

The impact of using a different parton shower and hadronisation model was evaluated
by comparing the nominal \ttbar sample with another event sample produced with the
\POWHEGBOX[v2]~\cite{Frixione:2007nw,Nason:2004rx,Frixione:2007vw,Alioli:2010xd}
generator using the \NNPDF[3.0nlo]~\cite{Ball:2014uwa} parton distribution function~(PDF). 
Events in the latter sample were interfaced with \HERWIG[7.13]~\cite{Bahr:2008pv,Bellm:2015jjp}, 
using the \HERWIG[7.1] default set of tuned parameters~\cite{Bellm:2015jjp,Bellm:2017jjp} 
and the \MMHT[lo] PDF set~\cite{Harland-Lang:2014zoa}.
The decays of bottom and charm hadrons
were simulated using the \EVTGEN[1.6.0] program~\cite{Lange:2001uf}. 


\paragraph{Long description:}

The impact of using a different parton shower and hadronisation model was evaluated
by comparing the nominal \ttbar sample with an event sample also produced with the
\POWHEGBOX[v2]~\cite{Frixione:2007nw,Nason:2004rx,Frixione:2007vw,Alioli:2010xd}
generator but interfaced with \HERWIG[7.13]~\cite{Bahr:2008pv,Bellm:2015jjp}, using the 
\HERWIG[7.1] default set of tuned parameters~\cite{Bellm:2015jjp,Bellm:2017jjp} and the 
\MMHT[lo] PDF set~\cite{Harland-Lang:2014zoa}. 
\POWHEGBOX provided matrix elements at next-to-leading
order~(NLO) in the strong coupling constant \alphas, and used the
\NNPDF[3.0nlo]~\cite{Ball:2014uwa} parton distribution function~(PDF) and
an \hdamp parameter value of 1.5\,\mtop~\cite{ATL-PHYS-PUB-2016-020}.
The functional form of the renormalisation and factorisation scales was
set to the default scale $\sqrt{\mtop^{2} + \pTX[2]}$.  
The decays of bottom and charm hadrons
were simulated using the \EVTGEN[1.6.0] program~\cite{Lange:2001uf}. 


\subsection[MadGraph5\_aMC@NLO+Pythia8]{\MGNLOPY[8]}
\label{subsubsec:ttbar_aMCP8}

\paragraph{Samples}
%\label{par:ttbar_aMCP8_samples}

The descriptions below correspond to the samples in \cref{tab:ttbar_aMCP8}.
\begin{table}[htbp]
  \caption{\ttbar samples produced with \MGNLOPY[8].}%
  \label{tab:ttbar_aMCP8}
  \centering
  \begin{tabular}{l l}
  \toprule
  DSID range & Description \\
  \midrule
  410464 & \ttbar single lepton \\
  410465 & \ttbar dileptonic \\
  410466 & \ttbar all-hadronic \\
  \bottomrule
  \end{tabular}
\end{table}

\paragraph{Short description:}

To assess the uncertainty in the matching of NLO matrix elements to the
parton shower, the \POWHEG sample was compared with a sample of events
generated with \MGNLO[2.6.0]~\cite{Alwall:2014hca} interfaced with
\PYTHIA[8.230]~\cite{Sjostrand:2014zea}. The \MGNLO calculation used the
\NNPDF[3.0nlo] set of PDFs~\cite{Ball:2014uwa} and \PYTHIA[8] used
the A14 set of tuned parameters~\cite{ATL-PHYS-PUB-2014-021} and
the \NNPDF[2.3lo] set of PDFs~\cite{Ball:2012cx}.
The decays of bottom and charm hadrons
were simulated using the \EVTGEN[1.6.0] program~\cite{Lange:2001uf}. 

\paragraph{Long description:}

To assess the uncertainty due to the choice of matching scheme,
the \POWHEG sample was compared with a sample generated by 
\MGNLOPY[8]. For the calculation of the hard-scattering,
\MGNLO[2.6.0]~\cite{Alwall:2014hca} with the \NNPDF[3.0nlo]~\cite{Ball:2014uwa} PDF set was
used. The events were interfaced with
\PYTHIA[8.230]~\cite{Sjostrand:2014zea}, using the A14 set of tuned
parameters~\cite{ATL-PHYS-PUB-2014-021} and the \NNPDF[2.3lo] set of PDFs~\cite{Ball:2012cx}.
Top quarks were decayed at LO using
\MADSPIN~\cite{Frixione:2007zp,Artoisenet:2012st} to preserve spin
correlations. The decays of bottom and charm hadrons were simulated
using the \EVTGEN[1.6.0] program~\cite{Lange:2001uf}.  The parton-shower starting
scale had the functional form $\muQ = \HT/2$~\cite{ATL-PHYS-PUB-2017-007}, 
where \HT is defined as the scalar sum of the \pT of all outgoing partons.  
The renormalisation and factorisation scale choice was the same as for the
\POWHEGBOX set-up.

\subsection[MadGraph5\_aMC@NLO+Herwig7.13]{\MGNLOHER[7.13]}
%\label{subsubsec:ttbar_aMCH713}

\paragraph{Samples}
%\label{par:ttbar_aMCH713_samples}

The descriptions below correspond to the samples in \cref{tab:ttbar_aMCH713}.
\begin{table}[htbp]
  \caption{\ttbar samples produced with \MGNLOHER[7.13].}%
  \label{tab:ttbar_aMCH713}
  \centering
  \begin{tabular}{l l}
    \toprule
    DSID range & Description \\
    \midrule
    412116 & \ttbar single lepton \\
    412117 & \ttbar dileptonic \\
    412175 & \ttbar all-hadronic \\
    \bottomrule
  \end{tabular}
\end{table}

\paragraph{Short description:}

To assess the uncertainty in the matching of NLO matrix elements to the
parton shower, a sample produced with the
\POWHEGBOX[v2]
generator was compared with a sample generated with \MGNLO[2.6.0]~\cite{Alwall:2014hca}, both using the 
\NNPDF[3.0nlo]~\cite{Ball:2014uwa} parton distribution function~(PDF) and interfaced with
\HERWIG[7.13]~\cite{Bahr:2008pv,Bellm:2015jjp}, using the \HERWIG[7.1] default set of 
tuned parameters~\cite{Bellm:2015jjp} and the \MMHT[lo] PDF set~\cite{Harland-Lang:2014zoa}.
The decays of bottom and charm hadrons
were simulated using the \EVTGEN[1.6.0] program~\cite{Lange:2001uf}. 


\paragraph{Long description:}

To assess the uncertainty in the matching of NLO matrix elements to the parton shower, 
a \POWHEG sample was compared with a sample generated by \MGNLO~\cite{Alwall:2014hca}. 
The first sample was produced with the same hard-scatter set-up as the nominal sample using the 
\POWHEGBOX[v2]~\cite{Frixione:2007nw,Nason:2004rx,Frixione:2007vw,Alioli:2010xd}
generator, which provided matrix elements at next-to-leading
order~(NLO) in the strong coupling constant \alphas, with the
\NNPDF[3.0nlo]~\cite{Ball:2014uwa} parton distribution function~(PDF) and
the \hdamp parameter set to 1.5\,\mtop~\cite{ATL-PHYS-PUB-2016-020}.
The functional form of the renormalisation and factorisation scales was
set to the default scale $\sqrt{\mtop^{2} + \pTX[2]}$.  
The second sample used \MGNLO[2.6.0] with the 
\NNPDF[3.0nlo]~\cite{Ball:2014uwa} PDF set for the calculation of the hard-scattering.
Top quarks were decayed at LO using
\MADSPIN~\cite{Frixione:2007zp,Artoisenet:2012st} to preserve spin
correlations. The parton-shower starting
scale had the functional form $\muQ = \HT/2$~\cite{ATL-PHYS-PUB-2017-007}, 
where \HT is defined
as the scalar sum of the \pT of all outgoing partons.
The events from both generators were interfaced with
\HERWIG[7.13]~\cite{Bahr:2008pv,Bellm:2015jjp}, using the \HERWIG[7.1] default set
of tuned parameters~\cite{Bellm:2015jjp} and the \MMHT[lo] PDF set
\cite{Harland-Lang:2014zoa}.
The renormalisation and factorisation scale choice in the \MGNLO set-up was the same as for the
\POWHEGBOX set-up. The decays of bottom and charm hadrons were simulated
using the \EVTGEN[1.6.0] program~\cite{Lange:2001uf} in both set-ups.


\subsection[Sherpa 2.2.1]{\SHERPA[2.2.1]}
\label{subsubsec:ttbar_sherpa}

\paragraph{Samples}
%\label{par:ttbar_sherpa_samples}

The descriptions below correspond to the samples in \cref{tab:ttbar_sherpa}.
\begin{table}[htbp]
  \caption{\ttbar samples produced with \SHERPA[2.2.1].}%
  \label{tab:ttbar_sherpa}
  \centering
  \begin{tabular}{l l}
  \toprule
  DSID range & Description \\
  \midrule
  410249 & \ttbar all-hadronic  \\
  410250 & \ttbar single lepton  \\
  410251 & \ttbar single lepton  \\
  410252 & \ttbar dileptonic  \\
  \bottomrule
  \end{tabular}
\end{table}

\paragraph{Short description:}

Additional samples of \ttbar events were produced with the
\SHERPA[2.2.1]~\cite{Bothmann:2019yzt} generator using NLO-accurate
matrix elements for up to one additional parton, and LO-accurate
matrix elements for up to four additional partons calculated with the
Comix~\cite{Gleisberg:2008fv} and
\OPENLOOPS~\cite{Buccioni:2019sur,Cascioli:2011va,Denner:2016kdg} libraries. They were
matched with the \SHERPA parton shower~\cite{Schumann:2007mg} using
the \MEPSatNLO
prescription~\cite{Hoeche:2011fd,Hoeche:2012yf,Catani:2001cc,Hoeche:2009rj}
and the set of tuned parameters developed by the \SHERPA authors
to match the \NNPDF[3.0nnlo] set of PDFs~\cite{Ball:2014uwa}.

\textbf{Additional information}: The central scale had the functional
form $\mu^{2} = \mtop^{2} + 0.5\times(\pTX[2][t]p + \pTX[2][\bar{t}])$.
The CKKW matching scale of the
additional emissions was set to \SI{30}{\GeV}.

% !TEX root = MC_snippets.tex

%%%%%%%%%%%%%%%%%%%%%%%%%%%%%%%%%%%%%%%%%%%
%%%              tW                     %%%
%%%%%%%%%%%%%%%%%%%%%%%%%%%%%%%%%%%%%%%%%%%
\section{Single-top \texorpdfstring{\(tW\)}{tW} associated production}
\label{subsec:tW}

This section describes the MC samples used for the modelling of single-top \(tW\) associated production.
\Cref{subsubsec:tW_PP8} describes the \POWPY[8] samples -- both for the diagram removal (DR) set-ups,
which are used for the nominal prediction as well as uncertainties due to additional radiation and PDFs,
and for the diagram subtraction (DS) set-ups, which are used for the uncertainty due to the treatment
of the overlap with \ttbar production.
\Cref{subsubsec:tW_PH7} describes the \POWHER[7] samples used for the uncertainty
due to parton showering and hadronisation modelling, and \cref{subsubsec:tW_aMCP8} describes the
\MGNLOPY[8] samples used for the uncertainty due to the choice of matching scheme.

The reference cross-section values are extracted from Ref.~\cite{LHCTopWGsgtopXsec}.

\subsection[Powheg+Pythia8]{\POWPY[8]}
\label{subsubsec:tW_PP8}

\paragraph{Samples}
%\label{par:tW_PP8_samples}

\Cref{tab:tW_PP8} gives the DSIDs of the \(tW\) \POWPY[8] samples, for both the DR and DS schemes.
Single-top and single-anti-top (\(tW^-\) and \(\bar{t}W^+\)) events were generated in different samples.
The dileptonic samples overlap with the inclusive ones.

\begin{table}[htbp]
  \caption{Single-top \(tW\) associated production samples produced with \POWPY[8].}%
  \label{tab:tW_PP8}
  \centering
  \begin{tabular}{l l}
    \toprule
    DSID & Description \\
    \midrule
    410646 & \(tW^-\) (DR) inclusive \\
    410647 & \(\bar{t}W^+\) (DR) inclusive \\
    410648 & \(tW^-\) (DR) dileptonic \\
    410649 & \(\bar{t}W^+\) (DR) dileptonic \\
    \midrule
    410654 & \(tW^-\) (DS) inclusive \\
    410655 & \(\bar{t}W^+\) (DS) inclusive \\
    410656 & \(tW^-\) (DS) dileptonic \\
    410657 & \(\bar{t}W^+\) (DS) dileptonic \\
    \bottomrule
  \end{tabular}
\end{table}

\paragraph{Short description:}

The associated production of top quarks with \(W\) bosons (\(tW\)) was
modelled by the
\POWHEGBOX[v2]~\cite{Re:2010bp,Nason:2004rx,Frixione:2007vw,Alioli:2010xd}
generator at NLO in QCD using the five-flavour scheme and the
\NNPDF[3.0nlo] set of PDFs~\cite{Ball:2014uwa}.
The diagram removal scheme~\cite{Frixione:2008yi} was used to
remove interference and overlap with \ttbar production.
The related uncertainty was estimated by comparison with an alternative sample
generated using the diagram subtraction scheme~\cite{Frixione:2008yi,ATL-PHYS-PUB-2016-020}.\footnote{Analyses which do not use this approach
should obviously not use this sentence in their description.}
The events were interfaced to \PYTHIA[8.230]~\cite{Sjostrand:2014zea} using the A14
tune~\cite{ATL-PHYS-PUB-2014-021} and the \NNPDF[2.3lo] set of
PDFs~\cite{Ball:2012cx}.

The uncertainty due to initial-state radiation (ISR) was estimated by
simultaneously varying the \hdamp parameter and the \muR and
\muF scales, and choosing the Var3c up/down variants of the A14 tune
as described in Ref.~\cite{ATL-PHYS-PUB-2017-007}. The impact of
final-state radiation (FSR) was evaluated by varying the renormalisation scale 
for emissions from the parton shower up or down by a factor two.



\paragraph{Long description:}

Single-top \(tW\) associated production was modelled using the
\POWHEGBOX[v2]~\cite{Re:2010bp,Nason:2004rx,Frixione:2007vw,Alioli:2010xd}
generator, which provided matrix elements at next-to-leading
order~(NLO) in the strong coupling constant \alphas\ in the five-flavour
scheme with the \NNPDF[3.0nlo]~\cite{Ball:2014uwa} parton
distribution function~(PDF) set.  The functional form of the
renormalisation and factorisation scales was set to the default scale,
which is equal to the top-quark mass (\(\mtop = \SI{172.5}{\GeV}\)).
The diagram removal scheme~\cite{Frixione:2008yi} was employed to handle the interference
with \ttbar production~\cite{ATL-PHYS-PUB-2016-020}.  The events were
interfaced with \PYTHIA[8.230]~\cite{Sjostrand:2014zea} using the A14
tune~\cite{ATL-PHYS-PUB-2014-021} and the \NNPDF[2.3lo] PDF set.  The
decays of bottom and charm hadrons were simulated using the
\EVTGEN[1.6.0] program~\cite{Lange:2001uf}.

The inclusive cross-section was corrected to the theory prediction
calculated at NLO in QCD with NNLL soft-gluon
corrections~\cite{Kidonakis:2010ux,Kidonakis:2013zqa}.  For proton--proton
collisions at a centre-of-mass energy of \(\rts = \SI{13}{\TeV}\), this
cross-section corresponds to \(\sigma(tW)_{\text{NLO+NNLL}} = 71.7 \pm 3.8\,\si{\pb}\),
using a top-quark mass of \(\mtop = \SI{172.5}{\GeV}\).  The uncertainty in
the cross-section due to the PDF was calculated using the \MSTW[nnlo] 90\%
CL~\cite{Martin:2009iq,Martin:2009bu} PDF set, and was added in
quadrature to the effect of the scale uncertainty.

The uncertainty due to initial-state radiation (ISR) was estimated by
comparing the nominal \ttbar sample with two additional
samples~\cite{ATL-PHYS-PUB-2017-007}.  To simulate higher parton
radiation, the factorisation and renormalisation scales were reduced by
a factor of 0.5 while simultaneously increasing the \hdamp value to
3.0\,\mtop and using the Var3c up variation from the A14 tune. For
lower parton radiation, \muR and \muF were increased by a factor of two
while keeping the \hdamp value set to 1.5\,\mtop and using the Var3c down
variation in the parton shower.  The Var3c A14 tune
variation~\cite{ATL-PHYS-PUB-2014-021} largely corresponds to the variation of
\alphas for ISR in the A14 tune. 
The impact of final-state radiation (FSR) was evaluated by
varying the renormalisation scale for emissions from the
parton shower up and down by a factor of two.


The nominal \POWPY[8] sample was compared with an alternative
sample generated using the diagram subtraction
scheme~\cite{Frixione:2008yi,ATL-PHYS-PUB-2016-020} to estimate the
uncertainty arising from the interference with \ttbar production.

To evaluate the PDF uncertainties for the nominal PDF, the 100
variations for \NNPDF[3.0nlo] were taken into account.
In addition, the central value of this PDF was compared with the
central values of the \CT[14nnlo]~\cite{Dulat:2015mca} and
\MMHT[nnlo]~\cite{Harland-Lang:2014zoa} PDF sets.


\subsection[Powheg+Herwig7]{\POWHER[7]}
\label{subsubsec:tW_PH7}

\paragraph{Samples}
%\label{par:tW_PH7_samples}

\Cref{tab:tW_PH7} gives the DSIDs of the \(tW\) \POWHER[7] DR samples.
Single-top and single-anti-top (\(tW^-\) and \(\bar{t}W^+\)) events were generated in different samples.
The dileptonic samples overlap with the inclusive ones.

\begin{table}[htbp]
  \caption{Single-top \(tW\) associated production samples produced with \POWHER[7].}%
  \label{tab:tW_PH7}
  \centering
  \begin{tabular}{l l}
    \toprule
    DSID & Description \\
    \midrule
    411036 & \(tW^-\) (DR) inclusive \\
    411037 & \(\bar{t}W^+\) (DR) inclusive \\
    411038 & \(tW^-\) (DR) dileptonic \\
    411039 & \(\bar{t}W^+\) (DR) dileptonic \\
    \bottomrule
  \end{tabular}
\end{table}

\paragraph{Short description:}

The uncertainty due to the parton shower and hadronisation model was
evaluated by comparing the nominal sample of events with a sample where
events generated with the
\POWHEGBOX[v2]~\cite{Re:2010bp,Nason:2004rx,Frixione:2007vw,Alioli:2010xd}
generator were interfaced to
\HERWIG[7.04]~\cite{Bahr:2008pv,Bellm:2015jjp}, using the H7UE set
of tuned parameters~\cite{Bellm:2015jjp} and the \MMHT[lo] PDF set~\cite{Harland-Lang:2014zoa}.


\paragraph{Long description:}

The impact of using a different parton shower and hadronisation model was evaluated
by comparing the nominal \(tW\) sample with another sample produced with the
\POWHEGBOX[v2]~\cite{Re:2010bp,Nason:2004rx,Frixione:2007vw,Alioli:2010xd}
generator but interfaced with \HERWIG[7.04]~\cite{Bahr:2008pv,Bellm:2015jjp},
using the H7UE set of tuned parameters~\cite{Bellm:2015jjp} and the
\MMHT[lo] PDF set \cite{Harland-Lang:2014zoa}.
\POWHEGBOX provided matrix elements at next-to-leading order~(NLO) in the
strong coupling constant \alphas in the five-flavour scheme with the
\NNPDF[3.0nlo]~\cite{Ball:2014uwa} parton distribution function~(PDF).
The functional form of the renormalisation and factorisation scales was set to
the default scale, which is equal to the top-quark mass.  The diagram removal
scheme~\cite{Frixione:2008yi} was employed to handle the interference
with \ttbar production~\cite{ATL-PHYS-PUB-2016-020}.  The decays of bottom
and charm hadrons are simulated using the \EVTGEN[1.6.0]
program~\cite{Lange:2001uf}.


\subsection[MadGraph5\_aMC@NLO+Pythia8]{\MGNLOPY[8]}
\label{subsubsec:tW_aMCP8}

\paragraph{Samples}
%\label{par:tW_aMCP8_samples}

\Cref{tab:tW_aMCP8} gives the DSIDs of the \(tW\) \MGNLOPY[8] samples.
The dileptonic sample overlaps with the inclusive one.

\begin{table}[htbp]
  \caption{Single-top \(tW\) associated production samples produced with \MGNLOPY[8].}%
  \label{tab:tW_aMCP8}
  \centering
  \begin{tabular}{l l}
    \toprule
    DSID & Description \\
    \midrule
    412002 & \(tW\) inclusive \\
    412003 & \(tW\) dileptonic \\
    \bottomrule
  \end{tabular}
\end{table}

\paragraph{Short description:}

To assess the uncertainty in the matching of NLO matrix elements to the
parton shower, the nominal \(tW\) sample was compared with a sample generated
with the \MGNLO[2.6.2]~\cite{Alwall:2014hca} generator at NLO in QCD using the five-flavour
scheme and the \NNPDF[2.3nlo]~\cite{Ball:2014uwa} PDF set. The events were
interfaced with \PYTHIA[8.230]~\cite{Sjostrand:2014zea}, using the A14
set of tuned parameters~\cite{ATL-PHYS-PUB-2014-021} and the \NNPDF[2.3lo]
PDF.


\paragraph{Long description:}

To assess the uncertainty due to the choice of matching scheme, the nominal \(tW\) sample was compared with a sample generated
with the \MGNLO[2.6.2]~\cite{Alwall:2014hca} generator, which provided matrix elements at next-to-leading order~(NLO) in the strong coupling constant \alphas
in the five-flavour scheme, using the \NNPDF[2.3nlo]~\cite{Ball:2014uwa} PDF set.
The functional form of the renormalisation and factorisation scale was set to the default scale, which is equal to the top-quark mass.
The parton-shower starting scale had the functional form \(\muQ = \HT/2\)~\cite{ATL-PHYS-PUB-2017-007},
where \HT is defined as the scalar sum of the \pT of all outgoing partons.
The diagram removal scheme~\cite{Frixione:2008yi} was employed to handle the interference with \ttbar production~\cite{ATL-PHYS-PUB-2016-020}.
The events were interfaced with \PYTHIA[8.230]~\cite{Sjostrand:2014zea}, using the A14 set of tuned parameters~\cite{ATL-PHYS-PUB-2014-021}
and the \NNPDF[2.3lo] PDF.
The decays of bottom and charm hadrons were simulated using the \EVTGEN[1.6.0] program~\cite{Lange:2001uf}.


%%%%%%%%%%%%%%%%%%%%%%%%%%%%%%%%%%%%%%%%%%%
%%%              t-channel              %%%
%%%%%%%%%%%%%%%%%%%%%%%%%%%%%%%%%%%%%%%%%%%
\section{Single-top \texorpdfstring{\(t\)}{t}-channel production}
\label{subsec:tchan}

This section describes the MC samples used for the modelling of single-top \(t\)-channel production.
\Cref{subsubsec:tchan_PP8} describes the \POWPY[8] samples used for the nominal prediction
and for the uncertainty from additional radiation and due to PDFs.
\Cref{subsubsec:tchan_PH7} describes the \POWHER[7] samples used for the uncertainty due to the choice of parton shower and hadronisation model,
and \cref{subsubsec:tchan_aMCP8} describes the \MGNLOPY[8] samples used for the uncertainty due to the choice of matching scheme.

The reference cross-section values are extracted from Ref.~\cite{LHCTopWGsgtopXsec}.


\subsection[Powheg+Pythia8]{\POWPY[8]}
\label{subsubsec:tchan_PP8}

\paragraph{Samples}
%\label{par:tchan_PP8_samples}

\Cref{tab:tchan_PP8} gives the DSIDs of the \(t\)-channel \POWPY[8] samples.
Single-top and single-anti-top events were generated in distinct samples.

\begin{table}[!htbp]
  \caption{Single-top \(t\)-channel event samples produced with \POWPY[8].}%
  \label{tab:tchan_PP8}
  \centering
  \begin{tabular}{l l}
    \toprule
    DSID & Description \\
    \midrule
    410658 & \(t\)-channel \(t\) leptonic \\
    410659 & \(t\)-channel \(\bar{t}\) leptonic \\
    \bottomrule
  \end{tabular}
\end{table}

\paragraph{Short description:}

Single-top \(t\)-channel production was modelled using the
\POWHEGBOX[v2]~\cite{Frederix:2012dh,Nason:2004rx,Frixione:2007vw,Alioli:2010xd}
generator at NLO in QCD using the four-flavour scheme and the
corresponding \NNPDF[3.0nlo] set of PDFs~\cite{Ball:2014uwa}.  The events were
interfaced with \PYTHIA[8.230]~\cite{Sjostrand:2014zea} using the A14
tune~\cite{ATL-PHYS-PUB-2014-021} and the \NNPDF[2.3lo] set of
PDFs~\cite{Ball:2012cx}.

The uncertainty due to initial-state radiation (ISR) was estimated by
simultaneously varying the \hdamp parameter and the \muR and
\muF scales, and choosing the Var3c up/down variants of the A14 tune
as described in Ref.~\cite{ATL-PHYS-PUB-2017-007}. The impact of
final-state radiation (FSR) was evaluated by varying the renormalisation scale 
for emissions from the parton shower up or down by a factor two.



\paragraph{Long description:}

Single-top \(t\)-channel production was modelled using the
\POWHEGBOX[v2]~\cite{Frederix:2012dh,Nason:2004rx,Frixione:2007vw,Alioli:2010xd}
generator, which provided matrix elements at next-to-leading-order~(NLO)
accuracy in the strong coupling constant \alphas\ in the four-flavour
scheme with the corresponding \NNPDF[3.0nlo]~\cite{Ball:2014uwa} parton
distribution function~(PDF) set.  The functional form of the
renormalisation and factorisation scales was set to
\(\sqrt{m_b^2 + \pTX[2][b]}\) following the
recommendation of Ref.~\cite{Frederix:2012dh}. Top quarks were decayed at
LO using \MADSPIN~\cite{Frixione:2007zp,Artoisenet:2012st} to preserve
all spin correlations.  The events were interfaced with
\PYTHIA[8.230]~\cite{Sjostrand:2014zea} using the A14
tune~\cite{ATL-PHYS-PUB-2014-021} and the \NNPDF[2.3lo] PDF set.
The decays of bottom and charm hadrons were simulated using the
\EVTGEN[1.6.0] program~\cite{Lange:2001uf}.

The inclusive cross-section was corrected to the theory prediction calculated at NLO in QCD with
\HATHOR[2.1]~\cite{Aliev:2010zk,Kant:2014oha}.
For proton--proton collisions at a centre-of-mass energy of \(\rts = \SI{13}{\TeV}\), this cross-section corresponds to
\(\sigma(t,t\text{-chan})_\text{NLO}= 136.02^{+5.40}_{-4.57}\)\,pb (\(\sigma(\bar{t},t\text{-chan})_\text{NLO}=80.95^{+4.06}_{-3.61}\)\,pb)
for single-top (single-anti-top) production, using a top-quark mass of \(\mtop = \SI{172.5}{\GeV}\).
The uncertainties in the cross-section due to the PDF and \alphas were calculated using the \PDFforLHC prescription~\cite{Butterworth:2015oua}
with the \MSTW[nlo] 68\% CL~\cite{Martin:2009iq,Martin:2009bu}, \CT[10nlo]~\cite{Lai:2010vv}
and \NNPDF[2.3nlo]~\cite{Ball:2012cx} PDF sets,
and were added in quadrature to the effect of the scale uncertainty.

The uncertainty due to initial-state radiation (ISR) was estimated by
comparing the nominal \ttbar sample with two additional
samples~\cite{ATL-PHYS-PUB-2017-007}.  To simulate higher parton
radiation, the factorisation and renormalisation scales were reduced by
a factor of 0.5 while simultaneously increasing the \hdamp value to
3.0\,\mtop and using the Var3c up variation from the A14 tune. For
lower parton radiation, \muR and \muF were increased by a factor of two
while keeping the \hdamp value set to 1.5\,\mtop and using the Var3c down
variation in the parton shower.  The Var3c A14 tune
variation~\cite{ATL-PHYS-PUB-2014-021} largely corresponds to the variation of
\alphas for ISR in the A14 tune. 
The impact of final-state radiation (FSR) was evaluated by
varying the renormalisation scale for emissions from the
parton shower up and down by a factor of two.


To evaluate the PDF uncertainties for the nominal PDF, the 100 variations for \NNPDF[3.0nlo] were taken into account.
In addition, the central value of this PDF was compared with the central values of the \CT[14nnlo]~\cite{Dulat:2015mca}
and \MMHT[nnlo]~\cite{Harland-Lang:2014zoa} PDF sets.


\subsection[Powheg+Herwig7]{\POWHER[7]}
\label{subsubsec:tchan_PH7}

\paragraph{Samples}
%\label{par:tchan_PH7_samples}

\Cref{tab:tchan_PH7} gives the DSIDs of the \(t\)-channel \POWHER[7] samples.
Single-top and single-anti-top events were generated in distinct samples.

\begin{table}[htbp]
  \caption{Single-top \(t\)-channel event samples produced with \POWHER[7].}%
  \label{tab:tchan_PH7}
  \centering
  \begin{tabular}{l l}
    \toprule
    DSID & Description \\
    \midrule
    411032 & \(t\)-channel \(\bar t\) leptonic \\
    411033 & \(t\)-channel \(t\) leptonic \\
    \bottomrule
  \end{tabular}
\end{table}

\paragraph{Short description:}

The uncertainty due to the parton shower and hadronisation model was
evaluated by comparing the nominal sample of events with a sample where
the events generated with the
\POWHEGBOX[v2]~\cite{Frederix:2012dh,Nason:2004rx,Frixione:2007vw,Alioli:2010xd}
generator were interfaced to
\HERWIG[7.04]~\cite{Bahr:2008pv,Bellm:2015jjp}, using the H7UE set
of tuned parameters~\cite{Bellm:2015jjp} and the \MMHT[lo] PDF set
\cite{Harland-Lang:2014zoa}.


\paragraph{Long description:}

The impact of using a different parton shower and hadronisation model was evaluated by comparing the nominal sample
with another sample produced with the \POWHEGBOX[v2]~\cite{Frederix:2012dh,Nason:2004rx,Frixione:2007vw,Alioli:2010xd}
generator but interfaced with \HERWIG[7.04]~\cite{Bahr:2008pv,Bellm:2015jjp}, using the H7UE set of
tuned parameters~\cite{Bellm:2015jjp} and the \MMHT[lo] PDF set \cite{Harland-Lang:2014zoa}.
\POWHEGBOX provided matrix elements at next-to-leading order~(NLO) in the strong coupling constant \alphas
in the four-flavour scheme with the corresponding \NNPDF[3.0nlo]~\cite{Ball:2014uwa} parton distribution function~(PDF).
The functional form of the renormalisation and factorisation scales was set to \(\sqrt{m_b^2 + \pTX[2][b]}\)
following the recommendation of Ref.~\cite{Frederix:2012dh}.
Top quarks were decayed at LO using \MADSPIN~\cite{Frixione:2007zp,Artoisenet:2012st} to preserve all spin correlations.
The decays of bottom and charm hadrons were simulated using the \EVTGEN[1.6.0] program~\cite{Lange:2001uf}.


\subsection[MadGraph5\_aMC@NLO+Pythia8]{\MGNLOPY[8]}
\label{subsubsec:tchan_aMCP8}

\paragraph{Samples}
%\label{par:tchan_aMCP8_samples}

\Cref{tab:tchan_aMCP8} gives the DSIDs of the \(t\)-channel \MGNLOPY[8] samples.

\begin{table}[htbp]
  \caption{Single-top \(t\)-channel event samples produced with \MGNLOPY[8].}%
  \label{tab:tchan_aMCP8}
  \centering
  \begin{tabular}{l l}
    \toprule
    DSID & Description \\
    \midrule
    412004 & \(t\)-channel leptonic \\
    \bottomrule
  \end{tabular}
\end{table}

\paragraph{Short description:}

To assess the uncertainty in the matching of NLO matrix elements to the
parton shower, the nominal sample was compared with a sample generated
with the \MGNLO[2.6.2]~\cite{Alwall:2014hca} generator at NLO in QCD using the five-flavour
scheme and the \NNPDF[2.3nlo]~\cite{Ball:2014uwa} PDF set. The events were
interfaced with \PYTHIA[8.230]~\cite{Sjostrand:2014zea}, using the A14
set of tuned parameters~\cite{ATL-PHYS-PUB-2014-021} and the \NNPDF[2.3lo] PDF set.


\paragraph{Long description:}

To assess the uncertainty due to the choice of matching scheme, the nominal sample was compared with a sample generated
with the \MGNLO[2.6.2]~\cite{Alwall:2014hca} generator, which provided matrix elements at next-to-leading order~(NLO) in the strong coupling constant \alphas
in the four-flavour scheme, using the corresponding \NNPDF[3.0nlo]~\cite{Ball:2014uwa} PDF set.
The functional form of the renormalisation and factorisation scales was set to \(\sqrt{m_b^2 + \pTX[2][b]}\)
following the recommendation of Ref.~\cite{Frederix:2012dh}.
The parton-shower starting scale had the functional form \(\muQ = \HT/2\)~\cite{ATL-PHYS-PUB-2017-007},
where \HT is defined as the scalar sum of the \pT of all outgoing partons.
Top quarks were decayed at LO using \MADSPIN~\cite{Frixione:2007zp,Artoisenet:2012st} to preserve all spin correlations.
The events were interfaced with \PYTHIA[8.230]~\cite{Sjostrand:2014zea}, using the A14 set of tuned parameters~\cite{ATL-PHYS-PUB-2014-021}
and the \NNPDF[2.3lo] PDF set.
The decays of bottom and charm hadrons were simulated using the \EVTGEN[1.6.0] program~\cite{Lange:2001uf}.


%%%%%%%%%%%%%%%%%%%%%%%%%%%%%%%%%%%%%%%%%%%
%%%              s-channel              %%%
%%%%%%%%%%%%%%%%%%%%%%%%%%%%%%%%%%%%%%%%%%%
\subsection{Single-top \texorpdfstring{\(s\)}{s}-channel production}
\label{subsec:schan}

This section describes the MC samples used for the modelling of single-top \(s\)-channel production.
\Cref{subsubsec:schan_PP8} describes the \POWPY[8] samples used for the nominal prediction
and for the uncertainty from additional radiation and due to PDFs.
\Cref{subsubsec:schan_PH7} describes the \POWHER[7] samples used for the uncertainty due to the parton shower and hadronisation model,
and \cref{subsubsec:schan_aMCP8} describes the \MGNLOPY[8] samples used for the uncertainty due to the choice of matching scheme.

The reference cross-section values are extracted from Ref.~\cite{LHCTopWGsgtopXsec}.

\subsection[Powheg+Pythia8]{\POWPY[8]}
\label{subsubsec:schan_PP8}

\paragraph{Samples}
%\label{par:schan_PP8_samples}

\Cref{tab:schan_PP8} gives the DSIDs of the \(s\)-channel \POWPY[8] samples.
Single-top and single-anti-top events were generated in distinct samples.

\begin{table}[htbp]
  \caption{Single-top \(s\)-channel event samples produced with \POWPY[8].}%
  \label{tab:schan_PP8}
  \centering
  \begin{tabular}{l l}
    \toprule
    DSID & Description \\
    \midrule
    410644 & \(s\)-channel \(t\) leptonic \\
    410645 & \(s\)-channel \(\bar{t}\) leptonic \\
    \bottomrule
  \end{tabular}
\end{table}

\paragraph{Short description:}

Single-top \(s\)-channel production was modelled using the \POWHEGBOX[v2]~\cite{Alioli:2009je,Nason:2004rx,Frixione:2007vw,Alioli:2010xd}
generator at NLO in QCD in the five-flavour scheme with the \NNPDF[3.0nlo]~\cite{Ball:2014uwa} parton distribution function~(PDF) set.
The events were interfaced with \PYTHIA[8.230]~\cite{Sjostrand:2014zea} using the A14 tune~\cite{ATL-PHYS-PUB-2014-021} and the
\NNPDF[2.3lo] PDF set.

The uncertainty due to initial-state radiation (ISR) was estimated by
simultaneously varying the \hdamp parameter and the \muR and
\muF scales, and choosing the Var3c up/down variants of the A14 tune
as described in Ref.~\cite{ATL-PHYS-PUB-2017-007}. The impact of
final-state radiation (FSR) was evaluated by varying the renormalisation scale 
for emissions from the parton shower up or down by a factor two.



\paragraph{Long description:}

Single-top \(s\)-channel production was modelled using the \POWHEGBOX[v2]~\cite{Alioli:2009je,Nason:2004rx,Frixione:2007vw,Alioli:2010xd}
generator, which provided matrix elements at next-to-leading order~(NLO) in the strong coupling constant \alphas in the
five-flavour scheme with the \NNPDF[3.0nlo]~\cite{Ball:2014uwa} parton distribution function~(PDF) set.
The functional form of the renormalisation and factorisation scales was set to the default scale, which was equal to the top-quark mass.
The events were interfaced with \PYTHIA[8.230]~\cite{Sjostrand:2014zea} using the A14 tune~\cite{ATL-PHYS-PUB-2014-021} and the
\NNPDF[2.3lo] PDF set.
The decays of bottom and charm hadrons were simulated using the \EVTGEN[1.6.0] program~\cite{Lange:2001uf}.

The inclusive cross-section was corrected to the theory prediction calculated at NLO in QCD with
\HATHOR[2.1]~\cite{Aliev:2010zk,Kant:2014oha}.
For proton--proton collisions at a centre-of-mass energy of \(\rts = \SI{13}{\TeV}\), this cross-section corresponds to
\(\sigma(t,s\text{-chan})_{\text{NLO}} = 6.35^{+0.23}_{-0.20}\,\si{\pb}\)
(\(\sigma(\bar{t},s\text{-chan})_{\text{NLO}} = 3.97^{+0.19}_{-0.17}\,\si{\pb}\))
for single-top (single-anti-top) production, using a top-quark mass of \(\mtop = \SI{172.5}{\GeV}\).
The uncertainties in the cross-section due to the PDF and \alphas were calculated using the \PDFforLHC prescription~\cite{Butterworth:2015oua}
with the \MSTW[nlo] 68\% CL~\cite{Martin:2009iq,Martin:2009bu}, \CT[10nlo]~\cite{Lai:2010vv} and \NNPDF[2.3nlo]~\cite{Ball:2012cx} PDF sets,
and were added in quadrature to the effect of the scale uncertainty.

The uncertainty due to initial-state radiation (ISR) was estimated by
comparing the nominal \ttbar sample with two additional
samples~\cite{ATL-PHYS-PUB-2017-007}.  To simulate higher parton
radiation, the factorisation and renormalisation scales were reduced by
a factor of 0.5 while simultaneously increasing the \hdamp value to
3.0\,\mtop and using the Var3c up variation from the A14 tune. For
lower parton radiation, \muR and \muF were increased by a factor of two
while keeping the \hdamp value set to 1.5\,\mtop and using the Var3c down
variation in the parton shower.  The Var3c A14 tune
variation~\cite{ATL-PHYS-PUB-2014-021} largely corresponds to the variation of
\alphas for ISR in the A14 tune. 
The impact of final-state radiation (FSR) was evaluated by
varying the renormalisation scale for emissions from the
parton shower up and down by a factor of two.


To evaluate the PDF uncertainties for the nominal PDF, the 100 variations for \NNPDF[3.0nlo] were taken into account.
In addition, the central value of this PDF was compared with the central values of the
\CT[14nnlo]~\cite{Dulat:2015mca} and \MMHT[nnlo]~\cite{Harland-Lang:2014zoa} PDF sets.


\subsection[Powheg+Herwig7]{\POWHER[7]}
\label{subsubsec:schan_PH7}

\paragraph{Samples}
%\label{par:schan_PH7_samples}

\Cref{tab:schan_PH7} gives the DSIDs of the \(s\)-channel \POWHER[7] samples.
Single-top and single-anti-top events were generated in distinct samples.

\begin{table}[htbp]
  \caption{Single-top \(s\)-channel event samples produced with \POWHER[7].}%
  \label{tab:schan_PH7}
  \centering
  \begin{tabular}{l l}
    \toprule
    DSID & Description \\
    \midrule
    411034 & \(s\)-channel \(t\) leptonic \\
    411035 & \(s\)-channel \(\bar{t}\) leptonic \\
    \bottomrule
  \end{tabular}
\end{table}

\paragraph{Short description:}

The impact of using a different parton shower and hadronisation model was evaluated by comparing the nominal sample
with another sample produced with the \POWHEGBOX[v2]~\cite{Alioli:2009je,Nason:2004rx,Frixione:2007vw,Alioli:2010xd}
generator at NLO in the strong coupling constant \alphas in the five-flavour scheme using the
\NNPDF[3.0nlo]~\cite{Ball:2014uwa} parton distribution function~(PDF).
Events in the latter sample were interfaced with \HERWIG[7.04]~\cite{Bahr:2008pv,Bellm:2015jjp}, using the H7UE set of
tuned parameters~\cite{Bellm:2015jjp} and the \MMHT[lo] PDF set \cite{Harland-Lang:2014zoa}.


\paragraph{Long description:}

The impact of using a different parton shower and hadronisation model was evaluated by comparing the nominal sample
with another sample produced with the \POWHEGBOX[v2]~\cite{Alioli:2009je,Nason:2004rx,Frixione:2007vw,Alioli:2010xd}
generator but interfaced with \HERWIG[7.04]~\cite{Bahr:2008pv,Bellm:2015jjp}, using the H7UE set of
tuned parameters~\cite{Bellm:2015jjp} and the \MMHT[lo] PDF set \cite{Harland-Lang:2014zoa}.
\POWHEGBOX provided matrix elements at next-to-leading order~(NLO) in the strong coupling constant \alphas
in the five-flavour scheme with the \NNPDF[3.0nlo]~\cite{Ball:2014uwa} parton distribution function~(PDF).
The functional form of the renormalisation and factorisation scales was set to the default scale, which is equal to the top-quark mass.
The decays of bottom and charm hadrons were simulated using the \EVTGEN[1.6.0] program~\cite{Lange:2001uf}.


\subsection[MadGraph5\_aMC@NLO+Pythia8]{\MGNLOPY[8]}
\label{subsubsec:schan_aMCP8}

\paragraph{Samples}
%\label{par:schan_aMCP8_samples}

\Cref{tab:schan_aMCP8} gives the DSIDs of the \(s\)-channel \MGNLOPY[8] samples.

\begin{table}[htbp]
  \caption{Single-top \(s\)-channel event samples produced with \MGNLOPY[8].}%
  \label{tab:schan_aMCP8}
  \centering
  \begin{tabular}{l l}
    \toprule
    DSID & Description \\
    \midrule
    412005 & \(s\)-channel leptonic \\
    \bottomrule
  \end{tabular}
\end{table}

\paragraph{Short description:}

%% the MG5_aMC@NLO version was double checked in the tag collector
To assess the uncertainty due to the choice of matching scheme, the nominal sample was compared with a sample generated
with the \MGNLO[2.6.2]~\cite{Alwall:2014hca} generator at NLO in the strong coupling constant \alphas in the five-flavour scheme,
using the \NNPDF[3.0nlo]~\cite{Ball:2014uwa} PDF set.
The events were interfaced with \PYTHIA[8.230]~\cite{Sjostrand:2014zea}, using the A14 set of tuned parameters~\cite{ATL-PHYS-PUB-2014-021}
and the \NNPDF[2.3lo] PDF set.


\paragraph{Long description:}

To assess the uncertainty due to the choice of matching scheme, the nominal sample was compared with a sample generated
with the \MGNLO[2.6.2]~\cite{Alwall:2014hca} generator, which provided matrix elements at next-to-leading order~(NLO) in the strong coupling constant \alphas
in the five-flavour scheme with the \NNPDF[3.0nlo]~\cite{Ball:2014uwa} parton distribution function~(PDF).
The functional form of the renormalisation and factorisation scales was set to the default scale, which is equal to the top-quark mass.
The parton-shower starting scale had the functional form \(\muQ = \HT/2\)~\cite{ATL-PHYS-PUB-2017-007},
where \HT is defined as the scalar sum of the \pT of all outgoing partons.
Top quarks were decayed at LO using \MADSPIN~\cite{Frixione:2007zp,Artoisenet:2012st} to preserve all spin correlations.
The events were interfaced with \PYTHIA[8.230]~\cite{Sjostrand:2014zea}, using the A14 set of tuned parameters~\cite{ATL-PHYS-PUB-2014-021}
and the \NNPDF[2.3lo] PDF set.
The decays of bottom and charm hadrons were simulated using the \EVTGEN[1.6.0] program~\cite{Lange:2001uf}.

\subsection[\ttHF]{\ttHF}
\label{subsec:ttHF}

In the following subsections, the set-ups of the current baseline samples for the production of \ttbar quark pairs in association with
$b$-quarks (\ttHF) are described. NLO predictions with massive $b$-quarks in the matrix element and matched to parton shower
programs are available within the \SHERPAOL, \MGNLO and, more recently, \POWHEGBOX frameworks.

\subsubsection[Sherpa]{\SHERPA}
%\label{subsubsec:ttHF_sherpa}
The descriptions below refer to the \SHERPA[2.2.1] samples. 
Details of the set-up are given in Ref.~\cite{ATL-PHYS-PUB-2016-016} and reported below. 

\paragraph{Samples} 
%\label{par:ttHF_sherpa_samples}

The descriptions below correspond to the samples in Table~\ref{tab:ttHF_Sh}.

\begin{table}[htbp]
  \caption{Nominal \ttbar+HF samples produced with \SHERPA.
      Variation samples are not explicitly listed.}%
  \label{tab:ttHF_Sh}
  \centering
  \begin{tabular}{l l}
    \toprule
    DSID range & Description \\
    \midrule
    410323--4 & \ttbar single lepton \\
    410325 & \ttbar dilepton \\
    410369 & \ttbar all-hadronic \\
    \bottomrule
  \end{tabular}
\end{table}

\paragraph{Description:}

Samples for \ttHF processes were produced with the \SHERPA[2.2.1]~\cite{Bothmann:2019yzt} generator,
using the MEPS@NLO prescription~\cite{Hoeche:2012yf} and interfaced with \OPENLOOPS~\cite{Buccioni:2019sur,Cascioli:2011va,Denner:2016kdg}
to provide the virtual corrections for matrix elements at NLO accuracy.
The four-flavour scheme is used with the $b$-quark mass set to 4.75\,\GeV.  
The renormalisation scale \muR has the functional form 
$\sqrt[4]{m_\text{T}(t) \cdot m_\text{T}(\bar{t}) \cdot m_\text{T}(b) \cdot m_\text{T}(\bar{b})}$. The 
factorisation scale \muF was set to $H_\text{T}/2$, where $H_\text{T}$ is the transverse-mass sum of the
partons in the matrix element, and this value was also the resummation scale \muQ of the parton shower.
The \CT[10nlo] PDF set was used in conjunction with a dedicated PS tune developed by the \SHERPA authors. 


\subsubsection[MadGraph5\_aMC@NLO+Pythia8]{\MGNLOPY[8]}
%\label{subsubsec:ttHF_aMCP8}
In the following, set-ups are described for \PYTHIA only. 
Details of the set-up are given in Ref.~\cite{ATL-PHYS-PUB-2016-016} and reported below. 

\paragraph{Samples}
%\label{par:ttHF_aMCP8_samples}

The descriptions below correspond to the samples in Table~\ref{tab:ttHF_amc}.

\begin{table}[htbp]
  \caption{Nominal \ttbar+HF samples produced with \MGNLOPY[8].}%
  \label{tab:ttHF_amc}
  \centering
  \begin{tabular}{l l}
    \toprule
    DSID range & Description \\
    \midrule
    410265 & \ttbar non-all-hadronic \\
    410266 & \ttbar dileptonic \\
    410267 & \ttbar all-hadronic \\
    \bottomrule
  \end{tabular}
\end{table}

\paragraph{Description:}

Samples for \ttHF processes were produced with the \MGNLO generator 
with the \NNPDF[3.0nlo]~\cite{Ball:2014uwa} PDF set. It was interfaced with \PYTHIA[8.230]~\cite{Sjostrand:2014zea},
using the A14 set of tuned parameters~\cite{ATL-PHYS-PUB-2014-021} and the \NNPDF[2.3lo] PDF.
The four-flavour scheme was used with the $b$-quark mass set to 4.75\,\GeV.  
The renormalisation scale \muR has the functional form 
$\sqrt[4]{m_\text{T}(t) \cdot m_\text{T}(\bar{t}) \cdot m_\text{T}(b) \cdot m_\text{T}(\bar{b})}$. The 
factorisation scale \muF was set to $H_\text{T}/2$, where $H_\text{T}$ is the transverse-mass sum of the partons in the matrix
element.
The resummation scale \muQ has the form $\muQ = f_\text{Q} \sqrt{\hat{s}}$, where 
the prefactor $f_\text{Q}$ is an external parameter randomly distributed in the 
range $[f^\text{min}_\text{Q}$, $f^\text{max}_\text{Q}]=[0.1,0.25]$.    


\subsubsection[PowhegBoxRes+Pythia8]{\POWHEGBOXRES+\PYTHIA[8]}
%\label{subsubsec:ttHF_PP8}
In the following, set-ups are described for \PYTHIA[8] only. 

\paragraph{Samples}
%\label{par:ttHF_PP8_samples}
The descriptions below correspond to the samples in Table~\ref{tab:ttHF_pp8}.

\begin{table}[htbp]
  \caption{Nominal \ttbar+HF samples produced with \POWHEGBOXRES+\PYTHIA[8].}%
  \label{tab:ttHF_pp8}
  \centering
  \begin{tabular}{l l}
    \toprule
    DSID range & Description \\
    \midrule
    411179--80 & \ttbar non-all-hadronic \\
    411178    & \ttbar dileptonic \\
    411275    & \ttbar all-hadronic \\
    \bottomrule
  \end{tabular}
\end{table}

\paragraph{Description:}

Samples for \ttHF processes were produced with the \POWHEGBOXRES~\cite{Jezo:2018yaf}
generator and \OPENLOOPS~\cite{Buccioni:2019sur,Cascioli:2011va,Denner:2016kdg}, using a pre-release 
of the implementation of this process in \POWHEGBOXRES provided by the authors~\cite{ttbbPowheg}, 
with the \NNPDF[3.0nlo]~\cite{Ball:2014uwa} PDF set. It was interfaced with \PYTHIA[8.240]~\cite{Sjostrand:2014zea},
using the A14 set of tuned parameters~\cite{ATL-PHYS-PUB-2014-021} and the \NNPDF[2.3lo] PDF set.
The four-flavour scheme was used with the $b$-quark mass set to 4.95\,\GeV.
The factorisation scale was set to $0.5\times\Sigma_{i=t,\bar{t},b,\bar{b},j}m_{\mathrm{T},i}$,
the renormalisation scale was set to $\sqrt[4]{m_{\text{T}}(t)\cdot m_{\text{T}}(\bar{t})\cdot m_{\text{T}}(b)\cdot m_{\text{T}}(\bar{b})}$,
and the \hdamp parameter was set to $0.5\times\Sigma_{i=t,\bar{t},b,\bar{b}}m_{\mathrm{T},i}$.


%-------------------------------------------------------------------------------
% RareTop
\section{Rare top-quark processes}
%\label{sec:RT}

This section describes the samples used for rare top-quark processes.
Section~\ref{subsec:ttH} describes the \ttH\ samples.
Section~\ref{subsec:ttV} describes the \ttV\ ($V=W/Z$) samples.
Section~\ref{subsec:ttgamma} describes the \ttgamma\ samples.
Section~\ref{subsec:tHq} describes the \tH\ samples.
Section~\ref{subsec:tZq} describes the \tZq\ samples.
Section~\ref{subsec:tWZ} describes the \tWZ\ samples.
%\tgamma\ samples are described in Section~\ref{subsec:tgamma}.
Finally, Section~\ref{subsec:tttt} describes the \tttt\ samples.

%%%%%%%%%%%%%%%%%%%%%%%%%%%%%%%%%%%%%%%%%%%
%%%              ttH                    %%%
%%%%%%%%%%%%%%%%%%%%%%%%%%%%%%%%%%%%%%%%%%%
\subsection[\ttH production]{\ttH}
\label{subsec:ttH}

\subsubsection[Powheg+Pythia8]{\POWPY[8]}
%\label{subsubsec:ttH_PP8}

Nominal $t\bar{t}H$ samples are produced with \POWPY[8].
The \hdamp value is set to $\SI{352.5}{\GeV} = 3/4 \cdot (\mH+2\mtop)$.

\paragraph{Samples}
%\label{par:ttH_PP8_samples}
Table~\ref{tab:ttH_PP8} gives the nominal \ttH samples.

\begin{table}[htbp]
  \caption{Nominal \ttH samples produced with \POWPY[8].}%
  \label{tab:ttH_PP8}
  \centering
  \begin{tabular}{l l}
    \toprule
    DSID range & Description \\
    \midrule
    346343 & \ttH, $H\to$ all, $\ttbar\to$ all-hadronic \\
    346344 & \ttH, $H\to$ all, $\ttbar\to$ semileptonic \\
    346345 & \ttH, $H\to$ all, $\ttbar\to$ dileptonic \\
    346525 & \ttH, $H\to \gamma\gamma$, $\ttbar\to$ all \\
    \bottomrule
  \end{tabular}
\end{table}

\paragraph{Short description:}

The production of \ttH events was modelled using the
\POWHEGBOX[v2]~\cite{Frixione:2007nw,Nason:2004rx,Frixione:2007vw,Alioli:2010xd,Hartanto:2015uka}
generator at NLO with the \NNPDF[3.0nlo]~\cite{Ball:2014uwa} PDF set.
The events were interfaced to \PYTHIA[8.230]~\cite{Sjostrand:2014zea}~using 
the A14 tune~\cite{ATL-PHYS-PUB-2014-021} and the
\NNPDF[2.3lo]~\cite{Ball:2014uwa} PDF set. The decays of bottom and charm hadrons
were performed by \EVTGEN[1.6.0]~\cite{Lange:2001uf}.

\paragraph{Long description:}

The production of \ttH events was modelled using the
\POWHEGBOX[v2]~\cite{Frixione:2007nw,Nason:2004rx,Frixione:2007vw,Alioli:2010xd,Hartanto:2015uka}
generator, which provided matrix elements at next-to-leading order (NLO) in the strong coupling 
constant \alphas in the five-flavour scheme with the \NNPDF[3.0nlo]~\cite{Ball:2014uwa} PDF set.
The functional form of the renormalisation and factorisation scales was
set to $\sqrt[3]{m_\text{T}(t)\cdot m_\text{T}(\bar{t}) \cdot m_\text{T}(H)}$.
The events were interfaced to \PYTHIA[8.230]~\cite{Sjostrand:2014zea}
using the A14 tune~\cite{ATL-PHYS-PUB-2014-021} and the
\NNPDF[2.3lo]~\cite{Ball:2014uwa} PDF set. The decays of bottom and charm hadrons
were performed by \EVTGEN[1.6.0]~\cite{Lange:2001uf}.

The cross-section was calculated at NLO QCD and NLO EW accuracy using
\MGNLO as reported in Ref.~\cite{deFlorian:2016spz}.
The predicted value at $\rts = \SI{13}{\TeV}$ is
$507^{+35}_{-50}\,\si{\fb}$, where the uncertainties were estimated from
variations of \alphas and the renormalisation and factorisation scales.

The uncertainty in the initial-state radiation (ISR) was estimated using the Var3c
up/down variations of the A14 tune. Uncertainties due to missing
higher-order corrections were evaluated through simultaneous variations of the
renormalisation and factorisation scales by factors of
2.0 and 0.5. Uncertainties in the PDFs were evaluated using the 100
variations of the \NNPDF[3.0nlo] set.


\subsubsection[Powheg+Herwig7]{\POWHER[7]}
%\label{subsubsec:ttH_PH7}

\paragraph{Samples}
%\label{par:ttH_PH7_samples}

Table~\ref{tab:ttH_PH7} presents alternative \ttH samples.

\begin{table}[htbp]
  \caption{Alternative \ttH \POWHER[7] samples produced to evaluate systematic uncertainties 
  due to different MC models for parton showering and hadronisation.}%
  \label{tab:ttH_PH7}
  \centering
  \begin{tabular}{l l}
    \toprule
    DSID range & Description \\
    \midrule
    346346 & \ttH, $H\to$ all, $\ttbar\to$ all-hadronic \\
    346347 & \ttH, $H\to$ all, $\ttbar\to$ semileptonic \\
    346348 & \ttH, $H\to$ all, $\ttbar\to$ dileptonic \\
    346526 & \ttH, $H\to\gamma\gamma$, $\ttbar\to$ all \\
    \bottomrule
  \end{tabular}
\end{table}

\paragraph{Short description:}

The impact of using a different parton shower and hadronisation model was evaluated by showering the nominal hard-scatter events with
\HERWIG[7.04]~\cite{Bahr:2008pv,Bellm:2015jjp} using the H7UE set of tuned parameters~\cite{Bellm:2015jjp} and 
the \MMHT[lo] PDF set~\cite{Harland-Lang:2014zoa}.

\paragraph{Long description:}

The impact of using a different parton shower and hadronisation model was evaluated
by comparing the nominal sample with another sample produced with
the \POWHEGBOX[v2]~\cite{Frixione:2007nw,Nason:2004rx,Frixione:2007vw,Alioli:2010xd}
generator but interfaced with \HERWIG[7.04]~\cite{Bahr:2008pv,Bellm:2015jjp}, using the 
H7UE set of tuned parameters~\cite{Bellm:2015jjp} and 
the \MMHT[lo] PDF set~\cite{Harland-Lang:2014zoa}.
\POWHEGBOX provided matrix elements at next-to-leading
order~(NLO) in the strong coupling constant \alphas with the
\NNPDF[3.0nlo]~\cite{Ball:2014uwa} parton distribution function~(PDF). 
The functional form of the renormalisation and factorisation scales was 
set to $\sqrt[3]{m_\text{T}(t)\cdot m_\text{T}(\bar{t}) \cdot m_\text{T}(H)}$.
The decays of bottom and charm hadrons
were simulated using the \EVTGEN[1.6.0] program~\cite{Lange:2001uf}. 

%%%%%%%%%%%%%%%%%%%%%%%%%%%%%%%%%%%%%%%%%%%
%%%              ttV                    %%%
%%%%%%%%%%%%%%%%%%%%%%%%%%%%%%%%%%%%%%%%%%%
\section[\ttV production]{\ttV production}
\label{subsec:ttV}

This section describes the MC samples used for the modelling of \ttV ($V=W/Z$) production.
\Cref{subsubsec:ttV_aMCP8} describes the \MGNLOPY[8] samples,
and \cref{subsubsec:ttV_sherpa} describes the \SHERPA samples.
(NOTE: this section is not frozen as the \SHERPA samples are likely to be updated and become the nominal samples in the near future.)

\subsection[MadGraph5\_aMC@NLO+Pythia8]{\MGNLOPY[8]}
\label{subsubsec:ttV_aMCP8}

\paragraph{Samples}
%\label{par:ttV_aMCP8_samples}

The descriptions below correspond to the samples in \cref{tab:ttV_aMCP8,tab:ttV_aMCP8_addRad}.

\begin{table}[htbp]
  \caption{Nominal \ttV samples produced with \MGNLOPY[8].}%
  \label{tab:ttV_aMCP8}
  \centering
  \begin{tabular}{l l}
    \toprule
    DSID range & Description \\
    \midrule
    410155 & \ttW \\
    410156 & \ttZnunu \\
    410157 & \ttZqq \\
    410218 & \ttee, $m_{\ell\ell} > \SI{5}{\GeV}$ \\
    410219 & \ttmumu, $m_{\ell\ell} > \SI{5}{\GeV}$ \\
    410220 & \tttautau, $m_{\ell\ell} > \SI{5}{\GeV}$ \\
    410276 & \ttee, $m_{\ell\ell} \in \SI[parse-numbers=false]{[1,5]}{\GeV}$\\
    410277 & \ttmumu, $m_{\ell\ell} \in \SI[parse-numbers=false]{[1,5]}{\GeV}$\\
    410278 & \tttautau, $m_{\ell\ell} \in \SI[parse-numbers=false]{[1,5]}{\GeV}$\\
    \bottomrule
  \end{tabular}
\end{table}

\begin{table}[htbp]
  \caption{\ttV samples produced with \MGNLOPY[8] used to estimate initial-state radiation systematic uncertainties.}%
  \label{tab:ttV_aMCP8_addRad}
  \centering
  \begin{tabular}{l l}
    \toprule
    DSID range & Description \\
    \midrule
    410376 & \ttW\ A14Var3c up \\
    410377 & \ttW\ A14Var3c down \\
    410378 & \ttZnunu\ A14Var3c up \\
    410379 & \ttZnunu\ A14Var3c down \\
    410380 & \ttZqq\ A14Var3c up \\
    410381 & \ttZqq\ A14Var3c down \\
    410370 & \ttee\ A14Var3c up \\
    410371 & \ttee\ A14Var3c down \\
    410372 & \ttmumu\ A14Var3c up \\
    410373 & \ttmumu\ A14Var3c down \\
    410374 & \tttautau\ A14Var3c up \\
    410375 & \tttautau\ A14Var3c down \\
    \bottomrule
  \end{tabular}
\end{table}

\paragraph{Short description:}

The production of \ttV\ events was modelled using the
\MGNLO[2.3.3]~\cite{Alwall:2014hca} generator at NLO with the
\NNPDF[3.0nlo]~\cite{Ball:2014uwa} parton distribution function~(PDF).
The events were interfaced to \PYTHIA[8.210]~\cite{Sjostrand:2014zea}
using the A14 tune~\cite{ATL-PHYS-PUB-2014-021} and the
\NNPDF[2.3lo]~\cite{Ball:2014uwa} PDF set. The decays of bottom and charm
hadrons were simulated using the \EVTGEN[1.2.0] program~\cite{Lange:2001uf}.

The uncertainty due to initial-state radiation (ISR) was estimated by
comparing the nominal event sample with two samples where the Var3c
up/down variations of the A14 tune were employed.


\paragraph{Long description:}

The production of \ttV events was modelled using the \MGNLO[2.3.3]
\cite{Alwall:2014hca} generator, which provided matrix elements at
next-to-leading order~(NLO) in the strong coupling constant \alphas
with the \NNPDF[3.0nlo]~\cite{Ball:2014uwa} parton distribution
function~(PDF). The functional form of the renormalisation and
factorisation scales was set to the default of $0.5 \times \sum_i
\sqrt{m^2_i+p^2_{\text{T},i}}$, where the sum runs over all the particles
generated from the matrix element calculation. Top quarks were decayed
at LO using \MADSPIN~\cite{Frixione:2007zp,Artoisenet:2012st} to
preserve spin correlations. The events were interfaced with
\PYTHIA[8.210]~\cite{Sjostrand:2014zea} for the parton shower and
hadronisation, using the A14 set of tuned
parameters~\cite{ATL-PHYS-PUB-2014-021} and the
\NNPDF[2.3lo]~\cite{Ball:2014uwa} PDF set.  
The decays of bottom and charm hadrons were simulated using the \EVTGEN[1.2.0] program~\cite{Lange:2001uf}. 

% Note : The cross-section is supplemented with an off-shell (down to \SI{5}{\GeV}) correction calculated in Ref.~\cite{ATLASPMGttVNLOXsec}. 
The cross-sections were calculated at NLO QCD and NLO EW accuracy using
\MGNLO as reported in Ref.~\cite{deFlorian:2016spz}.
In the case of \ttll the cross-section was scaled by 
an off-shell correction estimated at one-loop level in \alphas.
(\emph{Optionally:})
The predicted values at $\rts = \SI{13}{\TeV}$ are $0.88^{+0.09}_{-0.11}\,\si{\pb}$ and
$0.60^{+0.08}_{-0.07}\,\si{\pb}$ for \ttZ and \ttW, respectively, where the
uncertainties were estimated from variations of \alphas and the renormalisation and
factorisation scales.

The uncertainty due to initial-state radiation (ISR) was estimated by
comparing the nominal \ttV sample with two additional samples, which
have the same settings as the nominal one, but employed the Var3c up or down
variation of the A14 tune, which corresponds
to the variation of \alphas for initial-state radiation
(ISR) in the A14 tune.

Uncertainties due to missing higher-order corrections were evaluated
by simultaneously varying the renormalisation and factorisation scales
by factors of 2.0 and 0.5. Uncertainties in the PDFs
were evaluated using the 100 replicas of the \NNPDF[3.0nlo] set.


\subsection[Sherpa]{\SHERPA}
\label{subsubsec:ttV_sherpa}

\paragraph{Samples}
%\label{par:ttV_sherpa_samples}

The descriptions below correspond to the samples in \cref{tab:ttV_sherpa}.

\begin{table}[htbp]
  \caption{\ttV samples produced with \SHERPA.}%
  \label{tab:ttV_sherpa}
  \centering
  \begin{tabular}{l l}
    \toprule
    DSID range & Description \\
    \midrule
    410142 & \ttll \\
    410143 & \ttZqq, \ttZnunu \\
    410144 & \ttW \\
    \bottomrule
  \end{tabular}
\end{table}

\paragraph{Description:}

Additional \ttV samples were produced with the
\SHERPA[2.2.0]~\cite{Bothmann:2019yzt} generator at LO accuracy, using
the \MEPSatLO set-up~\cite{Catani:2001cc,Hoeche:2009rj} with up to one
additional parton for the \ttll sample and two additional partons for the
others. A dynamic renormalisation scale was used and is defined
similarly to that of the nominal \ttV samples. The CKKW matching
scale of the additional emissions was set to \SI{30}{\GeV}. The default
\SHERPA[2.2.0] parton shower was used along with the
\NNPDF[3.0nnlo]~\cite{Ball:2014uwa} PDF set.

%%%%%%%%%%%%%%%%%%%%%%%%%%%%%%%%%%%%%%%%%%%
%%%              ttgamma                    %%%
%%%%%%%%%%%%%%%%%%%%%%%%%%%%%%%%%%%%%%%%%%%
\section[\ttgamma production]{\ttgamma production}
\label{subsec:ttgamma}

This section describes the MC samples used for the modelling of \ttgamma\ production.
\Cref{subsubsec:ttgamma_aMCP8} describes the \MGNLOPY[8] samples,
and \cref{subsubsec:ttgamma_aMCH7} describes the \MGNLOHER[7] samples.

\subsection[MadGraph5\_aMC@NLO+Pythia8]{\MGNLOPY[8]}
\label{subsubsec:ttgamma_aMCP8}

\paragraph{Samples}
%\label{par:ttgamma_aMCP8_samples}

The descriptions below correspond to the samples in \cref{tab:ttgamma_aMCP8,tab:ttgamma_aMCP8_addRad}.

\begin{table}[htbp]
  \caption{Nominal \ttgamma\ samples produced with \MGNLOPY[8].}%
  \label{tab:ttgamma_aMCP8}
  \centering
  \begin{tabular}{l l}
    \toprule
    DSID range & Description \\
    \midrule
    410389 & \ttgamma, non-all-hadronic \\
    410394 & \ttgamma, all-hadronic \\
    \bottomrule
  \end{tabular}
\end{table}

\begin{table}[htbp]
  \caption{\ttgamma\ samples produced with \MGNLOPY[8] used to estimate initial-state radiation systematic uncertainties.}%
  \label{tab:ttgamma_aMCP8_addRad}
  \centering
  \begin{tabular}{l l}
    \toprule
    DSID range & Description \\
    \midrule
    410404 & \ttgamma\, non-all-hadronic, A14Var3c up \\
    410405 & \ttgamma\, non-all-hadronic, A14Var3c down \\
    410410 & \ttgamma\, all-hadronic, A14Var3c up \\
    410411 & \ttgamma\, all-hadronic, A14Var3c down \\
    \bottomrule
  \end{tabular}
\end{table}

\paragraph{Short description:}

The production of \ttgamma events was modelled using the \MGNLO[2.3.3]~\cite{Alwall:2014hca}
generator at LO with the \NNPDF[2.3lo]~\cite{Ball:2014uwa} parton distribution function~(PDF).
The events were interfaced with \PYTHIA[8.212]~\cite{Sjostrand:2014zea} using the A14 tune~\cite{ATL-PHYS-PUB-2014-021} and the
\NNPDF[2.3lo]~\cite{Ball:2014uwa} PDF set.
The decays of bottom and charm hadrons were simulated using the \EVTGEN[1.6.0] program~\cite{Lange:2001uf}.

The uncertainty due to initial-state radiation (ISR) was estimated by
comparing the nominal \ttgamma sample with two additional samples,
where the Var3c up/down variations of the A14 tune were employed.


\paragraph{Long description:}

The \ttgamma sample was simulated as a 2\(\to\)7 process at LO including the decay of the top quarks by
\MGNLO[2.3.3]~\cite{Alwall:2014hca} with the \NNPDF[2.3lo]~\cite{Ball:2014uwa} parton distribution function~(PDF), interfaced with
\PYTHIA[8.212]~\cite{Sjostrand:2014zea}, using the A14 set of tuned parameters~\cite{ATL-PHYS-PUB-2014-021} and the
\NNPDF[2.3lo]~\cite{Ball:2014uwa} PDF set. The photon could be radiated from an initial charged parton, an intermediate top quark,
or any of the charged final-state particles. The top-quark mass, top-quark decay width, \(W\)-boson decay width,
and fine structure constant were set to \SI{172.5}{\GeV}, \SI{1.320}{\GeV}, \SI{2.085}{\GeV}, and 1/137, respectively.
The five-flavour scheme was used, where all the quark masses are set to zero, except for the top quark. The renormalisation and the
factorisation scales were set to \(0.5\times \sum_i \sqrt{m^2_i+p^2_{\text{T},i}}\), where the sum runs over all the particles generated
from the matrix element calculation.
The decays of bottom and charm hadrons were simulated using the \EVTGEN[1.6.0] program~\cite{Lange:2001uf}.

The cross-section was calculated at NLO in QCD as reported in Ref.~\cite{Melnikov:2011ta}, resulting in a \(K\)-factor of 1.24 which was applied
to the samples, with a relative uncertainty of 14\% from variations of renormalisation and factorisation scales as well
as the choice of PDF set.

The uncertainty due to initial-state radiation (ISR) was estimated by comparing the nominal \ttV\ sample with two additional samples,
 which had the same settings as the nominal one, but employed the Var3c up or down variation of the A14 tune, which
corresponds to the variation of \alphas for initial-state radiation (ISR) in the A14 tune.

To evaluate the effect of renormalisation and factorisation scale uncertainties, the two scales were varied simultaneously by factors 2.0 and 0.5.
To evaluate the PDF uncertainties for the nominal PDF, the 100 replicas for \NNPDF[2.3lo] were taken into account.


\subsection[MadGraph5\_aMC@NLO+Herwig7]{\MGNLOHER[7]}
\label{subsubsec:ttgamma_aMCH7}

\paragraph{Samples}
%\label{par:ttgamma_aMCH7_samples}

The descriptions below correspond to the samples in \cref{tab:ttgamma_aMCH7}.

\begin{table}[htbp]
  \caption{\ttgamma\ samples produced with \MGNLOHER[7].}%
  \label{tab:ttgamma_aMCH7}
  \centering
  \begin{tabular}{l l}
    \toprule
    DSID range & Description \\
    \midrule
    410395 & \ttgamma non-all-hadronic \\
    410396 & \ttgamma all-hadronic \\
    \bottomrule
  \end{tabular}
\end{table}

\paragraph{Short description:}

Additional \ttgamma samples were produced with the parton shower of the nominal samples replaced by
\HERWIG[7.04]~\cite{Bahr:2008pv,Bellm:2015jjp} to evaluate the impact of using using a different parton shower and hadronisation model.
The H7UE set of tuned parameters~\cite{Bellm:2015jjp} and the \MMHT[lo] PDF set~\cite{Harland-Lang:2014zoa} were used.

%%%%%%%%%%%%%%%%%%%%%%%%%%%%%%%%%%%%%%%%%%%
%%%              tH                    %%%
%%%%%%%%%%%%%%%%%%%%%%%%%%%%%%%%%%%%%%%%%%%
\subsection[\tHq]{\tHq}
\label{subsec:tHq}

This section describes the MC samples used for the modelling of \tH\ production.
Section~\ref{subsubsec:tHq_aMCP8} describes the \MGNLOPY[8] samples,

\subsubsection[MadGraph5\_aMC@NLO+Pythia8]{\MGNLOPY[8]}
\label{subsubsec:tHq_aMCP8}

\paragraph{Samples}
%\label{par:tHq_aMCP8_samples}

The descriptions below correspond to the samples in Table~\ref{tab:tHq_aMCP8}.

\begin{table}[htbp]
\begin{center}
\caption{Nominal \tH\ samples produced with \MGNLOPY[8].} 
\label{tab:tHq_aMCP8}
\begin{tabular}{ l | l }
\hline
DSID range & Description \\
\hline
\hline
  346188 & \tHq\, \Hgg\, four flavour \\
  346229 & \tHq\, \Hbb\, four flavour \\
  346230 & \tHq\, \Htautau/\HZZ/\HWW\, four flavour \\
  346414 & \tHq\, \Hllll, four flavour \\
  \hline
  346676 & \tHq\, $H\rightarrow$ inclusive, four flavour, UFO model \\
  346677 & \tHq\, \Hgg, four flavour, UFO model \\
  346799 & \tHq\, \Htautau/\HZZ/\HWW\ + Nleptons=2 filter, four flavour, UFO model \\
  \hline
  \hline
\end{tabular}
\end{center}
\end{table}

\paragraph{Exceptions:}
If and only if you are using the UFO model sample: the correct version is
\MGNLO[2.6.2].

\paragraph{Short description:}

The production of \tHq events was modelled using the \MGNLO[2.6.0]~\cite{Alwall:2014hca}
generator at NLO with the \NNPDF[3.0nlo]~\cite{Ball:2014uwa} parton distribution function~(PDF).
The events were interfaced with \PYTHIA[8.230]~\cite{Sjostrand:2014zea} using the A14 tune~\cite{ATL-PHYS-PUB-2014-021} and the
\NNPDF[2.3lo]~\cite{Ball:2014uwa} PDF set.
The decays of bottom and charm hadrons were simulated using the \EVTGEN[1.6.0] program~\cite{Lange:2001uf}. 

\paragraph{Long description:}

The \tHq samples were simulated using the \MGNLO[2.6.0]~\cite{Alwall:2014hca}
generator at NLO with the \NNPDF[3.0nlo]~\cite{Ball:2014uwa} parton distribution function~(PDF). The events were interfaced with
\PYTHIA[8.230]~\cite{Sjostrand:2014zea}~ using the A14 tune~\cite{ATL-PHYS-PUB-2014-021} and the \NNPDF[2.3lo]~\cite{Ball:2014uwa} PDF set. 
The top quark was decayed at LO using \MADSPIN~\cite{Frixione:2007zp,Artoisenet:2012st} to preserve spin correlations,
whereas the Higgs boson was decayed by \PYTHIA in the parton shower. The samples
were generated in the four-flavour scheme.
The functional form of the renormalisation and factorisation scales was set to the 
default scale $0.5\times \sum_i \sqrt{m^2_i+p^2_{\text{T},i}}$, where the sum runs over 
all the particles generated from the matrix element calculation.
The decays of bottom and charm hadrons were simulated using the \EVTGEN[1.6.0] program~\cite{Lange:2001uf}.

%%%%%%%%%%%%%%%%%%%%%%%%%%%%%%%%%%%%%%%%%%%
%%%              tHW                    %%%
%%%%%%%%%%%%%%%%%%%%%%%%%%%%%%%%%%%%%%%%%%%
\section[\tHW]{\tHW}
\label{subsec:tHW}

This section describes the MC samples used for the modelling of \tHW\ production.
\Cref{subsubsec:tHW_aMCP8} describes the \MGNLOPY[8] samples.

\subsection[MadGraph5\_aMC@NLO+Pythia8]{\MGNLOPY[8]}
\label{subsubsec:tHW_aMCP8}

\paragraph{Samples}

The descriptions below correspond to the samples in \cref{tab:tHW_aMCP8}.

\begin{table}
  \caption{Nominal \tHW\ samples produced with \MGNLOPY[8].}%
  \label{tab:tHW_aMCP8}
  \centering
  \begin{tabular}{l l}
    \toprule
    DSID & Description \\
    \midrule
    346486 & \tHW\, \Hgg\ \\
    346511 & \tHW\, \Hllll \\
    346678 & \tHW\, \(H\rightarrow\) inclusive, UFO model \\
    346759 & \tHW\, \Hgg{}, UFO model \\
    \bottomrule
  \end{tabular}
\end{table}

\paragraph{Short description:}

The \tHW{} production is modelled using the \MGNLO[2.6.2]~\cite{Alwall:2014hca} generator at NLO with the \NNPDF[3.0nlo]~\cite{Ball:2014uwa} parton distribution function~(PDF). The overlap with the \ttH{} production is removed using the diagram removal scheme~\cite{Frixione:2008yi,Demartin:2016axk}. The events are interfaced with \PYTHIA[8.235]~\cite{Sjostrand:2014zea} using the A14 parameter set~\cite{ATL-PHYS-PUB-2014-021} and the \NNPDF[2.3lo]~\cite{Ball:2014uwa} PDF set.
The decays of bottom and charm hadrons are simulated using the \EVTGEN[1.6.0] program~\cite{Lange:2001uf}.


\paragraph{Long description:}

The \tHW{} samples were simulated using the \MGNLO[2.6.2]~\cite{Alwall:2014hca}
generator at NLO with the \NNPDF[3.0nlo]~\cite{Ball:2014uwa} parton distribution function~(PDF). The events were interfaced with
\PYTHIA[8.235]~\cite{Sjostrand:2014zea}~ using the A14 tune~\cite{ATL-PHYS-PUB-2014-021} and the \NNPDF[2.3lo]~\cite{Ball:2014uwa} PDF set.
The top quark was decayed at LO using \MADSPIN~\cite{Frixione:2007zp,Artoisenet:2012st} to preserve spin correlations,
whereas the Higgs boson was decayed by \PYTHIA in the parton shower. The samples
were generated in the five-flavour scheme.
The functional form of the renormalisation and factorisation scales was set to the
default scale \(0.5\times \sum_i \sqrt{m^2_i+p^2_{\text{T},i}}\), where the sum runs over
all the particles generated from the matrix element calculation.
The decays of bottom and charm hadrons were simulated using the \EVTGEN[1.6.0] program~\cite{Lange:2001uf}.

%%%%%%%%%%%%%%%%%%%%%%%%%%%%%%%%%%%%%%%%%%%
%%%              tZq                    %%%
%%%%%%%%%%%%%%%%%%%%%%%%%%%%%%%%%%%%%%%%%%%
\subsection[\tZq]{\tZq}
\label{subsec:tZq}

This section describes the MC samples used for the modelling of \tZq production.
Section~\ref{subsubsec:tZq_aMCP8} describes the \MGNLOPY[8] samples,

\subsubsection[MadGraph5\_aMC@NLO+Pythia8]{\MGNLOPY[8]}
\label{subsubsec:tZq_aMCP8}

\paragraph{Samples}
%\label{par:tZq_aMCP8_samples}

The descriptions below correspond to the samples in Tables~\ref{tab:tZq_aMCP8} and \ref{tab:tZq_aMCP8_addRad}.

\begin{table}[htbp]
\begin{center}
\caption{Nominal \tZq samples produced with \MGNLOPY[8].} 
\label{tab:tZq_aMCP8}
\begin{tabular}{ l | l }
\hline
DSID range & Description \\
\hline
412063 & \tZq \\
\hline
\end{tabular}
\end{center}
\end{table}

\begin{table}[htbp]
\begin{center}
\caption{\tZq samples produced with \MGNLOPY[8] used to estimate initial-state radiation systematic uncertainties.}
\label{tab:tZq_aMCP8_addRad}
\begin{tabular}{ l | l }
\hline
DSID range & Description \\
\hline
412065 & \tZq, A14Var3c up \\
410064 & \tZq, A14Var3c down \\
\hline
\end{tabular}
\end{center}
\end{table}


\paragraph{Short description:}

The production of \tZq events was modelled using the \MGNLO[2.3.3]~\cite{Alwall:2014hca}
generator at NLO with the \NNPDF[3.0nlo]~\cite{Ball:2014uwa} parton distribution function~(PDF).
The events were interfaced with \PYTHIA[8.230]~\cite{Sjostrand:2014zea} using the A14 tune~\cite{ATL-PHYS-PUB-2014-021} and the
\NNPDF[2.3lo]~\cite{Ball:2014uwa} PDF set.

The uncertainty due to initial-state radiation (ISR) was estimated by comparing the nominal \tZq sample with two additional samples,
which had the same settings as the nominal one, but employed the Var3c up and down variations of the A14 tune. 

\paragraph{Long description:}

The \tZq sample was simulated using the \MGNLO[2.3.3]~\cite{Alwall:2014hca}
generator at NLO with the \NNPDF[3.0nlo]~\cite{Ball:2014uwa} parton distribution function~(PDF). The events were interfaced with
\PYTHIA[8.230]~\cite{Sjostrand:2014zea} using the A14 tune~\cite{ATL-PHYS-PUB-2014-021} 
and the \NNPDF[2.3lo]~\cite{Ball:2014uwa} PDF set. Off-resonance events away from the $Z$ mass peak were included. 
The top quark was decayed at LO using \MADSPIN~\cite{Frixione:2007zp,Artoisenet:2012st} to preserve spin correlations.
The four-flavour scheme was used, where all the quark masses are set to zero, except for the top and bottom quarks. 
Following the discussion in Ref.~\cite{Frederix:2012dh}, the functional form of the renormalisation and factorisation scales 
was set to 4$\sqrt{m_b^2+p_{\text{T},b}^2}$, where the $b$-quark was the one produced by a gluon splitting in the event. 
The decays of bottom and charm hadrons were simulated using the \EVTGEN program~\cite{Lange:2001uf}.

The \tZq total cross-section, calculated at next-to-leading order (NLO) using \MGNLO[2.3.3] with the \NNPDF[3.0nlo] PDF set, 
is 800\,fb, with an uncertainty of $^{+6.1}_{-7.4}$\%. The uncertainty was computed by varying the renormalisation and 
factorisation scales by a factor of two and by a factor of 0.5.

The uncertainty due to initial-state radiation (ISR) was estimated by comparing the nominal \tZq sample with two additional samples,
which have the same settings as the nominal one, but employed the Var3c up or down variation of the A14 tune, which
corresponds to the variation of \alphas for ISR in the A14 tune.

To evaluate the effect of renormalisation and factorisation scale uncertainties, the two scales were varied simultaneously by factors 2.0 and 0.5.
To evaluate the PDF uncertainties for the nominal PDF, the 100 variations for \NNPDF[2.3lo] were taken into account. 

%%%%%%%%%%%%%%%%%%%%%%%%%%%%%%%%%%%%%%%%%%%
%%%              tWZ                    %%%
%%%%%%%%%%%%%%%%%%%%%%%%%%%%%%%%%%%%%%%%%%%
\subsection[\tWZ]{\tWZ}
\label{subsec:tWZ}

This section describes the MC samples used for the modelling of \tWZ\ production.
Section~\ref{subsubsec:tWZ_aMCP8} describes the \MGNLOPY[8] samples,

\subsubsection[MadGraph5\_aMC@NLO+Pythia8]{\MGNLOPY[8]}
\label{subsubsec:tWZ_aMCP8}

\paragraph{Samples}
%\label{par:tWZ_aMCP8_samples}

The descriptions below correspond to the samples in Table~\ref{tab:tWZ_aMCP8}.

\begin{table}[htbp]
  \caption{Nominal \tWZ\ samples produced with \MGNLOPY[8].}%
  \label{tab:tWZ_aMCP8}
  \centering
  \begin{tabular}{l l}
    \toprule
    DSID range & Description \\
    \midrule
    410408 & \tWZ\, DR1 \\
    410409 & \tWZ\, DR2 \\
    \bottomrule
  \end{tabular}
\end{table}

\paragraph{Short description:}

The production of \tWZ\ events was modelled using the \MGNLO[2.3.3]~\cite{Alwall:2014hca}
generator at NLO with the \NNPDF[3.0nlo]~\cite{Ball:2014uwa} parton distribution function~(PDF).
The events were interfaced with \PYTHIA[8.212]~\cite{Sjostrand:2014zea}~ using the A14 tune~\cite{ATL-PHYS-PUB-2014-021} and the
\NNPDF[2.3lo]~\cite{Ball:2014uwa} PDF set.
The decays of bottom and charm hadrons were simulated using the \EVTGEN[1.2.0] program~\cite{Lange:2001uf}. 


\paragraph{Long description:}

The production of \tWZ events was modelled using the \MGNLO[2.3.3]~\cite{Alwall:2014hca}
generator at NLO with the \NNPDF[3.0nlo]~\cite{Ball:2014uwa} parton distribution function~(PDF).
The events were interfaced with
\PYTHIA[8.212]~\cite{Sjostrand:2014zea} using the A14 tune~\cite{ATL-PHYS-PUB-2014-021} and the \NNPDF[2.3lo]~\cite{Ball:2014uwa} PDF set.
The top quark and the $Z$ boson were decayed at LO using \MADSPIN~\cite{Frixione:2007zp,Artoisenet:2012st} to preserve spin correlations.
While the top quark was allowed to decay inclusively, the $Z$ boson decay was restricted to a pair of charged leptons.
The five-flavour scheme was used, where all the quark masses are set to zero, except the top quark. 
The renormalisation and factorisation scales were set to the top-quark mass.
The diagram removal scheme described in Ref.~\cite{Frixione:2008yi} was employed to handle the interference 
between \tWZ and \ttZ, and was applied to the \tWZ sample.
A sample with the alternative scheme described in Ref.~\cite{Demartin:2016axk} was produced to assess the associated systematic uncertainty.
The decays of bottom and charm hadrons were simulated using the \EVTGEN[1.2.0] program~\cite{Lange:2001uf}.


%%%%%%%%%%%%%%%%%%%%%%%%%%%%%%%%%%%%%%%%%%%%
%%%              tgamma                    %%%
%%%%%%%%%%%%%%%%%%%%%%%%%%%%%%%%%%%%%%%%%%%
\section[tgamma production]{\tgamma production}
\label{subsec:tgamma}

This section describes the MC samples used for the modelling of \tgamma production.
For now, it will be empty since the sample is still in production.

%%%%%%%%%%%%%%%%%%%%%%%%%%%%%%%%%%%%%%%%%%%
%%%              tttt                    %%%
%%%%%%%%%%%%%%%%%%%%%%%%%%%%%%%%%%%%%%%%%%%
\subsection[\tttt production]{\tttt production}
\label{subsec:tttt}

This section describes the MC samples used for the modelling of \tttt\ production.
Section~\ref{subsubsec:tttt_aMCP8} describes the \MGNLOPY[8] samples,
and Section~\ref{subsubsec:tttt_aMCH7} describes the \MGNLOHER[7] samples.

\subsubsection[MadGraph5\_aMC@NLO+Pythia8]{\MGNLOPY[8]}
\label{subsubsec:tttt_aMCP8}

\paragraph{Samples}
%\label{par:tttt_aMCP8_samples}

The descriptions below correspond to the samples in Table~\ref{tab:tttt_aMCP8}.

\begin{table}[htbp]
\begin{center}
\caption{Nominal \tttt\ samples produced with \MGNLOPY[8].} 
\label{tab:tttt_aMCP8}
\begin{tabular}{ l | l }
\hline
DSID range & Description \\
\hline
412043 & \tttt \\
\hline
\end{tabular}
\end{center}
\end{table}

\paragraph{Short description:}

The production of \tttt\ events was modelled using the \MGNLO[2.3.3]~\cite{Alwall:2014hca}
generator at NLO with the \NNPDF[3.1nlo]~\cite{Ball:2014uwa} parton distribution function~(PDF).
The events were interfaced with \PYTHIA[8.230]~\cite{Sjostrand:2014zea}~ using the A14 tune~\cite{ATL-PHYS-PUB-2014-021} and the
\NNPDF[2.3lo]~\cite{Ball:2014uwa} PDF set.
The decays of bottom and charm hadrons were simulated using the \EVTGEN[1.6.0] program~\cite{Lange:2001uf}. 

\paragraph{Long description:}

The production of \tttt\ events was modelled using the \MGNLO[2.3.3]~\cite{Alwall:2014hca}
generator, which provided matrix elements at next-to-leading order~(NLO) in the strong coupling constant \alphas
with the \NNPDF[3.1nlo]~\cite{Ball:2014uwa} parton distribution function~(PDF).
The functional form of the renormalisation and factorisation scales was set to $0.25\times \sum_i \sqrt{m^2_i+p^2_{\text{T},i}}$,
where the sum runs over all the particles generated from the matrix element calculation, following the Ref.~\cite{Frederix:2017wme}.
Top quarks were decayed at LO using \MADSPIN~\cite{Frixione:2007zp,Artoisenet:2012st} to preserve all spin correlations. 
The events were interfaced with \PYTHIA[8.230]~\cite{Sjostrand:2014zea} for the parton shower and hadronisation,
using the A14 set of tuned parameters~\cite{ATL-PHYS-PUB-2014-021} and the \NNPDF[2.3lo]~\cite{Ball:2014uwa} PDF set.
The decays of bottom and charm hadrons were simulated using the \EVTGEN[1.6.0] program~\cite{Lange:2001uf}.

\subsubsection[MadGraph5\_aMC@NLO+Herwig7]{\MGNLOHER[7]}
\label{subsubsec:tttt_aMCH7}

\paragraph{Samples}
%\label{par:tttt_aMCH7_samples}

The descriptions below correspond to the samples in Table~\ref{tab:tttt_aMCH7}.
\begin{table}[htbp]
\begin{center}
\caption{\tttt\ samples produced with \MGNLOHER[7].}
\label{tab:tttt_aMCH7}
\begin{tabular}{ l | l }
\hline
DSID range & Description \\
\hline
412044 & \tttt \\
\hline
\end{tabular}
\end{center}
\end{table}

\paragraph{Description:}

Additional \tttt\ samples were produced with the parton shower of the nominal samples replaced by
\HERWIG[7.04]~\cite{Bahr:2008pv,Bellm:2015jjp} to evaluate the impact of using a different parton shower and hadronisation model. The H7UE set of
tuned parameters~\cite{Bellm:2015jjp} and the \MMHT[lo] PDF set~\cite{Harland-Lang:2014zoa} were used.


%-------------------------------------------------------------------------------                         
% Jets
\section{Jet processes}
%\label{sec:MJ}

This section describes the MC samples used for the modelling of multijet production. 
Section~\ref{subsec:jets-pythia} describes the \PYTHIA[8] samples,
Section~\ref{subsec:jets-herwig} describes the \HERWIG[7] samples,
Section~\ref{subsec:jets-powheg} describes the \POWPY[8] samples,
and finally Section~\ref{subsec:jets-sherpa} describes the \SHERPA samples.

\subsection[Pythia 8]{\PYTHIA[8]}
\label{subsec:jets-pythia}

The descriptions below correspond to the samples in Table~\ref{tab:mj_pythia}.

\begin{table}[!htbp]
  \caption{Nominal multijet samples produced with \PYTHIA.}%
  \label{tab:mj_pythia}
  \centering
  \begin{tabular}{l l}
    \toprule
    DSID range & Description \\
    \midrule
    364700--364712 & \PYTHIA with shower weights \\
    \bottomrule
  \end{tabular}
\end{table}

\paragraph{Description:}

Multijet production was generated using \PYTHIA[8.230]~\cite{Sjostrand:2014zea} with leading-order matrix elements
for dijet production which were matched to the parton shower.
%  supplemented by matrix element corrections (MEC) for the hardest
%  emission.

The renormalisation and factorisation scales were set to the geometric
mean of the squared transverse masses of the two outgoing particles in the matrix element,
$\pTHatPythia = \sqrt{(\pTX[2][1] + m_1^2) (\pTX[2][2] + m_2^2)}$. 
The \NNPDF[2.3lo] PDF set~\cite{Ball:2012cx} was used in
the ME generation, the parton shower, and the simulation of the
multi-parton interactions. The A14~\cite{ATL-PHYS-PUB-2014-021}
set of tuned parameters was used. Perturbative uncertainties were estimated
through event weights~\cite{Mrenna:2016sih} that encompass variations
of the scales at which the strong coupling constant is evaluated in
the initial- and final-state shower as well as the PDF uncertainty in
the shower and the non-singular part of the splitting functions.


\paragraph{Additional description:}

The modelling of fragmentation and
hadronisation was based on the Lund string
model~\cite{Andersson:1983ia,Sjostrand:1984ic}. To populate the
inclusive jet \pt{} spectrum efficiently, the sample used a biased
phase-space sampling which was compensated for by a continuously decreasing
weight for the event. Specifically, events at a scale
\pTHatPythia scale were oversampled by a factor of
$(\pTHatPythia/\SI{10}{\GeV})^4$.


\subsection[Herwig 7.1]{\HERWIG[7.1]}
\label{subsec:jets-herwig}

The descriptions below correspond to the samples in Table~\ref{tab:mj_herwig}.

\begin{table}[!htbp]
  \caption{Multijet samples produced with \HERWIG[7].}%
  \label{tab:mj_herwig}
  \centering
  \begin{tabular}{l l}
    \toprule
    DSID range & Description \\
    \midrule
    364922--364929 & angular ordering in shower \HERWIG[7] \\
    364902--364909 & dipole shower \HERWIG[7] \\
    \bottomrule
  \end{tabular}
\end{table}

\paragraph{Description:}

Multijet production at next-to-leading order (NLO) was generated using \HERWIG[7.1.3]~\cite{Bellm:2017jjp}. 
The renormalisation and factorisation scales were set to the \pt\ of the leading jet. The
\MMHT[nlo]~\cite{Harland-Lang:2014zoa} PDF set was used for the matrix element calculation. 
Two sets of samples were generated, where one makes use of the default parton shower with angular ordering,
and the other uses the dipole shower as an alternative. The description of
hadronisation was based on the cluster model~\cite{Winter:2003tt} for both of these samples.
Two different samples with the same matrix elements and hadronisation allow the effects of using different 
parton shower models to be investigated. These samples include variations from the hard scattering and shower. 


\subsection[Powheg+Pythia8]{\POWPY[8]}
\label{subsec:jets-powheg}

The descriptions below correspond to the samples in Table~\ref{tab:mj_powheg}.

\begin{table}[!htbp]
  \caption{Multijet samples produced with \POWHEGBOX[v2].}%
  \label{tab:mj_powheg}
  \centering
  \begin{tabular}{l l}
    \toprule
    DSID range & Description \\
    \midrule
    361281--361289 & \POWPY[8] \\
    \bottomrule
  \end{tabular}
\end{table}

\paragraph{Description:}

Alternative samples of multijet production at NLO accuracy were produced with \POWHEGBOX[v2]~\cite{Nason:2004rx, Frixione:2007vw} 
interfaced to \PYTHIA[8]. These were generated with the dijet process as implemented in \POWHEGBOX[v2]~\cite{Alioli:2010xd}.
The \pt\ of the underlying Born configuration was taken as the renormalisation and factorisation scales
and the \NNPDF[3.0nlo]~\cite{Ball:2014uwa} parton distribution function (PDF) was used. \PYTHIA with the A14 tune and the
\NNPDF[2.3lo]~\cite{Ball:2012cx} PDF was used for the shower and multi-parton interactions. 
These samples included per-event weight variations for different perturbative scales in the matrix element, 
different parton distribution functions and their uncertainties, and the \PYTHIA perturbative
shower uncertainties. 


\subsection[Sherpa 2.2]{\SHERPA[2.2]}
\label{subsec:jets-sherpa}

The descriptions below correspond to the samples in Table~\ref{tab:mj_sherpa}.

\begin{table}[!htbp]
  \caption{Multijet samples produced with \SHERPA.}%
  \label{tab:mj_sherpa}
  \centering
  \begin{tabular}{l l}
    \toprule
    DSID range & Description \\
    \midrule
    364677--364685 & \SHERPA AHADIC \\
    364686--364694 & \SHERPA Lund \\
    \bottomrule
  \end{tabular}
\end{table}

\paragraph{Description:}

Multijet production samples were also generated using the \SHERPA[2.2.5]~\cite{Bothmann:2019yzt} generator. 
The matrix element calculation was included for the $2\rightarrow2$ process at leading order, and the default \SHERPA parton
shower~\cite{Schumann:2007mg} based on Catani--Seymour dipole factorisation was used for the showering with \pt\
ordering, using the \CT[14nnlo] PDF set~\cite{Dulat:2015mca}.
The first of these samples made use of the dedicated \SHERPA AHADIC model for hadronisation~\cite{Winter:2003tt}, 
based on cluster fragmentation ideas. A second sample was generated with the same configuration but using the \SHERPA interface to
the Lund string fragmentation model of \PYTHIA[6]~\cite{Sjostrand:2006za} and its decay tables.  
These two sets of samples were used to evaluate uncertainties stemming from the hadronisation modelling. 

% need to comment on the slicing of these samples. JZ1 to JZ4 are JZW filtered to improve statistics for and an equal number of simulated events, while slices JZ9-JZ12 have been combined into a single JZ9 plus slice.

%-------------------------------------------------------------------------------
% Photon processes                                                                                             
\chapter{Photon processes}
%\label{sec:GJ}

The following paragraphs describe the set-up of the current ATLAS \(\gamma\)+jets and \(\gamma\gamma\)+jets baseline samples.

\section[Sherpa (MEPS@NLO)]{\SHERPA (\MEPSatNLO)}
%\label{sec:gammajets-sherpa-nlo}

\subsection*{Samples}
%\label{sec:gammajets-sherpa-nlo-samples}

The descriptions below correspond to the samples in \cref{tab:gammajets-sherpa-nlo}.

\begin{table}[!htbp]
  \caption{\(\gamma\)+jets and \(\gamma\gamma\)+jets samples with \SHERPA NLO\@.}%
  \label{tab:gammajets-sherpa-nlo}
  \centering
  \begin{tabular}{l l}
    \toprule
    DSID range & Description \\
    \midrule
    364541--364547 & single photon \\
    364350--364354 & diphoton  \\
    \bottomrule
  \end{tabular}
\end{table}


%same description as in~\cref{sec:vjets-sherpa-vjets}
\subsection{\(\gamma\)+jets}
%\label{sec:gammajets-sherpa-nlo-singlephoton}

\paragraph{Short description:}

Prompt single-photon production was simulated with the
\SHERPA[2.2]~\cite{Bothmann:2019yzt} generator. In this set-up, NLO-accurate
matrix elements for up to two partons, and LO-accurate matrix elements for up
to four partons were calculated with the Comix~\cite{Gleisberg:2008fv} and
\OPENLOOPS~\cite{Buccioni:2019sur,Cascioli:2011va,Denner:2016kdg} libraries. They were matched
with the \SHERPA parton shower~\cite{Schumann:2007mg} using the \MEPSatNLO
prescription~\cite{Hoeche:2011fd,Hoeche:2012yf,Catani:2001cc,Hoeche:2009rj}
with a dynamic merging cut~\cite{Siegert:2016bre} of \SI{20}{\GeV}.
Photons were required to be isolated according to
a smooth-cone isolation criterion~\cite{Frixione:1998jh}. Samples were generated using the
\NNPDF[3.0nnlo] PDF set~\cite{Ball:2014uwa}, along with the dedicated set of
tuned parton-shower parameters developed by the \SHERPA authors.


\paragraph{Long description:}

Prompt single-photon production was simulated with the
\SHERPA[2.2]~\cite{Bothmann:2019yzt} parton shower Monte Carlo
generator. In this set-up, NLO-accurate
matrix elements for up to two partons, and LO-accurate matrix elements for up
to four partons were calculated with the Comix~\cite{Gleisberg:2008fv} and
\OPENLOOPS~\cite{Buccioni:2019sur,Cascioli:2011va,Denner:2016kdg} libraries.
The default \SHERPA parton shower~\cite{Schumann:2007mg} based on
Catani--Seymour dipole factorisation and the cluster hadronisation model~\cite{Winter:2003tt}
were used. They employed the dedicated set of tuned parameters developed by the
\SHERPA authors for this generator version and the \NNPDF[3.0nnlo] PDF
set~\cite{Ball:2014uwa}.

The NLO matrix elements for a given jet multiplicity were matched to the parton
shower using a colour-exact variant of the MC@NLO
algorithm~\cite{Hoeche:2011fd}. Different jet multiplicities were then merged
into an inclusive sample using an improved CKKW matching
procedure~\cite{Catani:2001cc,Hoeche:2009rj} which was extended to NLO
accuracy using the \MEPSatNLO prescription~\cite{Hoeche:2012yf}.
The merging cut was set dynamically at a scale of \SI{20}{\GeV}
according to the prescription in Ref.~\cite{Siegert:2016bre}.

The renormalisation and factorisation scales for the photon-plus-jet core
process were set to the transverse energy of the photon, \(E_\text{T}^\gamma\).
The strong coupling constant was set to \(\alphas (m_Z)= 0.118\) and the QED coupling
constant was evaluated in the Thomson limit. Photons from the matrix elements were required
to be central, by being within the rapidity range \(|y_{\gamma}|<2.7\), and isolated according to a
smooth-cone isolation criterion~\cite{Frixione:1998jh} with \(\delta_0=0.1\), \(\epsilon_{\gamma}=0.1\) and \(n=2\).

The effects of QCD scale uncertainties were evaluated~\cite{Bothmann:2016nao} using
seven-point variations of the factorisation and renormalisation scales in the matrix elements.
The scales were varied independently by factors of \(0.5\) and \(2\), avoiding variations in opposite directions.

PDF uncertainties for the nominal PDF set were
evaluated using the 100 variation replicas, as well as \(\pm 0.001\) shifts
of \alphas. Additionally, the results were cross-checked using the central values of the
\CT[14nnlo]~\cite{Dulat:2015mca} and \MMHT[nnlo]~\cite{Harland-Lang:2014zoa}
PDF sets.

%\textcolor{red}{ Several Monte Carlo slices were generated to ensure good statistics over the whole phase space. In each of them, the photon was required to fall within a \(E_{T}^{\gamma}\) range that depends on the slice avoiding any double-counting when the different slices are combined.}


\subsection[yy+jets]{ \(\gamma\gamma\)+jets}
%\label{sec:gammajets-sherpa-nlo-diphoton}

\paragraph{Short description:}

Prompt diphoton production was simulated with the
\SHERPA[2.2]~\cite{Bothmann:2019yzt} generator. In this set-up, NLO-accurate
matrix elements for up to one parton, and LO-accurate matrix elements for up
to three partons were calculated with the Comix~\cite{Gleisberg:2008fv} and
\OPENLOOPS~\cite{Buccioni:2019sur,Cascioli:2011va,Denner:2016kdg} libraries. They were matched
with the \SHERPA parton shower~\cite{Schumann:2007mg} using the \MEPSatNLO
prescription~\cite{Hoeche:2011fd,Hoeche:2012yf,Catani:2001cc,Hoeche:2009rj}
with a dynamic merging cut~\cite{Siegert:2016bre} of \SI{10}{\GeV}.
Photons were required to be isolated according to a smooth-cone isolation
criterion~\cite{Frixione:1998jh}. Samples were generated using the
\NNPDF[3.0nnlo] PDF set~\cite{Ball:2014uwa}, along with the dedicated set of tuned
parton-shower parameters developed by the \SHERPA authors.


\paragraph{Long description:}

Prompt diphoton production was simulated with the
\SHERPA[2.2]~\cite{Bothmann:2019yzt} parton shower Monte Carlo
generator. In this set-up, NLO and LO-accurate
matrix elements were calculated with the Comix~\cite{Gleisberg:2008fv} and
\OPENLOOPS~\cite{Buccioni:2019sur,Cascioli:2011va,Denner:2016kdg} libraries.
The default \SHERPA parton shower~\cite{Schumann:2007mg} based on
Catani--Seymour dipole factorisation and the cluster hadronisation model~\cite{Winter:2003tt}
were used. They employed the dedicated set of tuned parameters developed by the
\SHERPA authors for this generator version and the \NNPDF[3.0nnlo]
PDF set~\cite{Ball:2014uwa}.

The NLO matrix elements for a given jet multiplicity were matched to the parton
shower using a colour-exact variant of the MC@NLO algorithm~\cite{Hoeche:2011fd}.
Different jet multiplicities were then merged
into an inclusive sample using an improved CKKW matching
procedure~\cite{Catani:2001cc,Hoeche:2009rj} which was extended to NLO
accuracy using the \MEPSatNLO prescription~\cite{Hoeche:2012yf}.
The merging cut was set dynamically to a scale of \SI{20}{\GeV},
according to the prescription in Ref.~\cite{Siegert:2016bre}.

The renormalisation and factorisation scales for the diphoton core process
were set to the invariant mass of the photon pair, \(m_{\gamma\gamma}\).
The strong coupling constant was set to \(\alphas (m_Z)= 0.118\) and the QED coupling
constant was evaluated in the Thomson limit. Photons from the matrix elements were required to be central,
by being within the rapidity range \(|y_{\gamma}|<2.7\), and isolated according to a smooth-cone isolation
criterion~\cite{Frixione:1998jh} with \(\delta_0=0.1\), \(\epsilon_{\gamma}=0.1\) and \(n=2\).
Additionally, the photons were required to be separated by \(\Delta R(\gamma_1,\gamma_2) > 0.2\).


The effects of QCD scale uncertainties were evaluated~\cite{Bothmann:2016nao} using
seven-point variations of the factorisation and renormalisation scales in the matrix elements.
The scales were varied independently by factors of \(0.5\) and \(2\), avoiding variations in opposite directions.

PDF uncertainties for the nominal PDF set were
evaluated using the 100 variation replicas, as well as \(\pm 0.001\) shifts
of \alphas. Additionally, the results were cross-checked using the central values of the
\CT[14nnlo]~\cite{Dulat:2015mca} and \MMHT[nnlo]~\cite{Harland-Lang:2014zoa}
PDF sets.
%\textcolor{red}{ Several Monte Carlo slices were generated to ensure good statistics over the whole phase space. In each of them, the photon pair was required to fall within a \(m_{\gamma\gamma}\) range that depends on the slice avoiding any double-counting when the different slices are combined.}


\section[Sherpa (MEPS@NLO)]{\SHERPA (\MEPSatNLO)}
%\label{sec:gammajets-sherpa-lo}

\subsection*{Samples}
%\label{sec:gammajets-sherpa-lo-samples}

The descriptions below correspond to the samples in \cref{tab:gammajets-sherpa-lo}.

\begin{table}[!htbp]
  \caption{\(\gamma\)+jets and  \(\gamma\gamma\)+jets samples with \SHERPA LO\@.}%
  \label{tab:gammajets-sherpa-lo}
  \centering
  \begin{tabular}{l l}
    \toprule
    DSID range & Description \\
    \midrule
    361039--361062 & single photon \\
    303727--303742 & diphoton \\
    \bottomrule
  \end{tabular}
\end{table}


\subsection[y+jets]{\(\gamma\)+jets}
%\label{sec:gammajets-sherpa-lo-singlephoton}

\paragraph{Description:}

Prompt single-photon production was simulated using the \SHERPA[2.1]~\cite{Bothmann:2019yzt}
generator. The tree-level matrix elements, generated for up to three
additional partons, were merged with the initial- and final-state parton showers using the
\MEPSatLO prescription~\cite{Hoeche:2009rj}. The \CT[10nlo] set of PDFs~\cite{Lai:2010vv} was
used to parameterise the proton structure in conjunction with the dedicated set of tuned
parton-shower parameters developed by the \SHERPA authors for this generator version. A
modified version of the cluster model~\cite{Winter:2003tt} was used
for the description of the fragmentation into hadrons. Photons from the matrix elements were
required to be isolated according to a smooth-cone hadronic isolation criterion~\cite{Frixione:1998jh}
with \(\delta_0=0.3\), \(\epsilon_{\gamma}=0.025\) and \(n=2\).


\subsection[yy+jets]{ \(\gamma\gamma\)+jets}
%\label{sec:gammajets-sherpa-lo-diphoton}

\paragraph{Description:}

Prompt diphoton production was simulated using the \SHERPA[2.1]~\cite{Bothmann:2019yzt}
generator. The tree-level matrix elements, generated for up to two
additional partons, were merged with the initial- and final-state parton showers using the
\MEPSatLO prescription~\cite{Hoeche:2009rj}. The \CT[10nlo] set of PDFs~\cite{Lai:2010vv} was
used to parameterise the proton structure in conjunction with the dedicated set of tuned
parton-shower parameters developed by the \SHERPA authors for this generator version. A
modified version of the cluster model~\cite{Winter:2003tt} was used
for the description of the fragmentation into hadrons. Photons from the matrix elements were
required to be isolated according to a smooth-cone hadronic isolation criterion~\cite{Frixione:1998jh}
with \(\delta_0=0.3\), \(\epsilon_{\gamma}=0.025\) and \(n=2\). Additionally, the photons were
required to be separated by \(\Delta R(\gamma_1,\gamma_2) > 0.2\).


\section[Pythia (LO)]{\PYTHIA (LO)}
%\label{sec:gammajets-pythia-lo}

The descriptions below correspond to the samples in \cref{tab:gammajets-pythia-lo}.

\begin{table}[!htbp]
  \caption{\(\gamma\)+jets and  \(\gamma\gamma\)+jets samples with \PYTHIA.}%
  \label{tab:gammajets-pythia-lo}
  \centering
  \begin{tabular}{l l}
    \toprule
    DSID range & Description \\
    \midrule
    423099--423112 & single photon \\
    344008, 302520--34, 364423 & diphoton \\
    \bottomrule
  \end{tabular}
\end{table}


\subsection[y+jets]{\(\gamma\)+jets}
%\label{sec:gammajets-pythia-lo-singlephoton}

\paragraph{Description:}

Prompt single-photon production was simulated using the \PYTHIA[8.186]~\cite{Sjostrand:2007gs} generator.
Events were generated using tree-level matrix elements for photon-plus-jet final
states as well as LO QCD dijet events, with the inclusion of initial-
and final-state parton showers. The fragmentation component was
modelled by final-state QED radiation arising from calculations of all
\(2\rightarrow 2\) QCD processes. The \NNPDF[2.3lo]~\cite{Ball:2012cx} PDF
set was used in the matrix element calculation, the parton shower, and
the simulation of the multi-parton interactions. The samples
include a simulation of the underlying event with parameters set
according to the A14 tune~\cite{ATL-PHYS-PUB-2014-021}. The Lund
string model~\cite{Andersson:1983ia,Sjostrand:1984ic} was used for the
description of the fragmentation into hadrons.


\subsection[yy+jets]{ \(\gamma\gamma\)+jets}
%\label{sec:gammajets-pythia-lo-diphoton}

\paragraph{Description:}

Prompt diphoton production was simulated using the
\PYTHIA[8.186]~\cite{Sjostrand:2007gs} generator. Events were
generated using tree-level matrix elements for diphoton final states,
with the inclusion of initial- and final-state parton showers.  The
fragmentation component was modelled by final-state QED radiation
arising from calculations of photon-plus-jet processes in dedicated
samples. The \NNPDF[2.3lo]~\cite{Ball:2012cx} PDF set was used in the
matrix element calculation, the parton shower, and in the simulation of the
multi-parton interactions. The samples include a simulation of the
underlying event with parameters set according to the A14
tune~\cite{ATL-PHYS-PUB-2014-021}. The Lund string
model~\cite{Andersson:1983ia,Sjostrand:1984ic} was used for the
description of the fragmentation into hadrons.



%-------------------------------------------------------------------------------


%-------------------------------------------------------------------------------
%\section*{Acknowledgements}
%-------------------------------------------------------------------------------

%% Acknowledgements for papers with collision data
% Version 27-Jan-2020

% Standard acknowledgements start here
%----------------------------------------------

We thank CERN for the very successful operation of the LHC, as well as the
support staff from our institutions without whom ATLAS could not be
operated efficiently.

We acknowledge the support of ANPCyT, Argentina; YerPhI, Armenia; ARC, Australia; BMWFW and FWF, Austria; ANAS, Azerbaijan; SSTC, Belarus; CNPq and FAPESP, Brazil; NSERC, NRC and CFI, Canada; CERN; CONICYT, Chile; CAS, MOST and NSFC, China; COLCIENCIAS, Colombia; MSMT CR, MPO CR and VSC CR, Czech Republic; DNRF and DNSRC, Denmark; IN2P3-CNRS and CEA-DRF/IRFU, France; SRNSFG, Georgia; BMBF, HGF and MPG, Germany; GSRT, Greece; RGC and Hong Kong SAR, China; ISF and Benoziyo Center, Israel; INFN, Italy; MEXT and JSPS, Japan; CNRST, Morocco; NWO, Netherlands; RCN, Norway; MNiSW and NCN, Poland; FCT, Portugal; MNE/IFA, Romania; MES of Russia and NRC KI, Russia Federation; JINR; MESTD, Serbia; MSSR, Slovakia; ARRS and MIZ\v{S}, Slovenia; DST/NRF, South Africa; MINECO, Spain; SRC and Wallenberg Foundation, Sweden; SERI, SNSF and Cantons of Bern and Geneva, Switzerland; MOST, Taiwan; TAEK, Turkey; STFC, United Kingdom; DOE and NSF, United States of America. In addition, individual groups and members have received support from BCKDF, CANARIE, Compute Canada and CRC, Canada; ERC, ERDF, Horizon 2020, Marie Sk{\l}odowska-Curie Actions and COST, European Union; Investissements d'Avenir Labex, Investissements d'Avenir Idex and ANR, France; DFG and AvH Foundation, Germany; Herakleitos, Thales and Aristeia programmes co-financed by EU-ESF and the Greek NSRF, Greece; BSF-NSF and GIF, Israel; CERCA Programme Generalitat de Catalunya and PROMETEO Programme Generalitat Valenciana, Spain; G\"{o}ran Gustafssons Stiftelse, Sweden; The Royal Society and Leverhulme Trust, United Kingdom.

The crucial computing support from all WLCG partners is acknowledged gratefully, in particular from CERN, the ATLAS Tier-1 facilities at TRIUMF (Canada), NDGF (Denmark, Norway, Sweden), CC-IN2P3 (France), KIT/GridKA (Germany), INFN-CNAF (Italy), NL-T1 (Netherlands), PIC (Spain), ASGC (Taiwan), RAL (UK) and BNL (USA), the Tier-2 facilities worldwide and large non-WLCG resource providers. Major contributors of computing resources are listed in Ref.~\cite{ATL-GEN-PUB-2016-002}.

%----------------------------------------------
% Created with Glance <Atlas.Glance@cern.ch>


%The \texttt{atlaslatex} package contains the acknowledgements that were valid 
%at the time of the release you are using.
%These can be found in the \texttt{acknowledgements} subdirectory.
%When your ATLAS paper or PUB/CONF note is ready to be published,
%download the latest set of acknowledgements from:\\
%\url{https://twiki.cern.ch/twiki/bin/view/AtlasProtected/PubComAcknowledgements}

%The supporting notes for the analysis should also contain a list of contributors.
%This information should usually be included in \texttt{mydocument-metadata.tex}.
%The list should be printed either here or before the table of contents.


%-------------------------------------------------------------------------------
%\clearpage
%\appendix
%\part*{Appendix}
%\addcontentsline{toc}{part}{Appendix}
%-------------------------------------------------------------------------------

%In a paper, an appendix is used for technical details that would otherwise disturb the flow of the paper.
%Such an appendix should be printed before the Bibliography.


%-------------------------------------------------------------------------------
% If you use biblatex and either biber or bibtex to process the bibliography
% just say \printbibliography here
\printbibliography
% If you want to use the traditional BibTeX you need to use the syntax below.
%\bibliographystyle{bibtex/bst/atlasBibStyleWoTitle}
%\bibliography{MC-snippets,bibtex/bib/ATLAS}
%-------------------------------------------------------------------------------

\end{document}
