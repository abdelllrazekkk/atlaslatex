%-------------------------------------------------------------------------------
% Guide to the use of the ATLAS LaTeX package
%-------------------------------------------------------------------------------
\documentclass{style/atlasdoc}

\usepackage{style/atlaspackage}
\usepackage{style/atlasphysics}

% Use biblatex and biber for the bibliography
\usepackage[backend=biber,
	style=numeric-comp,sorting=none,block=ragged,firstinits=true]{biblatex}
\addbibresource{../atlas-latex.bib}

% Paths for figures
\graphicspath{{../../figures/},{figures/}}

% Shorthand for \phantom to use in tables
\newcommand{\pho}{\phantom{0}}
\newcommand{\BibTeX}{\textsc{Bib\TeX}}
\newcommand{\Macro}[1]{\textbackslash #1\xspace}

%-------------------------------------------------------------------------------
% Generic document information
%-------------------------------------------------------------------------------

% Author and title for the PDF file
\hypersetup{pdftitle={ATLAS LaTeX guide},pdfauthor={Ian Brock}}

\atlastitle{Users guide to the ATLAS \LaTeX\ package}

\author{Ian Brock}

% \draftversion{1.0}

\abstracttext{%
  The usage of the ATLAS \LaTeX\ templates is documented here.
  Guidelines on how to format \BibTeX\ entries are given.

	This document was generated using version \ATPackageVersion\ of the ATLAS \LaTeX\ package.
}

%-------------------------------------------------------------------------------
% This is where the document really begins
%-------------------------------------------------------------------------------
\begin{document}

\tableofcontents
\clearpage

%-------------------------------------------------------------------------------
\section{Introduction}
\label{sec:intro}
%-------------------------------------------------------------------------------

This collection of ATLAS \LaTeX\ templates, style files and documentation
can be used for papers, preprints and notes. 
The class is available from the web pages of the Publication Committee~\cite{pubcom} and from 
SVN~\cite{pubcom-svn}.

The collection replaces and hopefully improves on the previous packages. It supersedes:
\begin{itemize}
\item \texttt{atlasnote-00-04-05}
\item \texttt{atlascover-00-00-11}
\item \texttt{atlaspreprint-00-00-05}
\item \texttt{atlasbib-00-00-04}
\end{itemize}

The collection includes the \texttt{atlasdoc} class, useful style files
and documentation of the package.
The documentation is provided as both PDF files and \LaTeX\ documents
that should provide examples of how to use the package and how to write
good \LaTeX.

The design principle is that you have a main document and 
the style files and \texttt{atlasdoc} class are in a subdirectory \texttt{style}.
This subdirectory can of course be a link to a centrally maintained \texttt{style} directory.

The usual procedure is that for each document that you create,
you first unpack the latest version of \texttt{atlaslatex} and
then create your main document in the top-level directory.
This structure means that it is easier to update the style files if a new version of
\texttt{atlaslatex} is released. 
Each document can then be independent of the \texttt{atlaslatex} release.

To create a new document you can issue the commands:
%
\begin{verbatim}
make new BASENAME=mydocument
\end{verbatim}
%
This command copies \texttt{atlas-document.tex}, 
{atlas-document-metadata.tex} and {atlas-document-contribute.tex}
to \texttt{mydocument.tex},  \texttt{mydocument-metadata.tex} and
\texttt{mydocument-contribute.tex}.
It also creates empty files \texttt{mydocument-defs.sty} and \texttt{mydocument.bib}.

In the \texttt{Makefile} you should change the \texttt{BASENAME} to the name of your document.
You can then compile your document with the command: \texttt{make}.


%-------------------------------------------------------------------------------
\section{Title, author and document numbers}
\label{sec:metadata}
%-------------------------------------------------------------------------------

An ATLAS document usually collects a series of labels over time.
These include the name of the analysis, e.g.\ \texttt{BPHY-2013-04}, 
and the CERN preprint number.
In addition the abstract and title are needed on the front page of the document and also on various cover pages.
The editors of the document, the editorial board members and the contributors to the analysis
also need to be specified at some point.

While all this information can be collected in the main file of he document, 
it is probably better to put it all in a separate document.
By default this file is called \texttt{mydocument-metadata.tex}.


%-------------------------------------------------------------------------------
\section{The \texttt{atlasdoc} class}
\label{app:atlasdoc}
%-------------------------------------------------------------------------------

This document has been typeset using the \texttt{atlasdoc.cls} class.

\texttt{atlasdoc.cls} derives from the standard \LaTeX{} {article.cls}
class, thus all the usual commands and options you would have used
with \texttt{article} will work with it. For instance, this paper has
been produced using this very simple preamble:

\begin{verbatim}
  \documentclass{style/atlasdoc}
  \usepackage{style/atlaspackage}
  \usepackage{style/atlasphysics}
  \graphicspath{{../},{figures/}}
\end{verbatim}

The style file \texttt{atlasphysics.sty} defines a lot of useful
macros for particles and more. See the separate document
\texttt{atlas-physics.pdf}~\cite{atlas-physics} for details.

%-------------------------------------------------------------------------------
\subsection{Dependencies}
%-------------------------------------------------------------------------------

The \texttt{atlasdoc} class depends on these packages, which presence in
your system is required:
\begin{itemize}
\item \texttt{graphicx}
\item \texttt{color}
\end{itemize}
These should be installed in any modern \LaTeX{}
installation, while the latter is part of the {\tt ednotes} package
bundle and is directly provided with this package; \texttt{atlasnote} was
tested on a CERN \texttt{lxplus} node and worked out of the box. 
The \texttt{atlasnote} class works both with \LaTeX{} and pdf\LaTeX{}.

The style file \texttt{atlaspackage.sty} includes further packages.
You can steer how mnay by using the option \texttt{minimal}, \texttt{default} or \texttt{full}.
The minimal set is:
\begin{itemize}
\item \texttt{mathptmx}
\item \texttt{lineno}
\end{itemize}

The default set adds in adition:

Finally the full set includes further packages that may be useful:

If you wish to use the \texttt{atlascover} package with the 
\texttt{atlasnote} class, load the latest version of the package in your
system, and invoke it using the \texttt{coverpage} option of the class:
\begin{verbatim}
  \documentclass[11pt,a4paper,coverpage]{style/atlasdoc}
\end{verbatim}
instead of the the usual \texttt{usepackage} command: this will ensure
that the cover page is produced before the note title page.

You can include it with \Macro{usepackage} if you prefer.
In this case you need the option \texttt{nomaketitle} in the \texttt{atlasdoc} class
and you include the \Macro{maketitle} command after \verb|\begin{documenmt}|.

%-------------------------------------------------------------------------------
\subsection{Custom commands}
%-------------------------------------------------------------------------------

The \texttt{atlasdoc} class implements some custom commands, mainly
used to typeset the frontpage content:

\begin{itemize}

  \item {\verb|\title{<Title>}|} typesets the paper title. If not
    given, a dummy \emph{Title goes here} title will be produced.

  \item {\verb|\author{<Author>}|} typesets the paper author. If not
    explicitly given, \emph{The ATLAS Collaborations} will be used by
    default. Note that the \verb|\author{}| command is pretty limited
    in case you want to display multiple author names and multiple
    affiliations. For this use case the \verb|authblk.sty| package is
    provided; this is a typical example of its use:
    \begin{verbatim}
\usepackage{authblk}
\renewcommand\Authands{, } % avoid ``. and'' for last author
\renewcommand\Affilfont{\itshape\small} % affiliation formatting

\author[a]{First Author}
\author[a]{Second Author}
\author[b]{Third Author}

\affil[a]{One Institution}
\affil[b]{Another Institution}
    \end{verbatim}

  \item {\verb|\abstracttext{<The abstract text>}|} typesets the
    abstract in the front page.

  \item {\verb|\date{<Date>}|} typesets the paper date. If not
    explicitly given, the current date (\verb|\today|) will be used.

  \item {\verb|\draftversion{<Draft Version>}|} displays the draft
    version on the front page, a DRAFT banner on all the other page
    headings, and add line numbers to all text to easy commenting abd
    reviewing. Can be omitted.

  \item {\verb|\journal{<Journal Name>}|} displays the phrase \emph{to
    be submitted to Journal Name} at the bottom of the front page. Can
    be omitted.

  \item {\verb|\skipbeforetitle{<length>}|} sets the distance between
    the title page header and the note title. The default value should
    be fine for most notes, but in case you have a long list of
    authors or a lengthy abstract you can use this command to buy
    some extra space. Note that \verb|<length>| can also be negative
    (use it at your own risk!).

\end{itemize}

\noindent {\tt emptynote.tex} contains a basic skeleton that can be
used to start typing a new note using the {\tt atlasnote} class. All
the custom commands described above are used in this example file, in
order to demonstrate their use.

%-------------------------------------------------------------------------------
\section{Symbols in \texttt{atlasphysics.sty}}
\label{sec:atlasphysics}
%-------------------------------------------------------------------------------

Many useful symbols are defined in \texttt{atlasphysics.sty}.
While this used to be a very long file, it has now been split into several smaller files,
which can be included or not using options.


%-------------------------------------------------------------------------------
\section{Miscellaneous \LaTeX{} tips}
%-------------------------------------------------------------------------------
\label{sec:latex}

\subsection{Graphics}

Use the {\tt graphicx} package \cite{} to include your plots and
figure. The use of older packages like {\tt espfig} is deprecated.
Since the {\tt graphicx} package is required by the {\tt atlasnote}
class, it is automatically loaded when using it, and there is no need
to explicitly included it in the document preamble.

Always include your graphics file without mentioning the file
extension. For instance, if you want to include the \texttt{figure.eps}
file, you should use a syntax like this:
\begin{verbatim}
  \includegraphics[width=\textwidth]{figure}
\end{verbatim}
This will allow you to compile your document using either \LaTeX{} or
pdf\LaTeX{} without changing your source file: you can in fact have
both \texttt{figure.eps} and \texttt{figure.pdf} in your working directory,
and the proper one will be picked up according to the processing method
you choose.

It is a good habit to keep you graphics file in a separate
sub-directory (e.g.\ in \texttt{figure/}. In this case you can include them
by mentioning it explicitly every time:
\begin{verbatim}
  \includegraphics[width=\textwidth]{figures/figure}
\end{verbatim}
or by telling the \texttt{graphicx} package where to look
for them, by using this command:
\begin{verbatim}
  \graphicspath{{figures/}}
\end{verbatim}


\subsection{Definitions}

You can use \verb|\ensuremath| in definitions, so that they will work
in both text mode and math mode, e.g.
\verb|\newcommand{\UoneS}{\ensuremath{\Upsilon(\mathrm{1S})}}| to get
\UoneS{} in either mode (\verb|\UoneS{}| or \verb|$\UoneS$|).


\section{Remarks on units and symbols}

As discussed in the ATLAS Paper Template~\cite{atlas-paper},
it is highly recommended to use a units package to format your units properly.
The package \texttt{siunitx} works very well and is the package of choice.
Alternatives include \texttt{units} and \texttt{hepunits},
which is based on \texttt{SIunits}.

The basic command to use in \texttt{siunitx} is \verb|\SI{20}{\GeV}| to get
\SI{20}{\GeV}. 
There are also several other useful commands for specifying ranges:
\verb|\numrange| for a range of numbers and \verb|SIrange| for a range of numbers with a unit. 
Options exist for specifying how they are formatted.
The options can be set for an individual command or for the whole document.
For example, in this document I have specified the options:
\verb|\sisetup{separate-uncertainty, range-units = repeat, detect-family = true}|.
and
\verb|\sisetup{group-digits=integer, group-minimum-digits=4}|.

In addition several extra units are defined:
\begin{itemize}
\item \verb|\micron| for \si{\micron};
\item \verb|\mrad| for \si{\mrad};
\item \verb|\nb| for \si{\nb};
\item \verb|\pb| for \si{\pb};
\item \verb|\fb| for \si{\fb}.
\end{itemize}
Use the syntax \verb|\SI{20.3}{\per\fb}| to get \SI{20.3}{\per\fb}.

Some things to note about using \texttt{siunitx}:
\begin{itemize}
\item It tries to isolate itself from other packages.
	If you just want to write \si{\GeV} in your text,
	then you must write \verb|\si{\GeV}|.
\item It also contains two new column specifiers for tables ``S'' and ``s'',
	which are extremely useful for formatting tables properly.
\end{itemize}


%-------------------------------------------------------------------------------
\section{\texttt{atlasdoc} and \texttt{atlasnote}}
\label{sec:old}
%-------------------------------------------------------------------------------

The \texttt{atlasdoc} class replaces and supersedes \texttt{atlasnote}.
The decision was taken to give the class a new name, as it is supposed to be
able to be used for (almost) all ATLAS documents.
In addition, some small changes in the user setup are necessary to use the new
class, style files and templates.

All style files are collected in the \texttt{style} subdirectory.
It is assumed that this directory is a direct subdirectory of you main \LaTeX\ file.
If you want to keep the style files in a central place you can either put them in
\verb|${HOME}/texmf/tex/latex| or create a link from your main directoryto the location of
your \texttt{style} directory.

The main changes the user has to make are:
\begin{itemize}
\item Change the class name from \texttt{atlasnote} to \texttt{style/atlasdoc};
\item Add \verb|\usepackage{style/atlaspackage}| at the beginning of the document;
\item Change \verb|\usepackage{atlasphysics}| to \verb|\usepackage{style/atlasphysics}|; 
\item Use the macro \Macro{atlastitle} instead of \Macro{title}.
\end{itemize}

All the documentation now uses \texttt{biblatex} and \texttt{biber} instead of \BibTeX.

The class and style files have been cleaned up and things 
that were thought to no longer be necessary have been removed.
These pieces have been collected in \texttt{style/atlasnote-obsolete.sty} in case they are needed.

\end{document}
