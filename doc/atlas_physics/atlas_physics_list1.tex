\newpage
%-------------------------------------------------------------------------------
\section{\File{atlasjournal.sty}}

Turn on including these definitions with the option \Option{journal=true} and off with the option \Option{journal=false}.

% Journal names used in ATLAS documents
%
\RequirePackage{xspace}

\newcommand*{\AcPA}{Acta Phys. Austriaca\xspace}
\newcommand*{\ARevNS}{Ann. Rev.\ Nucl.\ Sci.\xspace}
\newcommand*{\CPC}{Comp.\ Phys.\ Comm.\xspace}
\newcommand*{\EPJ}{Eur.\ Phys.\ J.\xspace}
\newcommand*{\EPJC}{Eur.\ Phys.\ J.\ C\xspace}
\newcommand*{\FortP}{Fortschr.\ Phys.\xspace}
\newcommand*{\IJMP}{Int.\ J.\ Mod.\ Phys.\xspace}
\newcommand*{\JETP}{Sov.\ Phys.\ JETP\xspace}
\newcommand*{\JETPL}{JETP Lett.\xspace}
\newcommand*{\JaFi}{Jad.\ Fiz.\xspace}
\newcommand*{\JMP}{J.\ Math.\ Phys.\xspace}
\newcommand*{\MPL}{Mod.\ Phys.\ Lett.\xspace}
\newcommand*{\NCim}{Nuovo Cimento\xspace}
\newcommand*{\NIM}{Nucl.\ Instrum.\ Meth.\xspace}
\newcommand*{\NP}{Nucl.\ Phys.\xspace}
\newcommand*{\NPB}{Nucl.\ Phys.\ B\xspace}
\newcommand*{\PL}{Phys.\ Lett.\xspace}
\newcommand*{\PLB}{Phys.\ Lett.\ B\xspace}
\newcommand*{\PR}{Phys.\ Rev.\xspace}
\newcommand*{\PRC}{Phys.\ Rev.\ C\xspace}
\newcommand*{\PRD}{Phys.\ Rev.\ D\xspace}
\newcommand*{\PRL}{Phys.\ Rev.\ Lett.\xspace}
\newcommand*{\PRep}{Phys.\ Rep.\xspace}
\newcommand*{\RMP}{Rev.\ Mod.\ Phys.\xspace}
\newcommand*{\ZfP}{Z.\ Phys.\xspace}
%
% +--------------------------------------------------------------------+
% |                                                                    |
% |  Useful abbreviations in references                                |
% |                                                                    |
% +--------------------------------------------------------------------+
%
\newcommand*{\collab}{Collaboration\xspace}



\newpage
%-------------------------------------------------------------------------------
\section{\File{atlasmisc.sty}}

Turn on including these definitions with the option \Option{misc=true} and off with the option \Option{misc=false}.

\begin{xtabular}{ll}
\verb|\pT| & \pT \\
\verb|\pt| & \pt \\
\verb|\pTX| & \pTX \\
\verb|\ET| & \ET \\
\verb|\eT| & \eT \\
\verb|\et| & \et \\
\verb|\HT| & \HT \\
\verb|\pTsq| & \pTsq \\
\verb|\MET| & \MET \\
\verb|\met| & \met \\
\verb|\sumET| & \sumET \\
\verb|\EjetRec| & \EjetRec \\
\verb|\PjetRec| & \PjetRec \\
\verb|\EjetTru| & \EjetTru \\
\verb|\PjetTru| & \PjetTru \\
\verb|\EjetDM| & \EjetDM \\
\verb|\Rcone| & \Rcone \\
\verb|\abseta| & \abseta \\
\verb|\Ecm| & \Ecm \\
\verb|\rts| & \rts \\
\verb|\sqs| & \sqs \\
\verb|\Nevt| & \Nevt \\
\verb|\zvtx| & \zvtx \\
\verb|\dzero| & \dzero \\
\verb|\zzsth| & \zzsth \\
\verb|\RunOne| & \RunOne \\
\verb|\RunTwo| & \RunTwo \\
\verb|\RunThr| & \RunThr \\
\verb|\kt| & \kt \\
\verb|\antikt| & \antikt \\
\verb|\Antikt| & \Antikt \\
\verb|\pileup| & \pileup \\
\verb|\Pileup| & \Pileup \\
\verb|\btag| & \btag \\
\verb|\btagged| & \btagged \\
\verb|\bquark| & \bquark \\
\verb|\bquarks| & \bquarks \\
\verb|\bjet| & \bjet \\
\verb|\bjets| & \bjets \\
\verb|\mh| & \mh \\
\verb|\mW| & \mW \\
\verb|\mZ| & \mZ \\
\verb|\mH| & \mH \\
\verb|\ACERMC| & \ACERMC \\
\verb|\ALPGEN| & \ALPGEN \\
\verb|\AMCatNLO| & \AMCatNLO \\
\verb|\BLACKHAT| & \BLACKHAT \\
\verb|\CALCHEP| & \CALCHEP \\
\verb|\COLLIER| & \COLLIER \\
\verb|\COMPHEP| & \COMPHEP \\
\verb|\EVTGEN| & \EVTGEN \\
\verb|\FEYNRULES| & \FEYNRULES \\
\verb|\GGTOVV| & \GGTOVV \\
\verb|\GOSAM| & \GOSAM \\
\verb|\HATHOR| & \HATHOR \\
\verb|\Herwig| & \Herwig \\
\verb|\HERWIG| & \HERWIG \\
\verb|\HERWIGpp| & \HERWIGpp \\
\verb|\HRES| & \HRES \\
\verb|\JIMMY| & \JIMMY \\
\verb|\MADSPIN| & \MADSPIN \\
\verb|\MADGRAPH| & \MADGRAPH \\
\verb|\MGNLO| & \MGNLO \\
\verb|\MCatNLO| & \MCatNLO \\
\verb|\MCFM| & \MCFM \\
\verb|\METOP| & \METOP \\
\verb|\OPENLOOPS| & \OPENLOOPS \\
\verb|\POWHEG| & \POWHEG \\
\verb|\POWHEGBOX| & \POWHEGBOX \\
\verb|\POWHEGBOXRES| & \POWHEGBOXRES \\
\verb|\PHOTOS| & \PHOTOS \\
\verb|\PHOTOSpp| & \PHOTOSpp \\
\verb|\PROPHECY| & \PROPHECY \\
\verb|\PROTOS| & \PROTOS \\
\verb|\Pythia| & \Pythia \\
\verb|\PYTHIA| & \PYTHIA \\
\verb|\RECOLA| & \RECOLA \\
\verb|\Sherpa| & \Sherpa \\
\verb|\SHERPA| & \SHERPA \\
\verb|\TOPpp| & \TOPpp \\
\verb|\VBFNLO| & \VBFNLO \\
\verb|\MGNLOHER| & \MGNLOHER \\
\verb|\MGNLOPY| & \MGNLOPY \\
\verb|\MGHER| & \MGHER \\
\verb|\MGPY| & \MGPY \\
\verb|\POWHER| & \POWHER \\
\verb|\POWPY| & \POWPY \\
\verb|\SHERPABH| & \SHERPABH \\
\verb|\SHERPAOL| & \SHERPAOL \\
\verb|\ABM| & \ABM \\
\verb|\ABKM| & \ABKM \\
\verb|\CT| & \CT \\
\verb|\CTEQ| & \CTEQ \\
\verb|\GJR| & \GJR \\
\verb|\HERAPDF| & \HERAPDF \\
\verb|\LUXQED| & \LUXQED \\
\verb|\MSTW| & \MSTW \\
\verb|\MMHT| & \MMHT \\
\verb|\MSHT| & \MSHT \\
\verb|\NNPDF| & \NNPDF \\
\verb|\PDFforLHC| & \PDFforLHC \\
\verb|\AUET| & \AUET \\
\verb|\AZNLO| & \AZNLO \\
\verb|\FXFX| & \FXFX \\
\verb|\GEANT| & \GEANT \\
\verb|\MENLOPS| & \MENLOPS \\
\verb|\MEPSatLO| & \MEPSatLO \\
\verb|\MEPSatNLO| & \MEPSatNLO \\
\verb|\MINLO| & \MINLO \\
\verb|\Monash| & \Monash \\
\verb|\Perugia| & \Perugia \\
\verb|\Prospino| & \Prospino \\
\verb|\UEEE| & \UEEE \\
\verb|\LO| & \LO \\
\verb|\NLO| & \NLO \\
\verb|\NLL| & \NLL \\
\verb|\NNLO| & \NNLO \\
\verb|\muF| & \muF \\
\verb|\muQ| & \muQ \\
\verb|\muR| & \muR \\
\verb|\hdamp| & \hdamp \\
\verb|\NLOEWvirt| & \NLOEWvirt \\
\verb|\ra| & \ra \\
\verb|\la| & \la \\
\verb|\rarrow| & \rarrow \\
\verb|\larrow| & \larrow \\
\verb|\lapprox| & \lapprox \\
\verb|\rapprox| & \rapprox \\
\verb|\gam| & \gam \\
\verb|\stat| & \stat \\
\verb|\syst| & \syst \\
\verb|\radlength| & \radlength \\
\verb|\StoB| & \StoB \\
\verb|\dif| & \dif \\
\verb|\alphas| & \alphas \\
\verb|\NF| & \NF \\
\verb|\NC| & \NC \\
\verb|\CF| & \CF \\
\verb|\CA| & \CA \\
\verb|\TF| & \TF \\
\verb|\Lms| & \Lms \\
\verb|\Lmsfive| & \Lmsfive \\
\verb|\kperp| & \kperp \\
\verb|\Vcb| & \Vcb \\
\verb|\Vub| & \Vub \\
\verb|\Vtd| & \Vtd \\
\verb|\Vts| & \Vts \\
\verb|\Vtb| & \Vtb \\
\verb|\Vcs| & \Vcs \\
\verb|\Vud| & \Vud \\
\verb|\Vus| & \Vus \\
\verb|\Vcd| & \Vcd \\
\end{xtabular}


\noindent A length \Macro{figwidth} is defined that is \qty{2}{\cm} smaller than \Macro{textwidth}.

\noindent Monte Carlo generators and PDFs have an optional argument
that allows you to include the version, e.g.\
\verb|\PYTHIA[8]| to produce \PYTHIA[8] or\\
\verb|\PYTHIA[(v8.160)]| to produce \PYTHIA[(v8.160)].

\noindent A macro \Macro{pTX} is included that allows you to add an extra subscript and/or a superscript to \pT.
e.g.\\
\verb|\pTX| to produce \pTX;\\
\verb|\pTX[2]| to produce \pTX[2];\\
\verb|\pTX[][\text{had}]| to produce \pTX[][\text{had}];\\
\verb|\pTX[3][\text{had}]| to produce \pTX[3][\text{had}].

\noindent A generic macro \verb|\twomass| is defined, so that for example
\verb|\twomass{\mu}{\mu}| produces \twomass{\mu}{\mu} and \verb|\twomass{\mu}{e}| produces \twomass{\mu}{e}.

A macro \verb|\dk| is also defined which makes it easier to write down decay chains.
For example
\begin{verbatim}
\[\eqalign{a \to & b+c\\
   & \dk & e+f \\
   && \dk g+h}
\]
\end{verbatim}
produces
\[\eqalign{a \to & b+c\cr
   & \dk & e+f \cr
   && \dk g+h}
\]
Note that \Macro{eqalign} is also redefined in this package so that \Macro{dk} works.

The following macro names have been changed:\\
\verb|\ptsq| \(\to\) \verb|\pTsq|.


\newpage
%-------------------------------------------------------------------------------
\section{\File{atlasxref.sty}}

Turn on including these definitions with the option \Option{xref=true} and off with the option \Option{xref=false}.

%
% +--------------------------------------------------------------------+
% |                                                                    |
% |  Useful abbreviations in text for cross references                 |
% |  The abbreviations can be adjusted depending on the journal        |
% |                                                                    |
% +--------------------------------------------------------------------+
%
\newcommand*{\Eqn}[1]{Eq.~#1\xspace}
\newcommand*{\Fig}[1]{Fig.~#1\xspace}
\newcommand*{\Ref}[1]{Ref.~#1\xspace}
\newcommand*{\Sect}[1]{Sect.~#1\xspace}
\newcommand*{\Tab}[1]{Table~#1\xspace}
\newcommand*{\Eqns}[2]{Eqs.~#1--#2\xspace}
\newcommand*{\Figs}[2]{Figs.~#1--#2\xspace}
\newcommand*{\Refs}[2]{Refs.~#1--#2\xspace}
\newcommand*{\Sects}[2]{Sects.~#1--#2\xspace}
\newcommand*{\Tabs}[2]{Tables~#1--#2\xspace}


\noindent The following macros with arguments are also defined:
\begin{xtabular}{ll}
\verb|\App{1}|  & \App{1}\\
\verb|\Eqn{1}|  & \Eqn{1}\\
\verb|\Fig{1}|  & \Fig{1}\\
\verb|\Refn{1}|  & \Refn{1}\\
\verb|\Sect{1}| & \Sect{1}\\
\verb|\Tab{1}|  & \Tab{1}\\
\verb|\Apps{1}{4}| & \Apps{1}{4} \\
\verb|\Eqns{1}{4}| & \Eqns{1}{4} \\
\verb|\Figs{1}{4}| & \Figs{1}{4} \\
\verb|\Refns{1}{4}| & \Refns{1}{4} \\
\verb|\Sects{1}{4}| & \Sects{1}{4} \\
\verb|\Tabs{1}{4}| & \Tabs{1}{4} \\
\verb|\Apprange{1}{4}| & \Apprange{1}{4} \\
\verb|\Eqnrange{1}{4}| & \Eqnrange{1}{4} \\
\verb|\Figrange{1}{4}| & \Figrange{1}{4} \\
\verb|\Refrange{1}{4}| & \Refrange{1}{4} \\
\verb|\Sectrange{1}{4}| & \Sectrange{1}{4} \\
\verb|\Tabrange{1}{4}| & \Tabrange{1}{4}
\end{xtabular}

The idea is that you can adapt these definitions according to your own preferences (or those of a journal).
Note that the macros \Macro{Ref} and \Macro{Refs} were renamed to \Macro{Refn} and \Macro{Refns}
in \Package{atlaslatex} 08-00-00, as \Macro{Ref} is now defined in the \Package{hyperref} package.


\newpage
%-------------------------------------------------------------------------------
\section{\File{atlasbsm.sty}}

Turn on including these definitions with the option \Option{BSM} and off with the option \Option{BSM=false}.

The macro \Macro{susy} simply puts a tilde (\(\tilde{\ }\)) over its argument,
e.g.\ \verb|\susy{q}| produces \susy{q}.

For \susy{q}, \susy{t}, \susy{b}, \slepton, \sel, \smu and
\stau, L and R states are defined; for stop, sbottom and stau also the
light (1) and heavy (2) states.
There are four neutralinos and two charginos defined,
the index number unfortunately needs to be written out completely.
For the charginos the last letter(s) indicate(s) the charge:
\enquote{p} for \(+\), \enquote{m} for \(-\), and \enquote{pm} for \(\pm\).

\begin{supertabular}{ll}
\verb|\Azero| & \Azero \\
\verb|\hzero| & \hzero \\
\verb|\Hzero| & \Hzero \\
\verb|\Hboson| & \Hboson \\
\verb|\Hplus| & \Hplus \\
\verb|\Hminus| & \Hminus \\
\verb|\Hpm| & \Hpm \\
\verb|\Hmp| & \Hmp \\
\verb|\ggino| & \ggino \\
\verb|\chinop| & \chinop \\
\verb|\chinom| & \chinom \\
\verb|\chinopm| & \chinopm \\
\verb|\chinomp| & \chinomp \\
\verb|\chinoonep| & \chinoonep \\
\verb|\chinoonem| & \chinoonem \\
\verb|\chinoonepm| & \chinoonepm \\
\verb|\chinotwop| & \chinotwop \\
\verb|\chinotwom| & \chinotwom \\
\verb|\chinotwopm| & \chinotwopm \\
\verb|\nino| & \nino \\
\verb|\ninoone| & \ninoone \\
\verb|\ninotwo| & \ninotwo \\
\verb|\ninothree| & \ninothree \\
\verb|\ninofour| & \ninofour \\
\verb|\gravino| & \gravino \\
\verb|\Zprime| & \Zprime \\
\verb|\Zstar| & \Zstar \\
\verb|\squark| & \squark \\
\verb|\squarkL| & \squarkL \\
\verb|\squarkR| & \squarkR \\
\verb|\gluino| & \gluino \\
\verb|\stop| & \stop \\
\verb|\stopone| & \stopone \\
\verb|\stoptwo| & \stoptwo \\
\verb|\stopL| & \stopL \\
\verb|\stopR| & \stopR \\
\verb|\sbottom| & \sbottom \\
\verb|\sbottomone| & \sbottomone \\
\verb|\sbottomtwo| & \sbottomtwo \\
\verb|\sbottomL| & \sbottomL \\
\verb|\sbottomR| & \sbottomR \\
\verb|\slepton| & \slepton \\
\verb|\sleptonL| & \sleptonL \\
\verb|\sleptonR| & \sleptonR \\
\verb|\sel| & \sel \\
\verb|\selL| & \selL \\
\verb|\selR| & \selR \\
\verb|\smu| & \smu \\
\verb|\smuL| & \smuL \\
\verb|\smuR| & \smuR \\
\verb|\stau| & \stau \\
\verb|\stauL| & \stauL \\
\verb|\stauR| & \stauR \\
\verb|\stauone| & \stauone \\
\verb|\stautwo| & \stautwo \\
\verb|\snu| & \snu \\
\end{supertabular}



\newpage
%-------------------------------------------------------------------------------
\section{\File{atlasheavyion.sty}}

Turn on including these definitions with the option \Option{hion=true} and off with the option \Option{hion=false}.
The heavy ion definitions use the package \Package{mhchem} to help with the formatting of chemical elements.
This package is included by \File{atlasheavyion.sty}.

%-------------------------------------------------------------------------------
% Collection of heavy iondefinitions, typically not included in the other style files.
% Include with hion option in atlasphysics.sty.
% Not included by default.
% Also needs atlasmisc.sty (option misc).
% Compiled by Sasha Milov.
% Adapted for atlaslatex by Ian Brock.
%
% Note that this file can be overwritten when atlaslatex is updated.
%
% Copyright (C) 2002-2020 CERN for the benefit of the ATLAS collaboration
%-------------------------------------------------------------------------------

% Package used for chemical elements
\RequirePackage[version=3]{mhchem}

% +------------------------------------+
% |                                    |
% |  System related notations          |
% |                                    |
% +------------------------------------+
\newcommand*{\NucNuc}{\ce{A}+\ce{A}\xspace}

\newcommand*{\nn}{\ensuremath{nn}\xspace}
%\newcommand*{\pp}{\ensuremath{pp}\xspace}
\newcommand*{\pn}{\ensuremath{pn}\xspace}
\newcommand*{\np}{\ensuremath{np}\xspace}

\newcommand*{\PbPb}{\ce{Pb}+\ce{Pb}\xspace}
\newcommand*{\AuAu}{\ce{Au}+\ce{Au}\xspace}
\newcommand*{\CuCu}{\ce{Cu}+\ce{Cu}\xspace}

\providecommand*{\pA}{\ensuremath{p}+\ce{A}\xspace}
\newcommand*{\pNuc}{\pA\xspace}
\newcommand*{\pdA}{\ensuremath{p}/\ensuremath{d}+\ce{A}\xspace}
\newcommand*{\dAu}{\ensuremath{d}+\ce{Au}\xspace}
\newcommand*{\pPb}{\ensuremath{p}+\ce{Pb}\xspace}

% +--------------------------------------+
% |                                      |
% |  Centrality related notations        |
% |                                      |
% +--------------------------------------+
\newcommand*{\Npart}{\ensuremath{N_{\text{part}}}\xspace}
\newcommand*{\avgNpart}{\ensuremath{\langle\Npart\rangle}\xspace}

\newcommand*{\Ncoll}{\ensuremath{N_{\text{coll}}}\xspace}
\newcommand*{\avgNcoll}{\ensuremath{\langle\Ncoll\rangle}\xspace}

\newcommand*{\TA}{\ensuremath{T_{\ce{A}}}\xspace}
\newcommand*{\avgTA}{\ensuremath{\langle\TA\rangle}\xspace}

\newcommand*{\TPb}{\ensuremath{T_{\ce{Pb}}}\xspace}
\newcommand*{\avgTPb}{\ensuremath{\langle\TPb\rangle}\xspace}

\newcommand*{\TAA}{\ensuremath{T_{\text{AA}}}\xspace}
\newcommand*{\avgTAA}{\ensuremath{\langle\TAA\rangle}\xspace}

\newcommand{\TAB}{\ensuremath{T_{\text{AB}}}\xspace}
\newcommand{\avgTAB}{\ensuremath{\langle\TAB\rangle}\xspace}

\newcommand*{\TpPb}{\ensuremath{T_{p\ce{Pb}}}\xspace}
\newcommand*{\avgTpPb}{\ensuremath{\langle\TpPb\rangle}\xspace}

\newcommand{\Gl}{Glauber\xspace}
\newcommand{\GG}{Glauber--Gribov\xspace}

% +--------------------------------------+
% |                                      |
% |  C.M. energy related notations       |
% |                                      |
% +--------------------------------------+
\newcommand*{\sqn}{\ensuremath{\sqrt{s_{_\text{NN}}}}\xspace}
\newcommand{\lns}{\ensuremath{\ln(\kern -0.2em\sqrt{s})}\xspace}

% +--------------------------------------+
% |                                      |
% |  Some useful parameters              |
% |                                      |
% +--------------------------------------+
\newcommand*{\sumETPb}{\ensuremath{\Sigma E_{\text{T}}^{\ce{Pb}}}\xspace}
\newcommand*{\sumETp}{\ensuremath{\Sigma E_{\text{T}}^{p}}\xspace}
\newcommand*{\sumETA}{\ensuremath{\Sigma E_{\text{T}}^{\ce{A}}}\xspace}

% +--------------------------------------+
% |                                      |
% |  Some useful constructions           |
% |                                      |
% +--------------------------------------+
\newcommand*{\RAA}{\ensuremath{R_{\ce{AA}}}\xspace}
\newcommand*{\RCP}{\ensuremath{R_{\text{CP}}}\xspace}
\newcommand*{\RpA}{\ensuremath{R_{p\ce{A}}}\xspace}

\newcommand*{\RpPb}{\ensuremath{R_{p\ce{Pb}}}\xspace}

% Different differential symbols in American and British English
\iflanguage{USenglish}{%
  \providecommand*{\dif}{\ensuremath{d}}
}{%
  \providecommand*{\dif}{\ensuremath{\mathrm{d}}}
}
\newcommand*{\dNchdeta}{\ensuremath{\dif N_{\text{ch}}/\dif \eta}\xspace}
\newcommand*{\dNevtdET}{\ensuremath{\dif N_{\text{evt}}/\dif \ET}\xspace}

% +--------------------------------------+
% |                                      |
% |  Framework transforms                |
% |                                      |
% +--------------------------------------+
\newcommand*{\ystar}{\ensuremath{y^{*}}\xspace}
\newcommand*{\ycms}{\ensuremath{y_\text{CM}}\xspace}
\newcommand*{\ygappb}{\ensuremath{\Delta \eta_{\text{gap}}^{\ce{Pb}}}\xspace}
\newcommand*{\ygapp}{\ensuremath{\Delta \eta_{\text{gap}}^{p}}\xspace}
\newcommand*{\fgap}{\ensuremath{f_{\text{gap}}}\xspace}


%The following symbols were removed or modified with respect to the original submission
%\input{atlasheavyion-mod}


\newpage
%-------------------------------------------------------------------------------
\section{\File{atlasjetetmiss.sty}}

Turn on including these definitions with the option \Option{jetetmiss=true} and off with the option \Option{jetetmiss=false}.

%-------------------------------------------------------------------------------
% Useful definitions from the Jet/Etmiss group.
% Include with jetetmiss option in atlasphysics.sty.
% Not included by default.
% Also needs atlasmisc.sty (option misc).
% Compiled by David Miller.
% Adapted for atlaslatex by Ian Brock.
%
% Note that this file can be overwritten when atlaslatex is updated.
%
% Copyright (C) 2002-2023 CERN for the benefit of the ATLAS collaboration.
%-------------------------------------------------------------------------------

%%% Topoclusters
\newcommand*{\topo}{topo-cluster}
\newcommand*{\Topo}{Topo-cluster}
\newcommand*{\topos}{topo-clusters}
\newcommand*{\Topos}{Topo-clusters}


%%% Standard terms
\newcommand*{\insitu}{in~situ\xspace}
\newcommand*{\Insitu}{In~situ\xspace}


%%% Event generator
\newcommand*{\LS}{\ensuremath{\text{LS}}\xspace}
\newcommand*{\NLOjet}{\textsc{NLOJet}\texttt{++}\xspace}
\newcommand*{\Fastjet}{\textsc{FastJet}\xspace}
\newcommand*{\TwoToTwo}{\ensuremath{2\rightarrow2}\xspace}


%%% Jets
\newcommand*{\largeR}{large-\ensuremath{R}\xspace}
\newcommand*{\LargeR}{Large-\ensuremath{R}\xspace}
%\newcommand*{\kt}{\ensuremath{k_{t}}\xspace}
%\newcommand*{\antikt}{anti-\kt}
%\newcommand*{\Antikt}{Anti-\kt}
\newcommand*{\akt}{\antikt}
\newcommand*{\Akt}{\Antikt}
\newcommand*{\AKT}{\akt}
\newcommand*{\AKTFat}{\antikt, \ensuremath{R=1.0}\xspace}
\newcommand*{\AKTPrune}{\antikt, \ensuremath{R=1.0} (pruned)\xspace}
\newcommand*{\AKTFilt}{\antikt, \ensuremath{R=1.0} (filtered)\xspace}
\newcommand*{\KTSix}{\kt, \ensuremath{R=0.6}\xspace}
\newcommand*{\ca}{Cambridge--Aachen\xspace}
\newcommand*{\CamKt}{\ensuremath{\text{C/A}}\xspace}
\newcommand*{\CASix}{\CamKt, \ensuremath{R=0.6}\xspace}
\newcommand*{\CAFat}{\CamKt, \ensuremath{R=1.2}\xspace}
\newcommand*{\CAPrune}{\CamKt, \ensuremath{R=1.2} (pruned)\xspace}
\newcommand*{\CAFilt}{\CamKt, \ensuremath{R=1.2} (filtered)\xspace}

%%% HEPTopTagger
\newcommand*{\htt}{HEPTopTagger\xspace}
\newcommand*{\mcut}{\ensuremath{m_{\text{cut}}}\xspace}
\newcommand*{\Nfilt}{\ensuremath{N_{\text{filt}}}\xspace}

%%% Groomed jets
\newcommand*{\Rfilt}{\ensuremath{R_{\text{filt}}}\xspace}
\newcommand*{\ymin}{\ensuremath{y_{\min}}\xspace}
\newcommand*{\fcut}{\ensuremath{f_{\text{cut}}}\xspace}
\newcommand*{\Rsub}{\ensuremath{R_{\text{sub}}}\xspace}
%\newcommand*{\drsub}{\ensuremath{\Rsub}\xspace}
\newcommand*{\mufrac}{\ensuremath{\mu_{\text{frac}}}\xspace}
\newcommand*{\Rcut}{\ensuremath{R_{\text{cut}}}\xspace}
\newcommand*{\zcut}{\ensuremath{z_{\text{cut}}}\xspace}

% Jet calibration (GSC)
\newcommand*{\ftile}{\ensuremath{f_{\text{Tile}0}}\xspace}
\newcommand*{\fem}{\ensuremath{f_{\text{LAr}3}}\xspace}
\newcommand*{\fpres}{\ensuremath{f_{\text{PS}}}\xspace}
\newcommand*{\fhec}{\ensuremath{f_{\text{HEC}0}}\xspace}
\newcommand*{\ffcal}{\ensuremath{f_{\text{FCal}1}}\xspace}

\newcommand*{\central}{\ensuremath{0.3 \leq |\eta| < 0.8}\xspace}
\newcommand*{\ecap}{\ensuremath{2.1 \leq |\eta| < 2.8}\xspace}
\newcommand*{\forward}{\ensuremath{3.6 \leq |\eta| < 4.5}\xspace}


%%% Pileup
\newcommand*{\avg}[1]{\ensuremath{\langle #1 \rangle}\xspace} % for e.g. <\mu>
\newcommand*{\Npv}{\ensuremath{N_{\text{PV}}}\xspace}
\newcommand*{\Nref}{\ensuremath{\Npv^{\text{ref}}}\xspace}
\newcommand*{\Navg}{\ensuremath{\avg{\Npv}}\xspace}
\newcommand*{\avgmu}{\ensuremath{\avg{\mu}}\xspace}


%%% JES
\newcommand*{\JES}{\text{JES}\xspace}
\newcommand*{\JMS}{\text{JMS}\xspace}
\newcommand*{\EMJES}{\text{EM+JES}\xspace}
\newcommand*{\GCWJES}{\text{GCW+JES}\xspace}
\newcommand*{\LCWJES}{\text{LCW+JES}\xspace}
\newcommand*{\EM}{\text{EM}\xspace}
\newcommand*{\GCW}{\text{GCW}\xspace}
\newcommand*{\LCW}{\text{LCW}\xspace}
\newcommand*{\GSL}{\text{GSL}\xspace}
\newcommand*{\GS}{\text{GS}\xspace}
\newcommand*{\MTF}{\text{MTF}\xspace}
\newcommand*{\MPF}{\text{MPF}\xspace}


%%% Jet multiplicity
\newcommand*{\Njet}{\ensuremath{N_\text{jet}}\xspace}
\newcommand*{\njet}{\Njet}


%%% Kinematics
\newcommand*{\ETjet}{\ensuremath{\ET^{\text{jet}}}\xspace}
\newcommand*{\etjet}{\ETjet}
\newcommand*{\pTavg}{\ensuremath{\pT^\text{avg}}\xspace}
\newcommand*{\ptavg}{\pTavg}
\newcommand*{\pTjet}{\ensuremath{\pT^\text{jet}}\xspace}
\newcommand*{\ptjet}{\pTjet}
\newcommand*{\pTcorr}{\ensuremath{\pT^\text{corr}}\xspace}
\newcommand*{\ptcorr}{\pTcorr}
\newcommand*{\pTjeti}{\ensuremath{p_{\text{T}, i}^\text{jet}}\xspace}
\newcommand*{\ptjeti}{\pTjeti}
\newcommand*{\pTrecoil}{\ensuremath{\pT^\text{recoil}}\xspace}
\newcommand*{\ptrecoil}{\pTrecoil}
\newcommand*{\pTleading}{\ensuremath{\pT^{\text{leading}}}\xspace}
\newcommand*{\ptleading}{\pTleading}
\newcommand*{\pTjetEM}{\ensuremath{p_{\text{T, \EM}}^{\text{jet}}}\xspace}
\newcommand*{\ptjetEM}{\pTjetEM}
\newcommand*{\pThat}{\ensuremath{\hat{p}_\text{T}}\xspace}
\newcommand*{\pthat}{\pThat}
\newcommand*{\pTprobe}{\ensuremath{\pT^{\text{probe}}}\xspace}
\newcommand*{\ptprobe}{\pTprobe}
\newcommand*{\pTref}{\ensuremath{\pT^{\text{ref}}}\xspace}
\newcommand*{\ptref}{\pTref}
\newcommand*{\pToff}{\ensuremath{\mathcal{O}}\xspace}
\newcommand*{\ptoff}{\pToff}
\newcommand*{\pToffjet}{\ensuremath{\pToff^{\text{jet}}}\xspace}
\newcommand*{\ptoffjet}{\pToffjet}
\newcommand*{\pTZ}{\ensuremath{\pT^{Z}}\xspace}
\newcommand*{\ptZ}{\pTZ}

% Use siunitx definitions if available. Otherwise assume \GeV is defined.
\@ifpackageloaded{siunitx}{%
  \newcommand*{\pTRange}[2]{\ensuremath{{#1} \leq \pTjet < \qty{#2}{\GeV}}\xspace}
  \newcommand*{\ptRange}[2]{\ensuremath{{#1} \leq \ptjet < \qty{#2}{\GeV}}\xspace}
}{%
  \newcommand*{\pTRange}[2]{\ensuremath{{#1} \leq \pTjet < {#2} \GeV}\xspace}
  \newcommand*{\ptRange}[2]{\ensuremath{{#1} \leq \ptjet < {#2} \GeV}\xspace}
}
\newcommand*{\pTtrue}{\ensuremath{\pT^{\text{true}}}\xspace}
\newcommand*{\pttrue}{\pTtrue}
\newcommand*{\pTtruth}{\ensuremath{\pT^{\text{true}}}\xspace}
\newcommand*{\pttruth}{\pTtruth}
\newcommand*{\pTreco}{\ensuremath{\pT^{\text{reco}}}\xspace}
\newcommand*{\ptreco}{\pTreco}
\newcommand*{\pTtrk}{\ensuremath{\pT^{\text{track}}}\xspace} 
\newcommand*{\pttrk}{\pTtrk}
\newcommand*{\ptrk}{\ensuremath{p^{\text{track}}}\xspace} 
\newcommand*{\pTtrkjet}{\ensuremath{\pT^{\text{track jet}}}\xspace}
\newcommand*{\pttrkjet}{\pTtrkjet}
\newcommand*{\ntrk}{\ensuremath{n_{\text{track}}}\xspace}
\newcommand*{\EoverP}{\ensuremath{E/p}\xspace}
\newcommand*{\Etrue}{\ensuremath{E^{\text{true}}}\xspace}
\newcommand*{\Etruth}{\ensuremath{E^{\text{true}}}\xspace}
\newcommand*{\Ecalo}{\ensuremath{E^{\text{jet}}}\xspace}
\newcommand*{\EcaloEM}{\ensuremath{E^{\text{jet}}_{\EM}}\xspace}
\newcommand*{\asym}{\ensuremath{\mathcal{A}}\xspace}
\newcommand*{\Response}{\ensuremath{\mathcal{R}}\xspace} 
\newcommand*{\Rcalo}{\ensuremath{\Response^{\text{jet}}}\xspace} 
\newcommand*{\Rcalom}{\ensuremath{\Response^{\text{jet}}_{m}}\xspace} 
\newcommand*{\RcaloEM}{\ensuremath{\Rcalo_{\EM}}\xspace}
\newcommand*{\RMPF}{\ensuremath{\Response_{\MPF}}\xspace}
\newcommand*{\EcaloCALIB}{\ensuremath{E^{\text{jet}}}\xspace}
\newcommand*{\RcaloCALIB}{\ensuremath{\mathcal{R}^{\text{jet}}}\xspace} 
\newcommand*{\EcaloEMJES}{\ensuremath{E^{\text{jet}}_{\EMJES}}\xspace}
\newcommand*{\RcaloEMJES}{\ensuremath{\mathcal{R}^{\text{jet}}_{\EMJES}}\xspace}
\newcommand*{\EcaloGCWJES}{\ensuremath{E^{\text{jet}}_{\GCWJES}}\xspace}
\newcommand*{\RcaloGCWJES}{\ensuremath{\mathcal{R}^{\text{jet}}_{\GCWJES}}\xspace}
\newcommand*{\EcaloLCWJES}{\ensuremath{E^{\text{jet}}_{\LCWJES}}\xspace}
\newcommand*{\RcaloLCWJES}{\ensuremath{\mathcal{R}^{\text{jet}}_{\LCWJES}}\xspace}
\newcommand*{\Rtrack}{\ensuremath{\mathcal{R}^{\text{track jet}}}\xspace}
\newcommand*{\rtrk}{\ensuremath{r_{\text{trk}}}\xspace}
\newcommand*{\Rtrk}{\ensuremath{R_{\text{trk}}}\xspace}
\newcommand*{\rtrackjet}{\ensuremath{r^{\text{calo / track jet}}}\xspace}
\newcommand*{\rtrackjetiso}{\ensuremath{\rtrackjet_{\text{iso}}}\xspace}
\newcommand*{\rtrackjetnoniso}{\ensuremath{\rtrackjet_{\text{non-iso}}}\xspace}
\newcommand*{\rtrackjetisoratio}{\ensuremath{\rtrackjet_{\text{non-iso/iso}}}\xspace}
\newcommand*{\gammajet}{\ensuremath{\gamma\text{+jet}}\xspace}
\newcommand*{\deltaphijetgamma}{\ensuremath{\Delta \phi_{\text{jet--}\gamma}}\xspace}


%%% Spatial quantities
\newcommand*{\rapjet}{\ensuremath{y}\xspace}
\newcommand*{\etajet}{\ensuremath{\eta}\xspace}
\newcommand*{\phijet}{\ensuremath{\phi}\xspace}
\newcommand*{\etadet}{\ensuremath{\eta_{\text{det}}}\xspace}
\newcommand*{\etatrk}{\ensuremath{\eta^{\text{track}}}\xspace}
\newcommand*{\etaRange}[2]{\ensuremath{{#1} \leq |\eta| < {#2}}\xspace}
\newcommand*{\AetaRange}[1]{\ensuremath{|\eta| < {#1}}\xspace}

\newcommand*{\Rmin}{\ensuremath{R_{\min}}\xspace}
\newcommand*{\DeltaR}{\ensuremath{\Delta R}\xspace}
\newcommand*{\DetaDphi}{\ensuremath{\sqrt{(\Delta\eta)^{2}+(\Delta\phi)^{2}}}\xspace}
\newcommand*{\Deta}{\ensuremath{|\Delta\eta|}\xspace}
\newcommand*{\Drap}{\ensuremath{|\Delta y|}\xspace}
\newcommand*{\DetaOneTwo}{\ensuremath{|\Delta\eta(\text{jet}1, \text{jet}2)|}\xspace}
\newcommand*{\DyDphi}{\ensuremath{\sqrt{(\Delta y)^{2} + (\Delta\phi)^{2}}}\xspace}
\newcommand*{\DeltaRdef}{\ensuremath{\DeltaR = \DetaDphi}\xspace}
\newcommand*{\DeltaRydef}{\ensuremath{\DeltaR = \DyDphi}\xspace}
\newcommand*{\DeltaRtrk}{\ensuremath{\Delta R(\text{trk}_1,\text{trk}_2)}\xspace}


%%% Pile-up correction and mitigation related
\newcommand*{\JVF}{\ensuremath{\text{JVF}}\xspace}
\newcommand*{\cJVF}{\ensuremath{\text{corrJVF}}\xspace}
\newcommand*{\RpT}{\ensuremath{R_\text{\pT}}\xspace}
\newcommand*{\JVT}{\ensuremath{\text{JVT}}\xspace}
\newcommand*{\ghostpt}{\ensuremath{g_{t}}\xspace}
\newcommand*{\ghostptavg}{\ensuremath{\avg{\ghostpt}}\xspace}
\newcommand*{\ghostfm}{\ensuremath{g_{\mu}}\xspace}
\newcommand*{\ghostfmi}{\ensuremath{g_{\mu,i}}\xspace}
\newcommand*{\ghostdensity}{\ensuremath{\nu_{g}}\xspace}
\newcommand*{\ghostrho}{\ensuremath{\ghostdensity\ghostptavg}\xspace}
\newcommand*{\Aghost}{\ensuremath{A_{g}}\xspace}
\newcommand*{\Amu}{\ensuremath{A_{\mu}}\xspace}
\newcommand*{\Amui}{\ensuremath{A_{\mu, i}}\xspace}
\newcommand*{\jetarea}{\ensuremath{A^{\text{jet}}}\xspace}
\newcommand*{\jetareafm}{\ensuremath{A^{\text{jet}}_{\mu}}\xspace}
\newcommand*{\jetareai}{\ensuremath{\jetarea_{i}}\xspace}
\newcommand*{\Rkt}{\ensuremath{R_{\kt}}\xspace}
\newcommand*{\pTmuslope}{\ensuremath{\partial\langle\Delta \pt\rangle/\partial\avgmu}\xspace}
\newcommand*{\ptmuslope}{\pTmuslope}
\newcommand*{\pTnpvslope}{\ensuremath{\partial\langle\Delta \pt\rangle/\partial\Npv}\xspace}
\newcommand*{\ptnpvslope}{\pTnpvslope}
\newcommand*{\pTmuunc}{\ensuremath{\Delta\left(\ptmuslope\right)}\xspace}
\newcommand*{\ptmuunc}{\pTmuunc}
\newcommand*{\pTnpvunc}{\ensuremath{\Delta\left(\ptnpvslope\right)}\xspace}
\newcommand*{\ptnpvunc}{\pTnpvunc}

\newcommand*{\sumPt}{\ensuremath{\sum \vec{p}_{\text{T}}}\xspace}
\newcommand*{\sumpt}{\sumPt}
%\newcommand*{\sumPtTr}{\ensuremath{|\sum \vec{p}_{\text{T}}^{\text{ track}}|}\xspace}
%\newcommand*{\sumpt}{\sumPt}
\newcommand*{\sumpTtrk}{\ensuremath{\sum\pt^\text{track}}\xspace}
\newcommand*{\sumpttrk}{\sumpTtrk}
\newcommand*{\nPUtrk}{\ensuremath{n_\text{trk}^\text{PU}}\xspace}

%%% Jet structure observables
\newcommand*{\mjet}{\ensuremath{m^{\text{jet}}}\xspace}
\newcommand*{\mlead}{\ensuremath{\mjet_{1}}\xspace}
\newcommand*{\mleadavg}{\ensuremath{\langle\mlead\rangle}\xspace}
\newcommand*{\Mjet}{\mjet}
\newcommand*{\massjet}{\mjet}
\newcommand*{\masscorr}{\ensuremath{m^{\text{corr}}}}
\newcommand*{\mthresh}{\ensuremath{M_{\text{threshold}}}\xspace}
\newcommand*{\mjetavg}{\ensuremath{\langle\mjet\rangle}\xspace}
\newcommand*{\masstrkjet}{\ensuremath{m^{\text{track jet}}}\xspace}
\newcommand*{\width}{\ensuremath{w}\xspace}
\newcommand*{\wcalo}{\ensuremath{\width^{\text{calo}}}\xspace}
\newcommand*{\wtrk}{\ensuremath{\width^{\text{track}}}\xspace}
\newcommand*{\shapeV}{\ensuremath{\mathcal{V}}\xspace}
\newcommand*{\pTsubjet}{\ensuremath{\pT^\text{subjet}}\xspace}
\newcommand*{\ptsubjet}{\pTsubjet}
\newcommand*{\sjone}{\ensuremath{j_{1}}\xspace}
\newcommand*{\sjtwo}{\ensuremath{j_{2}}\xspace}
\newcommand*{\msubjone}{\ensuremath{m^{\sjone}}\xspace}
\newcommand*{\msubjtwo}{\ensuremath{m^{\sjtwo}}\xspace}
\newcommand*{\pTsubji}{\ensuremath{\pT^{i}}\xspace}
\newcommand*{\ptsubji}{\pTsubji}
\newcommand*{\pTsubjone}{\ensuremath{\pT^{\sjone}}\xspace}
\newcommand*{\ptsubjone}{\pTsubjone}
\newcommand*{\pTsubjtwo}{\ensuremath{\pT^{\sjtwo}}\xspace}
\newcommand*{\ptsubjtwo}{\pTsubjtwo}
\newcommand*{\Rsubjets}{\ensuremath{R_{\sjone,\sjtwo}}\xspace}
\newcommand*{\DRsubjets}{\ensuremath{\Delta\Rsubjets}\xspace}
\newcommand*{\yij}{\ensuremath{y_{ij}}\xspace}
\newcommand*{\dcut}{\ensuremath{d_{\text{cut}}}\xspace}
\newcommand*{\dmin}{\ensuremath{d_{\min}}\xspace}
\newcommand*{\dij}{\ensuremath{d_{ij}}\xspace}
\newcommand*{\Dij}{\ensuremath{\sqrt{\dij}}\xspace}
\newcommand*{\Donetwo}{\ensuremath{\sqrt{d_{12}}}\xspace}
\newcommand*{\Dtwothr}{\ensuremath{\sqrt{d_{23}}}\xspace}
\newcommand*{\yonetwo}{\ensuremath{y_{1}}\xspace}
\newcommand*{\ytwothr}{\ensuremath{y_{2}}\xspace}
\newcommand*{\yonetwoDef}{\ensuremath{\yonetwo=\Donetwo/\mjet}\xspace}
\newcommand*{\ytwothrDef}{\ensuremath{\ytwothr=\Dtwothr/\mjet}\xspace}
\newcommand*{\xj}{\ensuremath{x_{J}}\xspace}
\newcommand*{\jetFunc}{\ensuremath{J^{(eik),c} (\mjet,\pT,R)}\xspace}
\newcommand*{\tauone}{\ensuremath{\tau_1}\xspace}
\newcommand*{\tautwo}{\ensuremath{\tau_2}\xspace}
\newcommand*{\tauthr}{\ensuremath{\tau_3}\xspace}
\newcommand*{\tauN}{\ensuremath{\tau_N}\xspace}
\newcommand*{\tautwoone}{\ensuremath{\tau_{21}}\xspace}
\newcommand*{\tauthrtwo}{\ensuremath{\tau_{32}}\xspace}
\newcommand*{\dip}{\ensuremath{\mathcal{D}}\xspace}
\newcommand*{\diponetwo}{\ensuremath{\dip_{12}}\xspace}
\newcommand*{\diptwothr}{\ensuremath{\dip_{23}}\xspace}
\newcommand*{\diponethr}{\ensuremath{\dip_{13}}\xspace}
\newcommand*{\mtaSup}{\ensuremath{m^{\text{TA}}}\xspace}
\newcommand*{\mcalo}{\ensuremath{m^{\text{calo}}}\xspace}
\newcommand*{\mcomb}{\ensuremath{m^{\text{comb}}}\xspace}
\newcommand*{\ECFOne}{\ensuremath{ECF_1}\xspace}
\newcommand*{\ECFTwo}{\ensuremath{ECF_2}\xspace}
\newcommand*{\ECFThr}{\ensuremath{ECF_3}\xspace}
\newcommand*{\ECFThrNorm}{\ensuremath{e_3}\xspace}
\newcommand*{\DTwo}{\ensuremath{D_{2}}\xspace}
\newcommand*{\CTwo}{\ensuremath{C_{2}}\xspace}
\newcommand*{\FoxWolfRatio}{\ensuremath{R^{\text{FW}}_{2}}\xspace}
\newcommand*{\PlanarFlow}{\ensuremath{\mathcal{P}}\xspace}
\newcommand*{\Angularity}{\ensuremath{a_{3}}\xspace}
\newcommand*{\Aplanarity}{\ensuremath{A}\xspace}
\newcommand*{\KtDR}{\ensuremath{KtDR}\xspace}
\newcommand*{\Qw}{\ensuremath{Q_w}\xspace}
\newcommand*{\NConst}{\ensuremath{N^{\text{const}}}\xspace}

%-------------------------------------------------------------------------------
% The following definitions were in a first version of this style file
% that I received. They have either been moved into another file or
% are not really generic enough, or do not use a units package.
% They are kept for reference.
%-------------------------------------------------------------------------------

%%% b-tagging
%\newcommand*{\btag}{\ensuremath{b\text{-tagging}}\xspace}
%\newcommand*{\btagged}{\ensuremath{b\text{-tagged}}\xspace}
%\newcommand*{\bquark}{\ensuremath{b\text{-quark}}\xspace}
%\newcommand*{\bquarks}{\ensuremath{b\text{-quarks}}\xspace}
%\newcommand*{\bjet}{\ensuremath{b\text{-jet}}\xspace}
%\newcommand*{\bjets}{\ensuremath{b\text{-jets}}\xspace}


%%% Processes
%\newcommand*{\Zbb}{\ensuremath{Z \rightarrow \bbbar}\xspace}
%\newcommand*{\tWb}{\ensuremath{t \rightarrow Wb}\xspace}
%\newcommand*{\Wqqbar}{\ensuremath{W \rightarrow \qqbar}\xspace}
%%\newcommand*{\enu}{\ensuremath{e\nu}\xspace}  
%\newcommand*{\Zmumu}{\ensuremath{Z \rightarrow \mumu}\xspace}
%\newcommand*{\Wlnu}{\ensuremath{W \rightarrow \lnu}\xspace}
%\newcommand*{\Wenu}{\ensuremath{W \rightarrow \enu}\xspace}
%%\newcommand*{\munu}{\ensuremath{\mu\nu}\xspace}
%\newcommand*{\Wmunu}{\ensuremath{W \rightarrow \munu}\xspace}
%\newcommand*{\Wjets}{\ensuremath{\Wboson\text{+jets}}\xspace}
%\newcommand*{\Zjets}{\ensuremath{\Zboson\text{+jets}}\xspace}

%\newcommand*{\W}{\Wboson\xspace}
%\newcommand*{\Z}{\Zboson\xspace}


%%% Energy
%\newcommand*{\fourfgev}{\ensuremath{\qty{450}{\GeV}}\xspace}
%\newcommand*{\ninehgev}{\ensuremath{\qty{900}{\GeV}}\xspace}
%\newcommand*{\seventev}{\ensuremath{\qty{7}{\TeV}}\xspace}
%\newcommand*{\eighttev}{\ensuremath{\qty{8}{\TeV}}\xspace}
%\newcommand*{\sqsthirt}{\ensuremath{\sqrt{s} = \qty{13}{\TeV}}\xspace}
%\newcommand*{\sqsfourt}{\ensuremath{\sqrt{s} = \qty{14}{\TeV}}\xspace}
%\newcommand*{\sqsTev}{\ensuremath{\qty{1.96}{\TeV}}\xspace}
%\newcommand*{\threehtev}{\ensuremath{\qty{3.5}{\TeV}}\xspace}
%\newcommand*{\sqsnineh}{\ensuremath{\sqrt{s} = \ninehgev}\xspace}
%\newcommand*{\sqsseven}{\ensuremath{\sqrt{s} = \seventev}\xspace}
%\newcommand*{\sqseight}{\ensuremath{\sqrt{s} = \eighttev}\xspace}
%\newcommand*{\bunchSp}{\ensuremath{\tau_{\text{bunch}}}\xspace}


%%% Runs
%\newcommand*{\RunOne}{Run 1\xspace}
%\newcommand*{\RunTwo}{Run 2\xspace}
%\newcommand*{\RunThr}{Run 3\xspace}
%\newcommand*{\pp}{\ensuremath{pp}\xspace} -> atlasparticle.sty


%%% Luminosity
%\newcommand*{\invpb}{~\ensuremath{\text{pb}^{-1}}\xspace}
%\newcommand*{\invnb}{~\ensuremath{\text{nb}^{-1}}\xspace}
%\newcommand*{\invfb}{~\ensuremath{\text{fb}^{-1}}\xspace}
%\newcommand*{\invmub}{~\ensuremath{\mu\text{b}^{-1}}\xspace}
%\newcommand*{\cmSqSec}{cm\ensuremath{^{-2}}s\ensuremath{^{-1}}\xspace}
%\newcommand*{\lumi}{\ensuremath{\mathcal L}\xspace}
%\newcommand*{\calL}{\lumi}
%\newcommand*{\dldz}{\ensuremath{d\lumi/dz}\xspace}
%\newcommand*{\lofz}{\ensuremath{{\mathcal L}(z)}\xspace}
%\newcommand*{\sigyofz}{\ensuremath{\sigma_y(z)}\xspace}
%\newcommand*{\Lsp}{\ensuremath{{\mathcal L}_{sp}}\xspace}
%\newcommand*{\StoB}{\ensuremath{S/B}\xspace}


%%% Detectors -> atlasmisc
%\newcommand*{\radlength} {\ensuremath{X_0}}
%\newcommand*{\Lone}      {\texttt{L1}}
%\newcommand*{\HLT}       {\texttt{HLT}}
%\newcommand*{\ID}        {\texttt{ID}}
%\newcommand*{\Pixel}     {\texttt{Pixel}}
%\newcommand*{\SCT}       {\texttt{SCT}}
%\newcommand*{\TRT}       {\texttt{TRT}}
%
%\newcommand*{\LHC}       {LHC} - Dropped
%\newcommand*{\EMB}       {\texttt{EMB}}
%\newcommand*{\EME}       {\texttt{EME}}
%\newcommand*{\EMEC}      {\texttt{EMEC}}
%\newcommand*{\FCAL}      {\texttt{FCAL}}
%\newcommand*{\Cryo}      {\texttt{Cryo}}
%\newcommand*{\Gap}       {\texttt{Gap}}
%\newcommand*{\Scint}     {\texttt{Scint}}
%\newcommand*{\HEC}       {\texttt{HEC}}
%\newcommand*{\LAr}       {\texttt{LAr}}
%\newcommand*{\FCal}      {\texttt{FCal}}
%\newcommand*{\Tile}      {\texttt{Tile}}
%\newcommand*{\TileExt}   {\texttt{TileExt}}
%\newcommand*{\TileBar}   {\texttt{TileBar}}
%\newcommand*{\MBTS}      {\texttt{MBTS}}
%\newcommand*{\Presampler}{\texttt{Presampler}}


%%% Event generator stuff
%%\newcommand*{\Herwig}{\texttt{HERWIG}\xspace}
%\newcommand*{\herwigpp}{\texttt{HERWIG}++\xspace}
%%\newcommand*{\Herwigpp}{\herwigpp}
%\newcommand*{\geant}{\texttt{GEANT}4\xspace}
%\newcommand*{\jimmy}{\texttt{Jimmy}\xspace}
%\newcommand*{\Alpgen}{\texttt{Alpgen}\xspace}
%\newcommand*{\Sherpa}{\texttt{SHERPA}\xspace}
%\newcommand*{\Comphep}{\texttt{CompHep}\xspace}
%\newcommand*{\Madgraph}{\texttt{MadGraph}\xspace}
%\newcommand*{\Pythia}{\texttt{PYTHIA}\xspace}
%\newcommand*{\Powheg}{\texttt{POWHEG}\xspace}
%\newcommand*{\PowhegBox}{\texttt{POWHEG-BOX}\xspace}
%\newcommand*{\PowPythia}{\texttt{POWHEG}+\Pythia}
%\newcommand*{\Perugia}{\texttt{Perugia}\xspace}
%\newcommand*{\Prospino}{\texttt{Prospino}\xspace}
%\newcommand*{\Mcatnlo}{\texttt{MC@NLO}\xspace}
%\newcommand*{\Acermc}{\texttt{ACERMC}\xspace}
%\newcommand*{\PythiaP}{\Pythia (\Perugia 2010)\xspace}

%\newcommand*{\LO}        {\ensuremath{\text{LO}}\xspace}
%\newcommand*{\NLO}       {\ensuremath{\text{NLO}}\xspace}
%\newcommand*{\NLL}       {\ensuremath{\text{NLL}}\xspace}
%\newcommand*{\NNLO}      {\ensuremath{\text{N}}\NLO\xspace}
%\newcommand*{\muF}       {\ensuremath{\mu_\textsc{f}}\xspace}
%\newcommand*{\muR}       {\ensuremath{\mu_\textsc{r}}\xspace}


\noindent The macro \Macro{etaRange} produces what you would expect:
\verb|\etaRange{-2.5}{+2.5}| produces \etaRange{-2.5}{+2.5} while
\verb|\AetaRange{1.0}| produces \AetaRange{1.0}.
The macro \Macro{avg} can be used for average values:
\verb|\avg{\mu}| produces \avg{\mu}.


\newpage
%-------------------------------------------------------------------------------
\section{\File{atlasmath.sty}}

Turn on including these definitions with the option \Option{math=true} and off with the option \Option{math=false}.

%
% +--------------------------------------------------------------------+
% |                                                                    |
% |  "Box-squared" operator, as in Klein-Gordon. Command is "\boxsq".  |
% |                                                                    |
% +--------------------------------------------------------------------+
%
\newbox\boxsqbox
\newdimen\boxsize\boxsize=1.2ex%
\def\boxop{%
\setbox\boxsqbox=\vbox{\hrule depth0.8pt width0.8\boxsize height0pt%
                       \kern0.8\boxsize
                       \hrule height0.8pt width0.8\boxsize depth0pt}%
           \hbox{%
           \vrule height1.0\boxsize width0.8pt depth0pt%
           \copy\boxsqbox
           \vrule height1.0\boxsize width0.8pt depth0pt\kern1.5pt}}%
\newcommand*{\boxsq}{\ensuremath{\boxop^2}\xspace}
% +--------------------------------------------------------------------+
% |                                                                    |
% |  Theoretical notations                                             |
% |                                                                    |
% +--------------------------------------------------------------------+
%
\newcommand{\spinor}[1]{\ensuremath{\left(\begin{matrix}#1_1\cr#1_2\cr#1_3\cr#1_4\cr\end{matrix}\right)}}
\def\pmb#1{\setbox0=\hbox{$#1$}%  This is "poor man's boldface".
  \kern-.025em\copy0\kern-1.0\wd0%
  \kern.05em\copy0\kern-1.0\wd0%
  \kern-.025em\raise.0433em\box0}
\newcommand*{\grad}{\pmb{\nabla}\xspace}


\noindent The macro \Macro{spinor} is also defined.
\verb|\spinor{u}| produces \spinor{u}.


\newpage
%-------------------------------------------------------------------------------
\section{\File{atlasother.sty}}

Turn on including these definitions with the option \Option{other} and off with the option \Option{other=false}.

\begin{xtabular}{ll}
\verb|\etpt| & \etpt \\
\verb|\etptsig| & \etptsig \\
\verb|\begL| & \begL \\
\verb|\lowL| & \lowL \\
\verb|\highL| & \highL \\
\verb|\Epsb| & \Epsb \\
\verb|\Epsc| & \Epsc \\
\verb|\Mtau| & \Mtau \\
\verb|\swsq| & \swsq \\
\verb|\swel| & \swel \\
\verb|\swsqb| & \swsqb \\
\verb|\swsqon| & \swsqon \\
\verb|\gv| & \gv \\
\verb|\ga| & \ga \\
\verb|\gvbar| & \gvbar \\
\verb|\gabar| & \gabar \\
\verb|\Zzv| & \Zzv \\
\verb|\Abb| & \Abb \\
\verb|\Acc| & \Acc \\
\verb|\Aqq| & \Aqq \\
\verb|\Afb| & \Afb \\
\verb|\GZ| & \GZ \\
\verb|\GW| & \GW \\
\verb|\GH| & \GH \\
\verb|\GamHad| & \GamHad \\
\verb|\Gbb| & \Gbb \\
\verb|\Rbb| & \Rbb \\
\verb|\Gcc| & \Gcc \\
\verb|\Gvis| & \Gvis \\
\verb|\Ginv| & \Ginv \\
\end{xtabular}

