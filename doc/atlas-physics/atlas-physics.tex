%-------------------------------------------------------------------------------
% This note lists the symbols defined in atlasphysics.sty.
%-------------------------------------------------------------------------------
\documentclass{style/atlasdoc}

\usepackage[default]{style/atlaspackage}
% Include all style files
\usepackage[process,other]{style/atlasphysics}
%\usepackage{mathrsfs}

\usepackage[backend=biber,
	style=numeric-comp,sorting=none,block=ragged,firstinits=true]{biblatex}
\addbibresource{atlas-physics.bib}

\graphicspath{{../../figures/}}

%-------------------------------------------------------------------------------
% Generic document information
%-------------------------------------------------------------------------------

% Set author and title for the PDF file
\hypersetup{pdftitle={ATLAS LaTeX guide},pdfauthor={Ian Brock}}

\title{Symbols defined in \texttt{atlasphysics.sty}}

\author{Ian C. Brock}

\draftversion{00-95-00}

\abstracttext{%
  This note lists the symbols defined in \texttt{atlasphysics.sty}.
  These provide examples of how to define your own symbols, as well as many symbols
  that are often used in ATLAS documents.
}

%-------------------------------------------------------------------------------
% This is where the document really begins
%-------------------------------------------------------------------------------
\begin{document}

\tableofcontents

%-------------------------------------------------------------------------------
\section{The \texttt{atlasphysics.sty} style file}
\label{app:AtlasPhysicsSty}
%-------------------------------------------------------------------------------

The \texttt{atlasphysics.sty} style file implements a series of useful
shortcuts to typeset a physics paper, such as particle
symbols. It can included in the preamble of your paper with the usual
syntax:
%
\begin{verbatim}
  \usepackage{style/atlasphysics}
\end{verbatim}
%
As of version 01-00-00 the file is actually split into smaller files,
which can be included or not using options.
The following options are available:
\begin{description}
\item[BSM -- off] BSM and SUSY particles;
\item[math -- on] Maths
\item[misc -- on] Miscellaneous
\item[other -- off] Definitions that used to be in \texttt{atlasphysics.sty}, 
	but are probably too specialised to be needed by most people;
\item[particles -- on] Standard Model particles and some combinations;
\item[process -- off] Some example processes. 
	These are not included by default as the current choice is rather arbitrary
	and certainly not complete;
\item[units -- off] Units that used to be defined -- not needed if you use \texttt{siunitx}.
\end{description}

All definitions are done in a consistent way using \verb|\newcommand*|.
All definitions use \verb|\ensuremath| where appropriate and are terminated with
\verb|\xspace|, so you can simply write ``\verb|\ttbar production|'' instead of
``\verb|\ttbar\ production|'' or ``\verb|\ttbar{} production|'' to get ``\ttbar production''.

See Appendix~\ref{sec:old} for details on changes that were introduced when
when going from version 00-04-05 of \texttt{atlasnote}
to version 01-00-00 of \texttt{atlaslatex}.
Let me know if you spot some other changes that are not documented here!

The style file contains among other things:

\medskip

\begin{tabular}{llcllcll}
  \verb+\lapprox+ & \lapprox{} & \hspace{1cm} &
  \verb+\rapprox+ & \rapprox{}  &\hspace{1cm} &
  \verb+\rts+  & \rts{} \\
  \verb+\Ecm+ & \Ecm{} & &
  \verb+\stat+ & \stat{} & &
  \verb+\syst+ & \syst{} \\
\end{tabular}

\medskip

\begin{tabular}{llcllcll}
  \verb+\Zboson+ & \Zboson{} & \hspace{5mm} &
  \verb+\Wboson+ & \Wboson{} & \hspace{5mm} &
  \verb+\Wplus+ & \Wplus{} \\
  \verb+\Wminus+ & \Wminus{} & &
  \verb+\Wpm+ & \Wpm{} & &
  \verb+\Wmp+ & \Wmp{} \\
  \verb+\Afb+ & \Afb{} & &
  \verb+\GW+ & \GW{} & &
  \verb+\GZ+ & \GZ{} \\
  \verb+\Wln+ & \Wln{} & &
  \verb+\Zll+ & \Zll{} & &
  \verb+\Zee+ & \Zee{} \\
  \verb+\Zmm+ & \Zmm{} & &
  \verb+\mZ+ & \mZ{} \\
  \verb+\mW+ & \mW{} & &
  \verb+\mH+ & \mH{} \\
  \verb+\Mtau+ & \Mtau{} & &
  \verb+\swsq+ & \swsq{} & &
  \verb+\swel+ & \swel{} \\
  \verb+\swsqb+ &  \swsqb{} & &
  \verb+\swsqon+ & \swsqon{} & &
  \verb+\gv+ &  \gv{} \\
  \verb+\ga+ & \ga{} & &
  \verb+\gvbar+ & \gvbar{} & &
  \verb+\gabar+ & \gabar{} \\
  \verb+\Zprime+ & \Zprime{} & &
  \verb+\Hboson+ & \Hboson{} & & 
  \verb+\GH+ & \GH{} \\
\end{tabular}

\medskip

\noindent The command \verb+\Zzero+ is identical to \verb+\Zboson+.

\medskip

\begin{tabular}{llcllcll}
  \verb+\tbar+ & \tbar{} & \hspace{1cm} &
  \verb+\ttbar+ & \ttbar{} & \hspace{1cm} &
  \verb+\bbar+ & \bbar{} \\
  \verb+\bbbar+ & \bbbar{} & &
  \verb+\cbar+ & \cbar{} & &
  \verb+\ccbar+ & \ccbar{} \\
  \verb+\sbar+ & \sbar{} & &
  \verb+\ssbar+ &  \ssbar{} & &
  \verb+\ubar+ & \ubar{} \\
  \verb+\uubar+ & \uubar{} & &
  \verb+\dbar+ & \dbar{} & &
  \verb+\ddbar+ & \ddbar{} \\
  \verb+\fbar+ & \fbar{} & &
  \verb+\ffbar+ &  \ffbar{} & &
  \verb+\qbar+ & \qbar{} \\
  \verb+\qqbar+ & \qqbar{} & &
  \verb+\nbar+ & \nbar{} & &
  \verb+\nnbar+ & \nnbar{} \\
  % \verb+\e+ & \e{} & &
  \verb+\ee+ & \ee{} & &
  \verb+\mumu+ & \mumu{} & &
  \verb+\tautau+ & \tautau{} \\
  \verb+\epm+ & \epm{} & &
  % \verb+\epem+ & \epem{} & &
  \verb+\leplep+ & \leplep{} & & 
  \verb+\lnu+ & \lnu{} \\
  % \verb+\ellell+ & \ellell{} & & & \\
\end{tabular}

\medskip

\begin{tabular}{llcllcll}
  \verb+\BoBo+ & \BoBo{} & \hspace{1cm} &
  \verb+\BodBod+ & \BodBod{} & \hspace{1cm} &
  \verb+\BosBos+ & \BosBos{} \\
  \verb+\Bd+ & \Bd{} & &
  \verb+\Bs+ & \Bs{} & &
  \verb+\Bu+ & \Bu{} \\
  \verb+\Bc+ & \Bc{} & &
  \verb+\Lb+ & \Lb{} & &
  \verb+\jpsi+ & \jpsi{} \\
  \verb+\Jpsi+ & \Jpsi{} & &
  \verb+\Jee+ & \Jee{} & &
  \verb+\Jmm+ & \Jmm{} \\
  \verb+\psip+ & \psip{} & &
  \verb+\Kzero+ & \Kzero{} & &
  \verb+\Kzerobar+ & \Kzerobar{} \\
  \verb+\kaon+ & \kaon{} & &
  \verb+\Kplus+ & \Kplus{} & &
  \verb+\Kminus+ & \Kminus{} \\
  \verb+\Klong+ & \Klong{} & &
  \verb+\Kshort+ & \Kshort{} & &
  \verb+\Ups+ & \Ups{} \\
\end{tabular}

\medskip

\begin{tabular}{llcllcllcll}
  \verb+\alphas+ & \alphas{} & \hspace{1cm} &
  \verb+\Lms+ & \Lms{} & \hspace{1cm} &
  \verb+\Lmsfive+ & \Lmsfive{} & \hspace{1cm} &
  \verb+\KT+ & \KT{} \\
\end{tabular}

\medskip

\begin{tabular}{llcllcll}
  \verb+\Vud+ & \Vud{} & \hspace{1cm} &
  \verb+\Vus+ & \Vus{} & \hspace{1cm} &
  \verb+\Vub+ & \Vub{} \\
  \verb+\Vcd+ & \Vcd{} &  &
  \verb+\Vcs+ & \Vcs{} &  &
  \verb+\Vcb+ & \Vcb{} \\
  \verb+\Vtd+ & \Vtd{} & &
  \verb+\Vts+ & \Vts{} & & 
  \verb+\Vtb+ & \Vtb{} \\
\end{tabular}

\medskip

\begin{tabular}{llcllcll}
  \verb+\Azero+ & \Azero{} & \hspace{1cm} &
  \verb+\hzero+ & \hzero{} & \hspace{1cm} &
  \verb+\Hzero+ & \Hzero{} \\
  \verb+\Hplus+ & \Hplus{} & &
  \verb+\Hminus+ & \Hminus{} & &
  \verb+\Hpm+ & \Hpm{} \\
  % \verb+\Hmp+ \Hmp{}
\end{tabular}

\medskip

\noindent A generic macro \verb+\susy#1+ is defined, so that for
example \verb+\susy{q}+ produces \susy{q} and similar.

\medskip

\begin{tabular}{llcllcll}
  \verb+\chinop+ & \chinop{} & \hspace{1cm} &
  \verb+\chinotwom+ & \chinotwom{} & \hspace{1cm} &
  \verb+\chinopm+ & \chinopm{} \\
  \verb+\nino+ & \nino{} & &
  \verb+\ninothree+ & \ninothree{} & &
  \verb+\gravino+ & \gravino{} \\
  \verb+\squark+ & \squark{} & &
  \verb+\gluino+ & \gluino{} & &
  \verb+\slepton+ & \slepton{} \\
  \verb+\stop+ & \stop{} & &
  \verb+\stopone+ & \stopone{} & &
  \verb+\stopL+ & \stopL{} \\
  \verb+\sbottom+ & \sbottom{} & &
  \verb+\sbottomtwo+ & \sbottomtwo{} & &
  \verb+\sbottomR+ & \sbottomR{} \\
  \verb+\sleptonL+ & \sleptonL{} & &
  \verb+\sel+ & \sel{} & &
  \verb+\smuR+ & \smuR{} \\
  \verb+\stauone+ & \stauone{} & &
  \verb+\snu+ & \snu{} & &
  \verb+\squarkR+ & \squarkR{} \\
\end{tabular}

\medskip

\noindent For \susy{q}, \susy{t}, \susy{b}, \slepton, \sel, \smu and
\stau, L and R states are defined; for stop, sbottom and stau also the
light (1) and heavy (2) states. There are four neutralinos and two
charginos defined, the index number unfortunately needs to be written
out completely. For the charginos the last letter(s) indicate(s) the
charge: p for +, m for -, and pm for $\pm$.

\medskip

\begin{tabular}{llcllcll}
  \verb+\pt+ & \pt{} & \hspace{1cm} &
  \verb+\pT+ & \pT{} & \hspace{1cm} &
  \verb+\et+ & \et{} \\
  \verb+\eT+ & \eT{} & &
  \verb+\ET+ & \ET{} & &
  \verb+\HT+ & \HT{} \\
  \verb+\ptsq+ & \ptsq{} & &
  \verb+\met{}+ & \met{} & &
\end{tabular}

\medskip

\noindent Use \verb+\met{}+ rather than just \verb+\met+ to get the spacing
right. In principle this works for any macro, although in most cases it will
not be needed as {\tt xspace.sty} will take care of the spacing. Somehow
{\tt xspace.sty} doesn't do a good job for \met.

\vspace{5mm}

\begin{tabular}{llcllcll}
\verb+\ifb+ & \ifb{} & \hspace{1cm} &
\verb+\ipb+ & \ipb{} & \hspace{1cm} &
\verb+\inb+ & \inb{} \\
\verb+\TeV+ & \TeV{} & &
\verb+\GeV+ & \GeV{} & &
\verb+\MeV+ & \MeV{} \\
\verb+\keV+ & \keV{} & &
\verb+\eV+ & \eV{} & & & \\
\end{tabular}

\medskip

\noindent And \verb+\tev+, \verb+\gev+, \verb+\mev+, \verb+\kev+, and
\verb+\ev+ have the same results.

\medskip

\noindent A generic macro \verb+\mass#1+ is defined, so that for example
\verb+\mass{\mu}+ produces \mass{\mu} and similar.
\verb+\twomass{\mu e}+ will produce \twomass{\mu e}.


%-------------------------------------------------------------------------------
\section{Old macros}
\label{sec:old}
%-------------------------------------------------------------------------------

With the introduction of \texttt{atlaslatex} several macro names have been changed to make them more consistent.
A few have been removed. The chnages include:
\begin{itemize}
\item Kaons now have a capital ``K'' in the macro name, e.g.\ \verb|\Kplus| for \Kplus;
\item \verb|\Ztau|, \verb|{\Wtau}|, \verb|{\Htau}| \verb|{\Atau}| have been replaced by
	\verb|\Ztautau|, \verb|{\Wtautau}|, \verb|{\Htautau}| \verb|{\Atautau}|;
\item \verb|\Ups| replaces \verb|\ups|;
	the use of \verb|\ups| to produce $\Upsilon$ in text mode has been removed;
\end{itemize}

Quite a few macros are more related to \Zboson physics than they are to LHC physics and have
been moved to the \texttt{atlasother.sty} file, which is not included by default.
There are also macros for various decay processes, \texttt{atlasprocess.sty} which are not included by default,
but may be useful for how you can define your favourite process.

\end{document}
